\chapter{Theoretical Framework}

\textcolor{black}{Integrated information theory (IIT) postulates that
consciousness is identical to integrated information and that a system's
capacity for consciousness can be expressed by a quantitative measure
denoted by $\phi$}. \textcolor{black}{Tononi defines integrated information as ``the amount
of information generated by a complex of elements, above and beyond the information generated 
by its parts'' \cite{tononi2008consciousness}
and states, }\textcolor{black}{\emph{``The integrated information theory}}\textcolor{black}{{} (IIT) 
of consciousness
claims that, at a fundamental level, consciousness is integrated information'' 
\cite{tononi2008consciousness} (italics in original).} \textcolor{black}{IIT aims to explain 
``relationships between consciousness and the Physical Substrate of Consciousness (PSC), 
and starts from essential properties of phenomenal experience, and derives the
requirements for the physical substrate of consciousness.'' \cite{tononi2016}}

\textcolor{black}{To rigorously formulate this proposal, IIT introduces a mathematical framework 
for quantifying the integration of information within a system. The foundation of this formalism 
is built upon the calculus of $\phi$, which we describe in the next section.}

\subsection*{\textcolor{black}{Calculus of $\phi$}}

\textcolor{black}{The integrated information theory defines integrated information ($\phi$) 
as the effective information of the minimum information partition (MIP) in a system 
\cite{oizumi2014fromthe, oizumi2014phenomenology, tononi2004information, tononi2016}. 
The MIP is also defined as the partition having minimum effective information 
among all possible partitions.}

\begin{center}
\textcolor{black}{$\phi[X;x]=:\varphi[X;x,MIP(x)]$}
\par\end{center}

\begin{center}
\textcolor{black}{ $MIP(x)=:argmin{\varphi(X;x,P)}$}
\par\end{center}

\textcolor{black}{Where }\textcolor{black}{\emph{$X$}}\textcolor{black}{{} is the system, 
$x$ is a state, and $P$ is a partition $P=M_{1}, \ldots ,M_{r}$.}

\textcolor{black}{Importantly, identifying the MIP requires searching all possible
partitions and comparing their effective information $\phi$. 
This effective information is specified in terms of effect and causal information, 
that is, the distance between two probability distributions: one for the unpartitioned 
(unconstrained) partition (this can be the full set of nodes of the whole system 
or one of its possible partitions) and a partition of this latter. Such probability 
distributions determine probabilities  of all possible future (effect) or past 
(causal) states of an arbitrary partition being in a current state. 
This means that comparing one set of nodes that can be the full set of nodes 
of the system or a subset (partition) of itself with all possible partitions 
of this set of nodes, MIP represents the partitions with the minimal value of 
the distance between probability distributions of the set of nodes and one of 
all its possible partitions.}

\textcolor{black}{When a set of nodes is chosen to compute effective information, 
this is referred to as a `mechanism', and the partition to which it is compared 
is referred to as the `purview'. The distance between probability distributions 
is computed by means of an adaptation of the Earth Mover's Distance (EMD) algorithm, 
which is a method to evaluate dissimilarity between two multi-dimensional 
distributions in a given feature space where a distance measure between single 
features, which we call the ground distance, is given. The EMD ``lifts'' this 
distance from individual features to full distributions. Note that EMD is referred 
to as a Wasserstein metric in mathematics, and is commonly used in machine learning 
as a natural metric between two distributions~\cite{villani}.}

\textcolor{black}{Intuitively, given two distributions, one can be seen as a
 mass of earth properly spread in space, the other as a collection
of holes in that same space. Then, the EMD measures the least amount of 
work needed to fill the holes with earth. Here, a unit of work corresponds 
to transporting (by an optimal transport method) a unit of earth a unit of 
ground distance.}

\textcolor{black}{While the calculus of $\phi$ provides the quantitative backbone of IIT, 
its explanatory strength lies in its grounding in phenomenological principles. 
We now turn to the conceptual framework that connects this formalism to the nature 
of consciousness.}

\subsection*{\textcolor{black}{IIT’s Consciousness Framework}}

\textcolor{black}{Integrated Information Theory (IIT) provides a theoretical framework for consciousness by defining it as the capacity of a system to integrate information \cite{tononi2004information,tononi2016}. IIT is grounded in five phenomenological axioms: existence (consciousness exists), composition (it is structured), information (it is specific), integration (it is unified), and exclusion (it is definite) \cite{oizumi2014phenomenology}. These axioms imply that a conscious system must generate information through integrated interactions among its components, quantified by $\phi$, the degree of irreducible information \cite{tononi2008consciousness}. The Physical Substrate of Consciousness (PSC), typically neural networks, must support these properties, with $\phi$ measuring the system’s capacity to produce a unified conscious experience. IIT’s framework has been applied to model neural correlates of consciousness, distinguishing conscious from unconscious states \cite{sciencetranslationmedicine}. However, computing $\phi$ is computationally intensive, limiting its practical use. The $\phi_K$ metric, introduced in this thesis, leverages algorithmic probability to approximate integration more efficiently, enhancing IIT’s applicability to cognitive science research \cite{mainbook}.}

\subsection*{\textcolor{black}{Cognitive Science Applications}}

\textcolor{black}{IIT’s framework, augmented by $\phi_K$, offers significant applications in cognitive science, particularly in understanding consciousness. For example, IIT has been used to assess disorders of consciousness (e.g., coma, vegetative states) by measuring $\phi$ in neural systems via perturbation analysis \cite{sciencetranslationmedicine}. $\phi_K$ improves this by reducing computational demands, enabling real-time analysis of brain networks during tasks like attention or perception \cite{mainbook}. In neural modeling, $\phi_K$ facilitates simulations of integrated information in large-scale networks, aiding research into cognitive processes such as working memory \cite{dayan}. Additionally, $\phi_K$’s algorithmic probability approach allows for studying consciousness in artificial systems, exploring whether AI can exhibit integrated information akin to human cognition \cite{ruffini}. These applications strengthen IIT’s empirical utility, making $\phi_K$ a vital tool for cognitive science investigations into the neural and computational basis of consciousness \cite{tononi2016}.}

\textcolor{black}{The effectiveness of $\phi_K$ in these contexts rests on a deeper connection between information integration and the algorithmic nature of complex systems.}

\subsection*{\textcolor{black}{Algorithmic Probability in Cognitive Science}}

\textcolor{black}{Algorithmic Information Theory (AIT), originating with Turing’s concept of a universal computing machine, provides a foundation for understanding intelligence and complexity in cognitive science \cite{turing1950}. Turing’s machine, capable of simulating any computation, underpins AIT’s measure of complexity via Kolmogorov’s minimal program length \cite{kolmogorov}. Zenil’s Coding Theorem Method (CTM) and Block Decomposition Method (BDM) advance this by analyzing integration in complex systems, revealing how interactions yield emergent phenomena \cite{zenilaid,bdm}. This thesis, inspired by CTM and BDM’s philosophical core—where small rules govern complex behaviors—proposes $\phi_K$ to quantify integration without directly applying these methods \cite{mainbook}. Grounded in the causal principle that simple programs produce neural dynamics, $\phi_K$ leverages algorithmic probability to model consciousness under the context of IIT, revealing fractal patterns in information spread akin to schemata in information theory \cite{shannon1948}. These characteristics align with IIT’s view of the Physical Substrate of Consciousness, enhancing cognitive science’s exploration of emergent conscious phenomena \cite{oizumi2014phenomenology}.}
