\chapter{Conclusions}
\textcolor{black}{Here we have sketched connections and developed first approaches towards a 
calculus to $\phi$ metrics based on algorithmic perturbation analysis, which in turn has a 
solid mathematical foundation that must be further studied. Our computational approach 
targeted what is referred to as the IIT 3.0, defined as a calculus of probability distributions. 
Instead of considering distances between statistical distributions, we formulated 
the problem as a distance in an algorithmic complexity space, properly approximated, 
in response to perturbations of the system and introduced a meta-test whose answers may 
provide a guidance on the algorithmic complexity and integrated information of the system. 
More exploration of the theoretical and practical connections between these theories are 
still needed.}

\textcolor{black}{Interestingly, such a perturbation programmability test--initially 
inspired by the Turing test (establishing another interesting connection between these n
ew theories of consciousness and past ones)-- as applied to physical systems, is a working 
strategy to find explanations for the behaviour of systems. It remains for future work to 
make conceptual and computational connections to what Oizumi and Tononi et al. called the MIP 
(Minimum Information Partition)~\cite{oizumi2014fromthe} of a system. Having this first 
version of $\phi_{K}$, we conjecture that MIP definitions also obey and are connected to 
algorithmic complexity in about the same way, as they should remain based on rules of an 
algorithmic nature. Thus, the next step is to go further in the application of the test 
introduced in this paper to discover simple rules that would help to find MIP in a more 
natural and a faster way. Another possible direction is to systematize the finding of 
these simple rules and apply more powerful methods to enable computation of larger systems. 
However, here we have merely established the first principles and the directions that can be 
explored following these ideas.}

Finally, we think that these ideas about self-explanatory systems capable of providing 
answers to questions about their own behaviour can help in devising techniques to make 
other methods, in areas such as machine and deep learning, explain their own, often obscure, 
behaviour.

\textcolor{black}{While the primary contributions of this work are theoretical and computational, they resonate with deeper philosophical themes. The development of $\phi_K$ intersects with questions traditionally explored in cognitive philosophy, such as the nature of emergence, the unity of conscious experience, and the explanatory power of algorithmic representations. By modeling the way systems integrate information through minimal algorithmic descriptions and perturbative self-querying, this thesis gestures toward a mechanistic account of how subjectivity and self-coherence may arise in both biological and artificial systems. Future interdisciplinary work could further investigate whether these algorithmic frameworks serve not only as tools for analysis but as candidates for foundational explanatory models of consciousness.}

