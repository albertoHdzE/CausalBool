%%%%%%%%%%%%%%%%%%%%%%%%%%%%%%%%%%%%%%%%%%%%%%%%%%%%%%%%%%%%%%%%%%%%%%%%%%%%%%%%
% Add some citations in the text using \cite{key}
% Make sure bibliography entries exist in Bibliografia/referencias.bib
% Run LaTeX multiple times to resolve cross-references
% Consider using the bookmark package for PDF bookmarks
%%%%%%%%%%%%%%%%%%%%%%%%%%%%%%%%%%%%%%%%%%%%%%%%%%%%%%%%%%%%%%%%%%%%%%%%%%%%%%%%
%                         FORMATO DE TESIS UMSNH                               %
%%%%%%%%%%%%%%%%%%%%%%%%%%%%%%%%%%%%%%%%%%%%%%%%%%%%%%%%%%%%%%%%%%%%%%%%%%%%%%%%
% based on Harish Bhanderi's PhD/MPhil template, then Uni Cambridge
% http://www-h.eng.cam.ac.uk/help/tpl/textprocessing/ThesisStyle/
% corrected and extended in 2007 by Jakob Suckale, then MPI-iCBG PhD programme
% and made available through OpenWetWare.org - the free biology wiki

%                     Under GNU License v3
% Adaptado para UNAM: @Tepexic
% ADAPTADO PARA UMSNH:  @arturolp

\documentclass[twoside,11pt]{Latex/Classes/PhDthesisPSnPDF} % "thesisUMSNH" para formato de la UMSNH
%         PUEDEN INCLUIR EN ESTE ESPACIO LOS PAQUETES EXTRA, O BIEN, EN EL ARCHIVO "PhDthesisPSnPDF.cls" EN "./Latex/Classes/"

% Estos paquetes son opcionales y a necesidad de cada quien:
\usepackage[T1]{fontenc}
\usepackage[utf8]{inputenc}
\usepackage[english]{babel}
\usepackage{blindtext}                             % Para insertar texto dummy, de ejemplo, pues.
\usepackage{amssymb, amsmath, amsbsy, amsfonts}    % ECUACIONES Y SÍMBOLOS MATEMÁTICOS
\usepackage{listings}                    % PERMITE AGREGAR CÓDIGO DE LENGUAJES  DE PROGRAMACIÓN (DOCUMENTACIÓN EN GOOGLE)
\usepackage{mathdots}                    % para el comando \iddots
\usepackage{mathrsfs}                    % para formato de letra en ecuaciones
\usepackage[round, sort, numbers]{natbib}  % Personalizar la bibliografía a gusto de cada quien
\usepackage{pdflscape}
\usepackage{setspace} % Add this to your preamble if not already included
% \usepackage{algorithm}
% \usepackage{algorithmic}
% Add this to your preamble or at the beginning of the file
% Definir el color gray
\usepackage{xcolor}
\definecolor{gray}{rgb}{0.5, 0.5, 0.5}
\definecolor{grayer}{rgb}{0.95, 0.95, 0.95}
% Define a custom style for Mathematica code
\lstdefinestyle{MathematicaStyle}{
    language=Mathematica,
    breaklines=true,
    breakatwhitespace=false,
    breakindent=1em,
    basicstyle=\ttfamily\scriptsize,
    columns=flexible,
    linewidth=0.9\linewidth,
    frame=single,
    showstringspaces=false,
    numbers=left, % Added to show line numbers
    numberstyle=\tiny\color{gray}, % Style for line numbers
    backgroundcolor=\color{grayer}, % Changed to a lighter background color
    literate={+}{+}{1} {a}{a}{1} {b}{b}{1} {c}{c}{1} {d}{d}{1} {e}{e}{1} {f}{f}{1} {g}{g}{1} {h}{h}{1} {i}{i}{1} {j}{j}{1} {k}{k}{1} {l}{l}{1} {m}{m}{1} {n}{n}{1} {o}{o}{1} {p}{p}{1} {q}{q}{1} {r}{r}{1} {s}{s}{1} {t}{t}{1} {u}{u}{1} {v}{v}{1} {w}{w}{1} {x}{x}{1} {y}{y}{1} {z}{z}{1} {A}{A}{1} {B}{B}{1} {C}{C}{1} {D}{D}{1} {E}{E}{1} {F}{F}{1} {G}{G}{1} {H}{H}{1} {I}{I}{1} {J}{J}{1} {K}{K}{1} {L}{L}{1} {M}{M}{1} {N}{N}{1} {O}{O}{1} {P}{P}{1} {Q}{Q}{1} {R}{R}{1} {S}{S}{1} {T}{T}{1} {U}{U}{1} {V}{V}{1} {W}{W}{1} {X}{X}{1} {Y}{Y}{1} {Z}{Z}{1} {0}{0}{1} {1}{1}{1} {2}{2}{1} {3}{3}{1} {4}{4}{1} {5}{5}{1} {6}{6}{1} {7}{7}{1} {8}{8}{1} {9}{9}{1} {\}}{\}}{1},
    xleftmargin=1em,
    xrightmargin=1em
}
\usepackage[ruled,vlined]{algorithm2e}



% Note:
% The \blindtext or \Blindtext commands throughout this template generate dummy text
% to fill the template out. These commands should all be removed when 
% writing thesis content.


\include{Latex/Macros/MacroFile1}           % Archivo con funciones útiles





%%%%%%%%%%%%%%%%%%%%%%%%%%%%%%%%%%%%%%%%%%%%%%%%%%%%%%%%%%%%%%%%%%%%%%%%%%%%%%%%
%                                   DATOS                                      %
%%%%%%%%%%%%%%%%%%%%%%%%%%%%%%%%%%%%%%%%%%%%%%%%%%%%%%%%%%%%%%%%%%%%%%%%%%%%%%%%
\title{Estimations of Integrated Information Based on Algorithmic Complexity and Dynamic Querying}
\author{Alberto Hernández Espinosa} 
\facultad{Facultad de Filosofia}                 % Nombre de la facultad/escuela
\escudofacultad{Latex/Classes/Escudos/fi_azul} % Aquí ponen la ruta y nombre del escudo de su facultad, actualmente, la carpeta Latex/Classes/Escudos cuenta con los siguientes escudos:
% "fi_azul" Facultad de ingenieria en color azul
% "fi_negro" Facultad de ingenieria en color negro
% "fc_azul" Facultad de ciencias en color azul
% "fc_negro" Facultad de ciencias en color negro
% "fmed_grande" Facultad de medicina UMSNH
% Se agradecen sus aportaciones de escudos a jebus.velazquez@gmail.com

\degree{Doctor en Filosofía de la Ciencia}       % Carrera
\director{Hector Zenil}                 % Director de tesis
%\tutor{Francisco Hernandez Quiroz}                    % Tutor de tesis, si aplica
\degreedate{2025}                          % Año de la fecha del examen
\lugar{Ciudad Universitaria, CDMX}         % Lugar

%\portadafalse                              % Portada en NEGRO, descomentar y comentar la línea siguiente si se quiere utilizar
\portadatrue                                % Portada en COLOR



%% Opciones del posgrado (descomentar si las necesitan)
	%\posgradotrue                                                    
	%\programa{programa de maestría y doctorado en ingeniería}
	%\campo{Ingeniería Eléctrica - Control}
	%% En caso de que haya comité tutor
	%\comitetrue
	%\ctutoruno{Dr. Emmet L. Brown}
	%\ctutordos{Dr. El Doctor}
%% Datos del jurado                             
	%\presidente{Dr. 1}
	%\secretario{Dr. 2}
	%\vocal{Dr. 3}
	%\supuno{Dr. 4}
	%\supdos{Dr. 5}
	%\institucion{el Instituto de Ingeniería, UNAM}

\keywords{tesis,autor,tutor,etc}            % Palablas clave para los metadatos del PDF
\subject{tema_1,tema_2}                     % Tema para metadatos del PDF  

%%%%%%%%%%%%%%%%%%%%%%%%%%%%%%%%%%%%%%%%%%%%%%%%%%%%%
%                   PORTADA                         %
%%%%%%%%%%%%%%%%%%%%%%%%%%%%%%%%%%%%%%%%%%%%%%%%%%%%%
\begin{document}

\maketitle									% Se redefinió este comando en el archivo de la clase para generar automáticamente la portada a partir de los datos

%%%%%%%%%%%%%%%%%%%%%%%%%%%%%%%%%%%%%%%%%%%%%%%%%%%%%
%                  PRÓLOGO                          %
%%%%%%%%%%%%%%%%%%%%%%%%%%%%%%%%%%%%%%%%%%%%%%%%%%%%%
\frontmatter
\begin{dedication}
To all who have contributed to this journey that I call 'my life,' I am deeply grateful. Thank you.\\
\end{dedication}
       % Comentar línea si no se usa
%\chapter*{}
%\pagenumbering{Roman}

\begin{acknowledgements}

There is no space enough to put all the acknowledgements here. I'll to do it in person in a long talk, I promise.\\
%\blindtext % Dummy text
\end{acknowledgements}




   % Comentar línea si no se usa 
% ******************************* Thesis Declaration ********************************

\begin{declaration}

    I hereby declare that, except where specific reference is made to the work of others, the content of this thesis is original and has not been submitted, in whole or in part, for consideration for any other degree or qualification at this or any other university. This thesis is the result of my own work and does not include any material that is the outcome of collaborative work, unless specifically indicated in the text.
% Author and date will be inserted automatically from thesis.tex


\end{declaration}
           % Comentar línea si no se usa

% Thesis Abstract -----------------------------------------------------


%\begin{abstractslong}    %uncommenting this line, gives a different abstract heading
\begin{abstracts}        %this creates the heading for the abstract page

    Integrated information has been introduced as a metric to quantify the amount of 
    information generated by a system beyond the information generated by its individual elements. 
    While the metrics associated with the Greek letter $\phi$ require the calculation of the 
    interaction of an exponential number of sub-divisions of the system, most of these numerical 
    approaches related to the metric are based on the basics of classical information theory 
    and perturbation analysis. Here we introduce and sketch alternative approaches to connect 
    algorithmic complexity and integrated information based on the concept of algorithmic 
    perturbation rooted in algorithmic information dynamics and its concept of programmability. 
    We hypothesise that if an object is algorithmic random or algorithmic simple, algorithmic 
    random perturbations will have little to no effect to the internal capabilities of a system 
    to produce integrated information but when an object is more integrated the object will 
    also display elements able to perturb the object and increase or decrease its algorithmic 
    randomness. We sketch some of these ideas related to an object integrated information 
    value and its algorithmic information content. We propose that such an algorithmic 
    perturbation test quantifying compression sensitivity may provide a system with a means 
    to extract explanations--causal accounts--of its own behaviour hence making IIT and associated 
    measure $\phi$ more explainable and interpretable. Our technique may reduce the number of 
    calculations to arrive at some estimations with algorithmic perturbation guiding a more 
    efficient search. Our work sets the stage for a systematic exploration and further 
    investigation of the connections between algorithmic complexity and integrated information 
    at the level of both theory and practice.
% \blindtext

\end{abstracts}
%\end{abstractlongs}


% ----------------------------------------------------------------------                   % Comentar línea si no se usa

%%%%%%%%%%%%%%%%%%%%%%%%%%%%%%%%%%%%%%%%%%%%%%%%%%%%%
%                   ÍNDICES                         %
%%%%%%%%%%%%%%%%%%%%%%%%%%%%%%%%%%%%%%%%%%%%%%%%%%%%%
%Esta sección genera el índice
\setcounter{secnumdepth}{3} % organisational level that receives a numbers
\setcounter{tocdepth}{3}    % print table of contents for level 3
\tableofcontents            % Genera el índice 
%: ----------------------- list of figures/tables ------------------------
\listoffigures              % Genera el ínidce de figuras, comentar línea si no se usa
\listoftables               % Genera índice de tablas, comentar línea si no se usa


%%%%%%%%%%%%%%%%%%%%%%%%%%%%%%%%%%%%%%%%%%%%%%%%%%%%%
%                   CONTENIDO                       %
%%%%%%%%%%%%%%%%%%%%%%%%%%%%%%%%%%%%%%%%%%%%%%%%%%%%%
% the main text starts here with the introduction, 1st chapter,...
\mainmatter
\def\baselinestretch{1.5}                   % Interlineado de 1.5
\chapter{Introduction}

\textcolor{black}{Integration—the capacity of a system’s components to produce unified, 
emergent behavior through their interactions—is proposed as a key property in cognitive 
science, particularly in the study of consciousness, according to Integrated 
Information Theory (IIT) developed by Tononi \cite{tononi2004information,tononi2016}. 
IIT quantifies integration with the metric $\phi$, claiming to offer a framework 
to investigate conscious experience in neural systems. This thesis introduces $\phi_K$, 
a computationally efficient metric rooted in algorithmic probability, which enhances 
IIT’s ability to calculate $\phi$ while addressing its computational 
limitations \cite{mainbook}. Beyond cognitive science, integration underlies emergent 
phenomena in complexity science, spanning biological networks, artificial intelligence, 
and social systems \cite{barabasi}. By leveraging algorithmic probability to quantify 
integration’s compressibility, $\phi_K$ provides a versatile tool to explore integration 
in general, and therefore consciousness within the context of IIT, as well as the 
holistic nature of complex systems \cite{zenilreview}.}

\textcolor{black}{The development of techniques to decipher the structure and dynamics of complex 
systems is a rich inter-disciplinary research area which is not only of fundamental interest 
but also important in numerous applications. Broadly speaking, dynamical aspects such as 
stability and state-transitions within such systems have been of major interest in statistical 
physics, dynamical systems, and computational neuroscience \cite{strogatz,dayan,chandler}. 
Here, complex systems are defined by a set of non-linear evolution equations. 
Cellular automata, spin-glass systems, Hopfield networks, and Boolean networks, 
have for example been used as numerical experimental model systems to investigate 
the dynamical aspects of complex systems. Due to the complexity of the analysis, 
notions such as symmetries in the systems, averaging (e.g. mean-field techniques), 
and separation of time-scales, have all been instrumental in deciphering the core 
principles at work in such complex systems. In parallel, network science has emerged 
as a rich interdisciplinary field, essentially analyzing the structure of real networks 
in different areas of science and in diverse application domains \cite{barabasi}. 
Examples include social, biological and electrical networks, the web, business 
networks and the interconnected internet. 
By a structural analysis, which has dominated these investigations,
 we refer to statistical descriptions of network connectivity. 
 Networks can be described globally, in terms ranging from the 
 degree to which they differ from a random Poisson distribution of links, 
 to their modular organization, including their local properties such as 
 local clustering around nodes, special nodes/links with high degrees of 
 betweenness or serving specific roles in the network, and local motif structures. 
 Such network features can be used to classify and describe similarities 
 and differences between what appear to be different classes of networks 
 across and within different application domains. Finally, due to the 
 rich representational capacity of networks and their usefulness across 
 science, technology, and applications, work in machine learning, in 
 particular graph convolutional networks and embedding techniques, is 
 currently making headway in devising ways to map these non-regular 
 network objects onto a format such that machine learning techniques can 
 be used to analyze their properties \cite{Bronstein}.}

\textcolor{black}{Now, we may ask if integrated information theory (IIT) is 
proposed to be of relevance for the analysis of complex networks, we ask how is 
IIT related to fundamental questions underpinning research and thinking of 
complex systems? On the one hand, we find a rich body of work dealing with 
what could be referred to as technical, computational challenges, and application-driven 
investigations. For example, which global and local properties should be computed and 
how to do so in an efficient manner. However, at a more fundamental level we find 
essentially two challenges, which in our view have a bearing on
the core intellectual driving force of complex systems. First: What is the 
origin of and mechanisms propelling order in complex systems? Secondly, and of
 major concern for the present paper: Is the whole - in some sense - larger than 
 the sum of its parts? Both questions are vague when formulated in words, as above, 
 but they can readily be technically specified within a model class. The motivation 
 for the second question is that it appears that there are indeed phenomena in nature 
 which cannot easily be explained only with reference to their parts, but seem to require 
 that we adopt a holistic view. Since Anderson's classic 1972 essay, there has been an 
 animated and at times heated discussion of whether there is anything which could be 
 referred to as emergence.\cite{Anderson}}

\textcolor{black}{This interplay between emergent behavior, system-level causality, 
and holistic dynamics naturally intersects with deeper philosophical questions. 
Integrated Information Theory (IIT), while formal and quantitative, is also a 
metaphysical proposal—it claims that consciousness is identical to integrated 
information. This identity thesis bridges empirical neuroscience and foundational 
inquiries about the nature of subjective experience and self-coherence. By 
introducing a new metric, $\phi_K$, grounded in algorithmic probability, this 
thesis contributes not only to the improvement of computational methods, but 
also to the discourse on causality, representation, and emergence in cognitive 
systems. The ability to trace causal-generative explanations in network dynamics 
touches upon fundamental issues in the philosophy of mind: can consciousness be 
explained by the relationships between a system’s parts? And what does it mean 
for a system to simulate or reflect its own structure?}



\textcolor{black}{Tononi and his group have developed a formalism--targeting exactly 
such a holistic analysis--specifically to quantify the amount of information 
generated by a system -- defined as a set of interconnected elements--beyond the 
information generated by the parts (subsets) of the system. Their motivation was 
that in order to develop a theory of consciousness~\cite{oizumi2014phenomenology}. 
In that quest, they perceived a necessity to define a measure which could quantify 
the amount and degree of consciousness, a measure they refer to as $\phi$, which in 
turn constitutes the core of Integrated Information Theory or IIT. Importantly, in 
the present work we distinguish between the issue of the relevance of $\phi$ for 
consciousness versus the technical numerical question of how to calculate $\phi$. 
Here we address the computation of $\phi$, as it is potentially a means toward a 
precise formulation for the possible causal relation between a whole and the parts 
of a system, regardless of its purported relevance to consciousness. To calculate $\phi$, 
Tononi and collaborators have developed a computational toolbox~\cite{mayner2017pyphia}. 
Yet, calculating $\phi$ comes with a severe computational cost, as the calculation 
scales exponentially with the number of elements in the network. Furthermore, the 
computation requires knowledge of the transition probabilities of the system, which 
makes computation of anything larger than small systems of order of one magnitude 
intractable in practice. The calculation of $\phi$ requires a division of the system 
into smaller subsets, ranging from large pieces down to singletons, 
every division into $k$ pieces can be instantiated in $\dbinom{N}{k}$ different ways.
Using this procedure from Tononi, elements that have small causal influences on 
the activity of other elements can be identified. A system with low $\phi$ is 
therefore characterized by the fact that changes in subsets of the system do 
not affect the rest of the system. Such a system is therefore considered to be 
a non-integrated system. This observation entails a key insight, namely, that 
if a system is highly integrated among its parts, then the different parts 
can be related to each other, or more precisely, they can be used to describe 
other parts of the system. Then the parts are in some sense simple and should 
be compressible.}

\textcolor{black}{This is the observation and intuition behind our method, 
which employs a formalized notion of complexity to exploit this insight and 
thereby allow a more efficient, guided search in the space of algorithmic distances, 
in contrast to exhaustive computations of the distance between statistical 
distributions, as currently implemented in IIT. Technically we are therefore 
not required to perform a full computation of what is referred to as the input-output 
repertoire (see Methods for technical details). This, in brief, is our motivation 
for introducing our method, which is based on algorithmic information 
dynamics~\cite{maininfo,zenilaid,mainbook}. At its core is a causal perturbation 
analysis and a measure of sophistication connected to algorithmic complexity. 
Our approach exploits the idea that causal deterministic systems have a simple 
algorithmic description and thus a simple generating mechanism sufficient to 
simulate and reproduce complex systemic behaviour. Using this technique we can 
assess the effect of perturbations, and thereby exploit the fact that, depending 
on the algorithmic complexity of a system, the perturbation will induce different 
degrees of change in algorithmic space. In short, a system will be highly integrated 
only if the removal or perturbation of its parts has a non-linear effect on the 
generative program producing the system in the first place.}

\textcolor{black}{Interestingly, even Tononi suggested early on that algorithmic 
complexity could be connected to the computation of integrated information 
~\cite{sciencetranslationmedicine}. However, a lossless compression 
algorithm was used to approximate $\phi$. Here we contribute to the formalization 
of such a suggestion by using stronger tools, which we have recently developed, 
to approximate complexity. 
At the core of algorithmic information is the concept of minimal program-size 
and Kolmogorov-Chaitin complexity~\cite{kolmogorov,chaitin}. Briefly, the 
Kolmogorov-Chaitin complexity $K(x)$ of an object $x$ is the length of the 
shortest computer program that produces $x$ and halts. $K(x)$ is uncomputable 
but can be approximated from above, meaning one can find upper bounds by using 
compression algorithms, or rather more powerful techniques such as those 
based on algorithmic probability~\cite{d4,d5,bdm}, given that popular lossless 
compression algorithms are limited and more closely related to classical 
Shannon entropy than to $K$ itself~\cite{zkpaper,smalldata,liliana}. One 
reason for this state of affairs is that, as demonstrated in \cite{shalizi2001computational}, 
there is a fundamental difference between algorithmic and statistical 
complexity with respect to how randomness is characterised in opposition 
to causation. Specifically, algorithmic complexity implies a deterministic 
description of an object (it defines the algorithmic information content of an 
individual sequence/object), whereas statistical complexity implies a 
statistical description (it refers to an ensemble of sequences generated 
by a certain source). Approaches such as transfer entropy~\cite{transfer}, 
Granger causality~\cite{granger}, and Partial Information 
Decomposition~\cite{williams,williams2} that are based on regression, 
correlation and/or a combination of regression, correlation and 
intervention but ultimately relying on probability distributions, 
fall into this category. Hence for better-founded methods and algorithms 
for estimating algorithmic complexity, we recommend the use of our tools, 
which are already being used by independent groups working on, for example, 
biological modelling~\cite{victor}, cognition~\cite{ventresca} and 
consciousness~\cite{ruffini}. These tools are based on the theory of 
algorithmic probability, and are not free from challenges and limitations,
 but they are better connected to the algorithmic side of algorithmic 
 complexity, rather than only to the statistical pattern-matching 
 side that current approaches using popular lossless compression 
 algorithms exploit, making these approaches potentially misleading~\cite{zkpaper}.}

\textcolor{black}{Our procedure, in brief, is as follows. First, we deduce 
the rules in systems of interest: we apply the perturbation test introduced 
in \cite{zenilturingtest,maininfo,mainbook} to ascertain the computational 
capabilities of networks. Next, simple rules are formalized and implemented 
to simulate the behaviour of these systems. Following this analysis, 
we perform an automatic procedure, referred to as a meta-perturbation 
test, which is applied over the behaviour obtained by the aforementioned 
simple rules, in order to arrive at explanations of such behaviour.}
 We incorporate the ideas of an interventionist calculus 
 \textcolor{black}{(c.f. Judea Pearl~\cite{pearl})} and perturbation 
 analysis within what we call Algorithmic Information Dynamics, and 
 \textcolor{black}{we go beyond pattern identification using probability 
 theory, classical statistics, and correlation analysis by developing a 
 model-driven approach that is fed by data. This contrasts with a purely 
 data-driven approach, and is a consequence of the fact that our analysis 
 considers the algorithmic distance between models.}
            % ~10 páginas - Explicar el propósito de la tesis
\chapter{Theoretical Framework}

\textcolor{black}{Integrated information theory (IIT) postulates that
consciousness is identical to integrated information and that a system's
capacity for consciousness can be expressed by a quantitative measure
denoted by $\phi$}. \textcolor{black}{Tononi defines integrated information as ``the amount
of information generated by a complex of elements, above and beyond the information generated 
by its parts'' \cite{tononi2008consciousness}
and states, }\textcolor{black}{\emph{``The integrated information theory}}\textcolor{black}{{} (IIT) 
of consciousness
claims that, at a fundamental level, consciousness is integrated information'' 
\cite{tononi2008consciousness} (italics in original).} \textcolor{black}{IIT aims to explain 
``relationships between consciousness and the Physical Substrate of Consciousness (PSC), 
and starts from essential properties of phenomenal experience, and derives the
requirements for the physical substrate of consciousness.'' \cite{tononi2016}}

\textcolor{black}{To rigorously formulate this proposal, IIT introduces a mathematical framework 
for quantifying the integration of information within a system. The foundation of this formalism 
is built upon the calculus of $\phi$, which we describe in the next section.}

\subsection*{\textcolor{black}{Calculus of $\phi$}}

\textcolor{black}{The integrated information theory defines integrated information ($\phi$) 
as the effective information of the minimum information partition (MIP) in a system 
\cite{oizumi2014fromthe, oizumi2014phenomenology, tononi2004information, tononi2016}. 
The MIP is also defined as the partition having minimum effective information 
among all possible partitions.}

\begin{center}
\textcolor{black}{$\phi[X;x]=:\varphi[X;x,MIP(x)]$}
\par\end{center}

\begin{center}
\textcolor{black}{ $MIP(x)=:argmin{\varphi(X;x,P)}$}
\par\end{center}

\textcolor{black}{Where }\textcolor{black}{\emph{$X$}}\textcolor{black}{{} is the system, 
$x$ is a state, and $P$ is a partition $P=M_{1}, \ldots ,M_{r}$.}

\textcolor{black}{Importantly, identifying the MIP requires searching all possible
partitions and comparing their effective information $\phi$. 
This effective information is specified in terms of effect and causal information, 
that is, the distance between two probability distributions: one for the unpartitioned 
(unconstrained) partition (this can be the full set of nodes of the whole system 
or one of its possible partitions) and a partition of this latter. Such probability 
distributions determine probabilities  of all possible future (effect) or past 
(causal) states of an arbitrary partition being in a current state. 
This means that comparing one set of nodes that can be the full set of nodes 
of the system or a subset (partition) of itself with all possible partitions 
of this set of nodes, MIP represents the partitions with the minimal value of 
the distance between probability distributions of the set of nodes and one of 
all its possible partitions.}

\textcolor{black}{When a set of nodes is chosen to compute effective information, 
this is referred to as a `mechanism', and the partition to which it is compared 
is referred to as the `purview'. The distance between probability distributions 
is computed by means of an adaptation of the Earth Mover's Distance (EMD) algorithm, 
which is a method to evaluate dissimilarity between two multi-dimensional 
distributions in a given feature space where a distance measure between single 
features, which we call the ground distance, is given. The EMD ``lifts'' this 
distance from individual features to full distributions. Note that EMD is referred 
to as a Wasserstein metric in mathematics, and is commonly used in machine learning 
as a natural metric between two distributions~\cite{villani}.}

\textcolor{black}{Intuitively, given two distributions, one can be seen as a
 mass of earth properly spread in space, the other as a collection
of holes in that same space. Then, the EMD measures the least amount of 
work needed to fill the holes with earth. Here, a unit of work corresponds 
to transporting (by an optimal transport method) a unit of earth a unit of 
ground distance.}

\textcolor{black}{While the calculus of $\phi$ provides the quantitative backbone of IIT, 
its explanatory strength lies in its grounding in phenomenological principles. 
We now turn to the conceptual framework that connects this formalism to the nature 
of consciousness.}

\subsection*{\textcolor{black}{IIT’s Consciousness Framework}}

\textcolor{black}{Integrated Information Theory (IIT) provides a theoretical framework for consciousness by defining it as the capacity of a system to integrate information \cite{tononi2004information,tononi2016}. IIT is grounded in five phenomenological axioms: existence (consciousness exists), composition (it is structured), information (it is specific), integration (it is unified), and exclusion (it is definite) \cite{oizumi2014phenomenology}. These axioms imply that a conscious system must generate information through integrated interactions among its components, quantified by $\phi$, the degree of irreducible information \cite{tononi2008consciousness}. The Physical Substrate of Consciousness (PSC), typically neural networks, must support these properties, with $\phi$ measuring the system’s capacity to produce a unified conscious experience. IIT’s framework has been applied to model neural correlates of consciousness, distinguishing conscious from unconscious states \cite{sciencetranslationmedicine}. However, computing $\phi$ is computationally intensive, limiting its practical use. The $\phi_K$ metric, introduced in this thesis, leverages algorithmic probability to approximate integration more efficiently, enhancing IIT’s applicability to cognitive science research \cite{mainbook}.}

\subsection*{\textcolor{black}{Cognitive Science Applications}}

\textcolor{black}{IIT’s framework, augmented by $\phi_K$, offers significant applications in cognitive science, particularly in understanding consciousness. For example, IIT has been used to assess disorders of consciousness (e.g., coma, vegetative states) by measuring $\phi$ in neural systems via perturbation analysis \cite{sciencetranslationmedicine}. $\phi_K$ improves this by reducing computational demands, enabling real-time analysis of brain networks during tasks like attention or perception \cite{mainbook}. In neural modeling, $\phi_K$ facilitates simulations of integrated information in large-scale networks, aiding research into cognitive processes such as working memory \cite{dayan}. Additionally, $\phi_K$’s algorithmic probability approach allows for studying consciousness in artificial systems, exploring whether AI can exhibit integrated information akin to human cognition \cite{ruffini}. These applications strengthen IIT’s empirical utility, making $\phi_K$ a vital tool for cognitive science investigations into the neural and computational basis of consciousness \cite{tononi2016}.}

\textcolor{black}{The effectiveness of $\phi_K$ in these contexts rests on a deeper connection between information integration and the algorithmic nature of complex systems.}

\subsection*{\textcolor{black}{Algorithmic Probability in Cognitive Science}}

\textcolor{black}{Algorithmic Information Theory (AIT), originating with Turing’s concept of a universal computing machine, provides a foundation for understanding intelligence and complexity in cognitive science \cite{turing1950}. Turing’s machine, capable of simulating any computation, underpins AIT’s measure of complexity via Kolmogorov’s minimal program length \cite{kolmogorov}. Zenil’s Coding Theorem Method (CTM) and Block Decomposition Method (BDM) advance this by analyzing integration in complex systems, revealing how interactions yield emergent phenomena \cite{zenilaid,bdm}. This thesis, inspired by CTM and BDM’s philosophical core—where small rules govern complex behaviors—proposes $\phi_K$ to quantify integration without directly applying these methods \cite{mainbook}. Grounded in the causal principle that simple programs produce neural dynamics, $\phi_K$ leverages algorithmic probability to model consciousness under the context of IIT, revealing fractal patterns in information spread akin to schemata in information theory \cite{shannon1948}. These characteristics align with IIT’s view of the Physical Substrate of Consciousness, enhancing cognitive science’s exploration of emergent conscious phenomena \cite{oizumi2014phenomenology}.}
           % ~20 páginas - Poner un contexto a la tesis, hacer referencia a trabajos actuales en el tema

%%%%%%%%%%%%%%%%%%%%%%%%%%%%%%%%%%%%%%%%%%%%%%%%%%%%%%%%%%%%%%%%%%%%%%%%%
%           Capítulo 3: NOMBRE                   %
%%%%%%%%%%%%%%%%%%%%%%%%%%%%%%%%%%%%%%%%%%%%%%%%%%%%%%%%%%%%%%%%%%%%%%%%%

\chapter{Methods}
In this section we  introduction of the meta-perturbation analysis, 
additional technical details of which are presented in the Appendix. 
Next, we recap the causal perturbation and causal analysis leading up 
to the notion of program-size divergence, which is our core metric for 
how different programs, i.e. systems--more or less integrated--respond 
to perturbations
\subsection{\textcolor{black}{Programmability test and meta-test}}

\textcolor{black}{In \cite{zenilturingtest} a programmability test is 
introduced which was inspired by the Turing test, while being based on 
the view that the universe and all physical systems living in it and able 
to process information can be considered (natural) computers \cite{zenilturingtest} 
equipped with particular computational capabilities \cite{zenil2013behavioural}.}

\textcolor{black}{The programmability test is explained as: ``...replacing 
the question of whether a system is capable of digital computation with 
the question of whether a system can behave like a digital computer and 
whether a digital computer can exhibit the behaviour of a natural system.''}

\textcolor{black}{Then, in the same way that the Turing test proceeds
to ask questions of a computer in order to determine whether it is
capable of computing an intelligent behaviour, the programmability 
test aims to know what a specific system is capable of computing 
by means of algorithmic querying \cite{zenil2015causality}.}

\textcolor{black}{In practice, the programmability test is
a system perturbation test~\cite{zenilturingtest,zenilturingtest2} 
that "asks" questions of a computational system in the form: }
\textcolor{black}{\emph{what
is your output (answer) given this question (input)?}}\textcolor{black}{.
This idea is applied to $\phi_{K}$'s implementation so that once the 
set of all possible answers of a system is obtained, this set is analyzed 
and generalized to deduce the rules that should not just offer a picture 
of its computability capabilities, but also simulate and give an account 
of the behaviour of the system itself.}

\textcolor{black}{A second step after this perturbation test is to analyze
its results in order to construct a computer program--as simple as 
possible--capable not only of reproducing the output repertoire 
but also of giving an account of the programmability capabilities 
of the system itself, that is, rules capable of producing a certain 
output given an input, and at the same time explaining where, in ordinal terms, 
such an output could be placed relative to the order of the full output repertoire. 
This latter aspect we refer to as the meta-perturbation test.}

\textcolor{black}{Then, $\phi_{K}$ not only applies a perturbation
test over a system, but also a meta-perturbation test over results
obtained on the first test. The rules found in this meta-test are
used not only as compressed specifications or representations of
the behaviour itself, but also as rules that give a sort of account
of the behaviour of the system. }

\textcolor{black}{This can be done because the systems analyzed in IIT
are well known, or in other words, since all node-by-node operations
are well defined, it is easy to compute all possible outputs (answers)
for all possible inputs (questions or queries), corresponding to what in 
IIT are referred to as repertoires. In the context of $\phi_{K}$, 
a meta-test is applied in order to find the rules that describe the 
behaviour embedded in repertoires of a system, instead of trying to 
ascertain the rules that define how the system works.}

\textcolor{black}{A system specified in this manner turns on a ``computer\textquotedbl{}, recording it's own behaviour (e.g the repertoires) as well as probing itself, e.g. the action of $\phi_{K}$, in such a manner as to potentially give an account of its own behaviour. To make
this possible a system specification must be enabled with an explanatory
interface based on these simple embedding behaviour rules. $\phi_{K}$
goes beyond the original $\phi$ in that the programmability test
searches for the rules underlying the behaviour of a system rather than
generating a description of its possible causal connections. While in
IIT these rules are defined a priori and induced by perturbation,
$\phi_{K}$’s objective is not only to find rules that simulate, but
also describe such behaviour in a brief manner (thus simple rules)
and make predictions about the behaviour of the system. The field
of Algorithmic Information Dynamics~\cite{maininfo,mainbook} implements
this approach by asking what changes to hypothesized outputs mean
for the hypothesized underlying program generating an observation,
after an induced or natural perturbation.}

\textcolor{black}{The simple rules discovered and used for the calculation
of $\phi_{K}$ are used here exclusively to compose }\textcolor{black}{\emph{constrained/unconstrained
distributions}}\textcolor{black}{{} used in IIT for obtaining }\textcolor{black}{\emph{cause-and-effect
information}}\textcolor{black}{, a key concept from which the integration
of information derives. The rest of the calculus--earth mover's distance
measurements, the calculus of conceptual spaces, major complex and
finding the MIP-- remains as specified in IIT 3.0.}

\subsection{Causal perturbation analysis}

From a statistical standpoint, it would be typical to suggest that the 
behaviour of two time series, let's call them $X$ and $Z$, would potentially 
be causally connected if they were statistically correlated. Yet, there are 
several other cases that would not be distinguishable after a correlation test. 
A first possibility is that the time series simply shows similar behaviour 
without being causally connected, i.e. there is a shared upstream causal driver $Y$, 
concealed from the observer. Another possibility is that they are causally 
connected, but that correlation does not tell us whether it is a case 
of $X$ affecting $Z$, or vice versa. 

Perturbation analysis allows some disambiguation. The idea is to apply a 
perturbation on one time series and see how the perturbation spreads to 
the other time series. Perturbing the data-point in position 5 the time 
series $Z$ as shown in Figure~\ref{timeseriesbeforeafter} multiplying it by -2, $X$ 
does not respond to the perturbation. This means that for this data point, $X$ 
remains the same. This suggests that there is no causal influence of $Z$ on $X$.

\begin{figure*}[ht!]
  \centering
  \includegraphics[scale=0.3]{Capitulo3/figs/timeseriesbeforeafter.png}\\
  \includegraphics[scale=0.3]{Capitulo3/figs/timeseriesbeforeafter2.png}
  \caption{\label{timeseriesbeforeafter}\textcolor{black}
  {Causal intervention analysis on time series $X$ and $Z$ before and after 
  perturbation in $Z$ (top) and $X$ (bottom). The values of $Z$ come from 
  the moving average of $X$, so there is a one-way causal relationship: 
  perturbing $X$ has an effect on $Z$ but perturbing $Z$ has no effect on 
  $X$ thereby suggesting the causal relationship.}}
\end{figure*}

In contrast, if the perturbation is applied to a value of $X$, $Z$ 
changes and follows the direction of the new value, suggesting that 
the perturbation of $X$ has a causal influence on $Z$. From behind 
the scenes, we can reveal that $Z$ is the moving average of $X$, 
which means that each value of $Z$ takes two values of $X$ to 
calculate, and so is a function of $X$. The results of these 
perturbations produce evidence in favour of a causal relationship 
between these processes, if we did not know that they were related 
by the function we just described.

This suggests that it is $X$ which causally precedes $Z$. So we 
can say that this single perturbation suggests a causal 
relationship illustrated in Figure~\ref{causal1}.

\begin{figure*}[htp]
  \centering \includegraphics[scale=0.3]{Capitulo3/figs/causalrelationship} 
  \caption{\label{causal1}Possible self-loopless causal relationship 
  between two \textcolor{black}{unlabelled} variables
  $X$ and $Z$.}
\end{figure*}
  
\begin{figure*}[htp]
\centering \includegraphics[scale=0.25]{Capitulo3/figs/causalrelationshipdag} 
\caption{\label{timeseriesafterbefore2}Acyclic path graphs 
representing all possible \textcolor{black}{self-loopless}
connected causal relationships among 3 \textcolor{black}{unlabelled} 
variables.}
\end{figure*}

There are a number of possible types of causal relationship between three events 
(see Figure~\ref{timeseriesafterbefore2}) that can be represented in what is 
known as a directed acyclic graph (DAG), that is, a graph that has arrows 
implying a cause and effect relationship but has no loops, because a loop 
would make a cause into the cause of itself, or an effect that is also 
its own cause, something that would be incommensurate with causality. 
In these graphs, nodes are events and events are linked to each other if 
there is a direct cause-and-effect relation.

In the first case, labelled $A$ in orange, the event $X$ is the cause of event
 $Y$, and $Y$ is the cause of event $Z$, but $X$ is said to be an indirect 
 cause of $Z$. In general, we are, of course, always more interested in direct 
 causes, because almost anything can be an indirect cause of anything else. 
 In the second case $B$, an event $Y$ is a direct cause of both $Z$ and $X$. 
 Finally, in case $C$, the event $Y$ has 2 causes, $X$ and $Z$. With an 
 interventionist calculus such as the one performed on the time series above, 
 one may rule out some but not all cases, but more importantly, the perturbation 
 analysis offers the means to start constructing a model explaining the system and 
 data rather than merely describing it in terms of simpler correlations.

In our approach to integrated information, \textcolor{black}{the idea is to 
identify the set of most likely generating candidates able to produce certain 
observed behaviour even if such behaviour may not carry any statistical regularity 
and for all purposes appear statistically random~\cite{devine2006ait}. Strictly 
speaking, computational mechanics~\cite{Shalizi2003Optimal,shalizi2001computational}, 
is a framework that bridges statistical inference and stochastic modelling that 
suggests a model based on an automaton called an $\epsilon$-machine. However, 
such machines are stochastic in nature and, if the methods used to reconstruct 
such machines rely on statistical methods, the result is only an apparent causal 
representation with no correspondence between internal states and alleged states 
of the phenomenon observed. In contrast, approaches based on algorithmic probability 
as approached by algorithmic information dynamics can complement computational 
mechanics as they provide means to construct non-stochastic automata models that 
are independent of probability distributions and are in a strict sense optimal 
and universal~\cite{solomonoff1964formal}.}

In the case of our two time series experiments, the time series $X$
is produced by the mathematical function $f(x)=Sin(x)$, and thus $Sin(x)$ is the generating mechanism of time series $X$. On the other hand, the generating mechanism of $Z$ is $MovAvg(f(x))$, and clearly $MovAvg(f(x))$ depends on $f(x)$, which is $Sin(x)$, but $Sin(X)$ does not depend on $MovAvg(f(x))$. In the context of networks, the algorithmic-information
dynamics of a network is the trajectory of a network moving in algorithmic-information
space together with the identification of those elements that shoot
the network towards or away from randomness.

\subsection{Causal influence and sublog program-size divergence}

According to Algorithmic Information Dynamics~\cite{maininfo,mainbook} there 
is an algorithmic causal relationship between two states $s_{t}$ and $s_{t'}$ 
of a system $M$ and $M'$ if

\[
|K(M_{s_{t}}) - K(M'_{s_{t'}})|\leq \log_{2}(t)+c
\]

That is, if the descriptions of such systems can be bounded by $\log_{2}$ 
and a small constant $c$, then $M$ is most likely equal to $M'$ but in some 
other time state.
In other words, if there is a causal influence of $s_{t}$ on $s_{t+1}$ or $s_{t+1}$ 
on $s_{t}$ as a system in isolation, their $M$ and $M'$ short descriptions should 
not differ by more than the description of the difference.
\textcolor{black}{However, if the descriptions of the states of a system 
(which may be two systems) in different alleged state times are not causally 
connected, their difference will diverge beyond above bound. In integrated 
information, causal influence among its parts is what is claimed to be measured 
and how different elements of a system can be explained by a single model or the 
other parts of the system informs us as to how integrated a system may be. 
A system characterized by large divergence is less integrated compared to a 
system which evolves with small differences in its respective subpart descriptions.}

We will suggest that perturbations have to be algorithmic in nature because they 
need to be made or quantified at the level of the generating mechanisms from the 
whole or different parts of the integrated system and not at the level of the 
observations. For example, some $n-$ary expansions of the mathematical constant 
$\pi$ according to BPP (named after Bailey-Borwein-Plouffe) formulas~\cite{bpp} 
allow perturbations to the digits that do not have any further effect because 
no previous digits are needed to calculate any other segment of $\pi$ in the same base. 
The constant $\pi$ then can be said to be information disintegrated to the extent 
of the BPP representations. Algorithmically low complexity objects have low 
integrated information. Similarly, highly random systems have low integrated 
information, because perturbations have little to no impact. Integrated information is, 
therefore, a measure of sophistication that separates high integration from both 
random and trivially non-random states.


\subsection{A simplicity versus complexity test}

With the previous section in mind we can proceed to introduce the idea of $\phi_{K}$,
where $K$ stands for the letter often used for algorithmic (from Kolmogorov or
Kolmogorov-Chaitin) complexity, and $\phi$ for the traditional of integrated 
information theory~\cite{oizumi2014phenomenology}. The measure $\phi_{K}$ mostly follows methods that Oizumi and Tononi set forth in \cite{oizumi2014phenomenology}, where integrated information is
measured, roughly speaking, as distances between probability distributions 
that characterize a MIP (Minimum Information Partition), that is, 
``the partition of [a system] that makes the least difference''
~\cite{oizumi2014phenomenology}.


\textcolor{black}{However, the difference between IIT's $\phi$ and
$\phi_{K}$ lies in how $\phi_{K}$ circumvents what is called the
``intrinsic information bottleneck principle’’~\cite{oizumi2014fromthe}
that traditionally requires an exhaustive search for the MIP among
all possible partitions of a system, a procedure responsible for the
fact that integrated information computation requires super-exponential
computational resources. In contrast to $\phi$, which follows a statistical
approach to estimating and exhaustively reviewing repertoires, the approach
to $\phi_{K}$ is based on principles of algorithmic information.}

\textcolor{black}{Discovering the simple rules that govern a ``discrete
dynamical system'' \cite{mayner2017pyphia} like those studied in
IIT presupposes the analysis
of its general behaviour in pursuit of a dual agenda: first, to determine 
its computational capabilities, and secondly to obtain explanations and 
descriptions of the behaviour of the system.}

\textcolor{black}{As a consequence, one of the major adaptations of IIT is 
that $\phi_{K}$ uses the concept of Unconstrained Bit Probability Distribution 
(UBPD), that is, the individual probabilities associated with a node of a system 
taking values of 1 (ON) or 0 (OFF) after it has been ``fed'' all its possible 
inputs or after all possible perturbations.}

\textcolor{black}{In the context of $\phi_{K}$, UBDP is estimated by approximating 
the algorithmic complexity of the TPM (Transition Probability Matrix) to compute 
IIT's unconstrained/constrained probability distributions.} 

\textcolor{black}{In Figure \ref{fig:3nodeExample} and Table \ref{table:UBPD} the concept UBPD and its 
calculus is explained, using the example used by Oizumi et. al. 
in~\cite{oizumi2014phenomenology}.}



\textcolor{black}{In order to explain the notion of UBPD we use FigureS \ref{fig:3nodeExample}, 
\ref{fig:UBPD} and Table \ref{table:UBPD}. in Figure \ref{fig:3nodeExample} 
we use Oizumi's example used in~\cite{oizumi2014phenomenology} to 
calculate information integration. Figure \ref{fig:3nodeExample}-A shows the network representation: 
three nodes fully connected with different types of operation executed
 on its inputs, that is, for example, inputs to node A (coming from B and C nodes) 
 will be processed in a logical OR operation. In Figure \ref{fig:3nodeExample}-B the adjacency 
 matrix that represents the same same network is shown. This adjacency matrix 
 uses the number 1 to indicate if a node receives signals (inputs) for another 
 node. For example, the first row in the adjacency matrix indicates that node 
 1 or A receives inputs from nodes B and C, denoted as nodes 2 and 3.
  Finally, Figure \ref{fig:3nodeExample}-C shows the full input and output repertoires, that is, 
  for the full set of all possible inputs to this system, all corresponding 
  outputs are calculated according to the logical operations
defined.}

\begin{figure}{}
  \begin{centering}
  \includegraphics[scale=0.44]{Capitulo3/figs/Fig1-NewReferenceSystem.png}
  \par\end{centering}
  \caption{Three node example of a full connected network. 
  From \cite{oizumi2014phenomenology}.}
  \label{fig:3nodeExample}
\end{figure}

\begin{figure}{}
  \begin{centering}
  \includegraphics[scale=0.44]{Capitulo3/figs/Fig2-UBPCforEffectProbDistro.png}
  \par\end{centering}
  \caption{An example of using UBPC to calculate an unconstrained output distribution.}
  \label{fig:UBPD}
\end{figure}

\textcolor{black}{Table \ref{table:UBPD} shows code for computing UBPD for 
the system in Figure \ref{fig:3nodeExample}.
This computation starts with the specification of the adjacency matrix (line 1) 
and internal dynamic (line 2) of the
target system. Then, lines 1 and 2 in Table \ref{table:UBPD} represent code to network 
specified in Figures \ref{fig:3nodeExample}-A and \ref{fig:3nodeExample}-B.}

\begin{table}[H]
  \begin{lstlisting}[style=MathematicaStyle]
  In := am = {{0, 1, 1}, {1, 0, 1}, {1, 1, 0}};
  In := dyn = {"OR", "AND", "XOR"};
  In := calcUBPOutputs[1, am, dyn] // AbsoluteTiming
  In := calcUBPOutputs[2, am, dyn] // AbsoluteTiming
  In := calcUBPOutputs[3, am, dyn] // AbsoluteTiming
  
  Out = {0.000575, <|"ZeroProb" -> 0.25, "OneProb" -> 0.75|>}
  Out = {0.000287, <|"ZeroProb" -> 0.75, "OneProb" -> 0.25|>}
  Out = {0.000341, <|"ZeroProb" -> 0.5, "OneProb" -> 0.5|>}
  \end{lstlisting}
  \caption{Computing UBPD for system shown in Figure \ref{fig:3nodeExample}.
  \textbf{Lines 1, 2:} Definition of 
  the system in Figure \ref{fig:3nodeExample}-A by adjacency matrix (line 1) 
  and dynamics (line 2). 
  \textbf{Lines 3-5: } Calculation of individual probabilities that each node of the 
  system will take values 0/1 across the whole output repertoire. 
  \textbf{Lines 7-9: } time of computation in seconds and UBPD distribution. }
  \label{table:UBPD}
  \end{table}

\textcolor{black}{In the IIT approach, the system is perturbed with all possible 
inputs to obtain the full output repertoire (Figure \ref{fig:3nodeExample}-C).
Then, in the context of $\phi_{K}$, UBPD corresponds to the distribution 
of probabilities that each node will take values 0/1 in the output/input
repertoires after the perturbation. For instance, in Figure \ref{fig:UBPD}-A,
full input and output repertoires are shown for network in Figure \ref{fig:3nodeExample}-A.
Now, let's say we want to compute the future probability distribution,
that is, the probability necessary to compute effect information according
to \cite{oizumi2014phenomenology}. In this case we take output repertoire as 
a reference and we compute the probability of nodes in the future (outputs) 
taking the values 0 or 1. For node A, for example, the probability that node 
A takes the value of 1 is 0.25, that is 2/8, and that it takes the value
of 0 is 0.75 or 6/8. These values are called the UBPD for node A. A resume 
of UBPD for all nodes is given in Figure \ref{fig:UBPD}-B.}

\textcolor{black}{Once UBPD is computed for a subject partition, in this case 
the full system's probability distribution is computed by multiplying UBDPs. 
Let's take as an example the future probability of input \{0,
0, 0\}, computed as $P(A) = 0$ {*} $P(B) = 0$ {*} $P(C) = 0$, that is, 
0.25 {*} 0.75 {*} 0.5 (see first row in Figure \ref{fig:UBPD}-C). 
When all future probabilities are computed in this manner, the result is the 
distribution shown in Figure \ref{fig:UBPD}-D, which is exactly the same one computed in 
\cite{oizumi2014phenomenology},
as shown in Figure \ref{fig:UBPD}-E.}

\textcolor{black}{In general, UBDP is used to compute probability distributions
of a system in the context of $\phi_{K}$, which mirrors the 
``constrained/unconstrained
probability distributions'' in \cite{oizumi2014phenomenology}, that is, 
probability distributions of input/output patterns for specific configurations 
(partitions) of the system, in contrast to what IIT 3.0 does. In this 
last case, Mayner shows how probability distributions are computed in the context 
of IIT in his S1 text mentioned in \cite{mayner2017pyphia},
using terms such as ``marginalization'' and ``virtual
elements'' that seem to be highly complex methods.}

\textcolor{black}{Then, in the context of $\phi_{K}$, UBDP aims to 
obtain the 
same results in terms of probability distributions, in a manner equivalent 
to IIT but by following a different conceptual
approach. Our measure $\phi_{K}$ uses adapted methods, having
algorithmic complexity as a background, to compute information integration.}

\textcolor{black}{In Table \ref{table:UBPD}, lines 3-5 shows Mathematica code that
computes UBDP for the system specified in lines 1 and 2, that is, by means of 
an adjacency matrix and an array of computations that nodes perform, 
or the system dynamics. Table \ref{table:UBPD} also shows results of this computation 
in this order: 1) time needed to compute, followed by probability that 
a node take the value zero (zeroProb) or the value one(oneProb).} 
One can see how the results in Table \ref{table:UBPD} correspond to UBDP values 
shown in Figure \ref{fig:UBPD}-B.

\textcolor{black}{We should note that for $\phi_{K}$, computation
of probability distributions seems to be a task of counting, which for huge 
systems would be extremely difficult or even impossible, if attempted in 
a classical/brute force way. But, two important facts should be pointed out here: 
1) In the context of $\phi_{K}$, UBPD is not calculated in this traditional 
way, but is calculated using the simulation of the behaviour of a system 
represented by a set of simple rules. Then for $\phi_{K}$, an 
exhaustive review of repertoires is not needed to compute the 
individual probabilities shown in Table 1, and 2) despite strong 
theoretical and methodological differences between them, $\phi_{K}$ and $\phi$ lead 
to the same
results.}

In the next sections we derive simple rules of a system, using the 
perturbation test and its application to implement $\phi_{K}$.

      % ~20 páginas - Explicar el problema en específico que se va a resolver, la metodología y experimentos/métodos utilizados
\chapter{Results}
\subsection{Compression sensitivity as informative of integration}

To understand the relationship between IIT and algorithmic complexity, 
we shall briefly move away from the case of networks and focus on binary 
files and the binary programs that may generate them, the programs that 
are natural computable candidate models explaining the data. To illustrate 
the connection, let us take some extreme cases. Let's say we have a random file:


\begin{table}[H]
\begin{lstlisting}[style=MathematicaStyle]
In: = randomfile=RandomChoice[CharacterRange["a", "z"],100]
Out = {l,q,l,d,d,x,f,e,u,l,u,b,m,y,a,l,b,l,v,a,v,v,t,l,h,o,x,d,i,a,z,t,z,b,r,y,v,h,c,n,s,u,l,u,w,w,e,a,i,h,h,w,f,d,d,k,o,c,a,k,u,x,v,n,v,e,c,r,t,c,g,s,g,x,y,t,c,h,k,w,c,t,y,u,e,k,y,v,a,h,t,k,f,y,c,r,b,y,y,x}

In: = Compress[randomfile]
Out = 1:eJx1kEEOgyAQRWnSi/Q+XbUnEAUhtJoKInp6zbB5CenmZTL8mf+Hh55fdlBKxfuFp4/pfbuKj/CHegCL0AqNcIWy1lr4Fe7CDhqNOuM1g0zihDPcaxKP2QNTB1wWZMjY1gsnYfxzxQYaeHnscdDY5q8CkvfYEOBSkG1CbTC14LraGZF8xJ69UTo4bugn6Fc4hubHOuxJ0FgomVOjX1lOUdJPvg==}

In: = Length[Characters@Compress[randomfile]]
Out = 222
\end{lstlisting}
\caption{Code in Mathematica for generation of a file with random alphabetic content and 
measurement of its compressed version. \textbf{Line 1} generates a file with random selection
of alphabetic characters. \textbf{Line 2} compresses the file generated.}
\label{}
\end{table}


\textcolor{black}{So, using the \textbf{Compress} algorithm, the resulting 
 compressed object is even longer, this is because the compression algorithms inserts the decompression instructions together with the checksum which ends up increasing the size of the resulting object if the object was not long and compressible enough to begin with.}

This is what happens if we perform a couple of random perturbations to
the uncompressed file like replace the 5th and 12th characters with the letters `k' and `x:

\begin{table}[H]
\begin{lstlisting}[style=MathematicaStyle]
In: = mutatedfile=ReplacePart[randomfile,{5->"k", 12->"x"}]
Out = {l,q,l,d,k,x,f,e,u,l,u,x,m,y,a,l,b,l,v,a,v,v,t,l,h,o,x,d,i,a,z,t,z,b,r,y,v,h,c,n,s,u,l,u,w,w,e,a,i,h,h,w,f,d,d,k,o,c,a,k,u,x,v,n,v,e,c,r,t,c,g,s,g,x,y,t,c,h,k,w,c,t,y,u,e,k,y,v,a,h,t,k,f,y,c,r,b,y,y,x}
\end{lstlisting}
\caption{Code in Mathematica for replacing 5th and 12th characters by 'k' and 'x' respectively.}
\label{}
\end{table}

The difference between the original and perturbed files is:

\begin{table}[H]
\begin{lstlisting}[style=MathematicaStyle]
In := SequenceAlignment[randomfile,mutatedfile]//Column
Out= {l,q,l,d} {{d},{k}} {x,f,e,u,l,u} {{b},{x}} {m,y,a,l,b,l,v,a,v,v,t,l,h,o,x,d,i,a,z,t,z,b,r,y,v,h,c,n,s,u,l,u,w,w,e,a,i,h,h,w,f,d,d,k,o,c,a,k,u,x,v,n,v,e,c,r,t,c,g,s,g,x,y,t,c,h,k,w,c,t,y,u,e,k,y,v,a,h,t,k,f,y,c,r,b,y,y,x}
\end{lstlisting}
\caption{\textbf{SequenceAlignment} function to compare two sequences, randomfile 
and mutatedfile, identifying similarities and differences between them. 
The //Column operator formats the output into a vertical column for 
clarity. The output displays aligned segments: identical portions 
(e.g., \{l,q,l,d\}, \{x,f,e,u,l,u\}) are shown alongside divergent 
segments (e.g., \{\{d\},\{k\}\}, \{\{b\},\{x\}\}), indicating where mutations 
or differences occur in the sequences}
\label{}
\end{table}

The files only differ by 2 characters, which can be counted using the following code:

\begin{table}[H]
\begin{lstlisting}[style=MathematicaStyle]
In := Total[Length /@ 
First /@ Select[SequenceAlignment[randomfile, mutatedfile], Head[#[[1]]] == List &]]

Out = 2
\end{lstlisting}
\caption{Calculation of the total length of aligned sequence segments from
the comparison of \emph{randomfile} and \emph{mutatedfile} using \textbf{SequenceAlignment}. 
The Select function filters segments where the head of the first element 
is a List, ensuring only structured alignments are considered. 
The \textbf{Length/@First} computes the length of the first element in each 
selected segment, and \textbf{Total} sums these lengths. This metric quantifies 
the extent of aligned regions, useful for analysing sequence similarities 
in bioinformatics or data comparison studies.}
    \label{}
\end{table}

That is, 2/100 or 0.02 percent.\\
 
On the other hand, let's take a simple object consisting of the repetition of a single object, say the letter e:

\begin{table}[H]
\begin{lstlisting}[style=MathematicaStyle]
In := simplefile = Table["e",100] 
Out = {e,e,e,e,e,e,e,e,e,e,e,e,e,e,e,e,e,e,e,e,e,e,e,e,e,e,e,e,e,e,e,e,e,e,e,e,e,e,e,e,e,e,e,e,e,e,e,e,e,e,e,e,e,e,e,e,e,e,e,e,e,e,e,e,e,e,e,e,e,e,e,e,e,e,e,e,e,e,e,e,e,e,e,e,e,e,e,e,e,e,e,e,e,e,e,e,e,e,e,e}
\end{lstlisting}
\caption{Generation a sequence named \emph{simplefile} consisting of 100 
identical characters, specifically the letter "e".}
\label{}
\end{table}

A shortest program to generate such a file is just: 

\begin{table}[H]
\begin{lstlisting}[style=MathematicaStyle]
Table["e", 100]
\end{lstlisting}
\caption{Short probram for generated sequence of 100 characters, "e".}
\label{}
\end{table}

In other languages this could be produced by an equivalent \textbf{`For'} 
or \textbf{`Do-While'} program. We can now perturb the program again, 
without loss of generality. Let's allow the same 2 perturbations to 
the data only, and not to the program instructions (we will cover this 
case later). The only places that can be modified are thus `e' or 1 
instead of 5, say: \textbf{Table[``a'',500] }

\begin{table}[H]
\begin{lstlisting}[style=MathematicaStyle]
In := Length/@First/@Select[SequenceAlignment[Table["a",500],Table["e",100]],Head[#[[1]]]==List&] 
Out = {500} 
\end{lstlisting}
\caption{Calculation of similarity or divergence between two uniform 
sequences: one comprising 500 instances 
of the letter "a" and another with 100 instances of the letter "e", 
generated using the \textbf{Table} function. The \textbf{SequenceAlignment} 
function aligns these sequences, and \textbf{Select} filters segments
 where the head of the first element is a \textbf{List}, ensuring only 
 structured alignments are considered. The \textbf{Length/@First} 
 computes the length of the first element in each selected segment, 
 yielding a list of lengths suitable for analysing sequence similarity 
 or divergence.}
\label{}
\end{table}

Now, the original and decompressed versions differ by 500 elements, and 
not just a small fraction (compared to the total program length) 
as in the random case. This will happen in the general case with 
random and simple files; random perturbations will have a very
 different effect on each case.

 \textbf{An object that is highly integrated among its parts means that one 
can explain or describe part of each part with some other part when 
the object is algorithmically simple; then these parts can be 
compressed by exploiting the information that the said other 
parts carry over from yet others, and the resulting program will be 
highly integrated only if the removal of any of these parts has a 
non-linear effect on its generating program}. In a random system, 
no part contains any information about any other, and the distribution 
of the individual algorithmic-content contribution of each element 
is a normal distribution around the mean of the algorithmic-content 
contributions, hence poorly integrated and trivial. So integrated 
information is a measure of sophistication, filtering out simple 
and random systems, and only ascribing high algorithmic information 
content to highly integrated information systems.

The algorithmic information calculus thus consists of a 2-step 
procedure to determine:

\begin{enumerate}
    \item The complexity of the object (e.g. string, file, image) 
    \item  The elements in that object that are less, more, or not 
    sensitive to perturbations that can `causally steer the system,' 
    i.e. causally modify an object in a surgical algorithmic fashion 
    rather than on the basis of guesswork based on statistics.
\end{enumerate}

Note that this causal calculus is semi-computable, and one can perform 
guiding perturbations based upon approximations ~\cite{maininfo,mainbook}. 
Also note that we did not cover the case in which the actual instructions 
of the program were perturbed. This is actually just a subcase of the 
previous case, that separates data from program. For any program and data, 
however, we conceive an equivalent Turing machine with empty input, 
thus effectively embedding the data as part of its instructions.
Nevertheless, the chances of modifying the instruction \textbf{Print[]} 
in the random file case are constant, and for the specific example 
are: $7/107 = 0.0654$. While for the non-random case, the probability
of modifying any piece of the \textbf{Table[]} function is: $8/12 = 0.666667$. 
Thus, the break-up of a program of a highly causally generated system 
is more likely under random perturbations. 

Notice similarities to a checksum for, e.g., file exchange verification 
(e.g. from corruption or virus infection for downloading from the Internet), 
where the data to be transmitted is a program and the data block to 
check is the program's output file (which acts as a hash function).

Unlike regular checksums, the data block to check is longer than the 
program, and the checking is not for cryptographic purposes. 
Moreover, the dissimilarity distance between the original block 
(shared information) and the output of the actual shared program provides 
a measure of both how much the program was perturbed and the random 
or nonrandom nature of the data compressed by the program. And just 
like checksums, one cannot tamper with the program without perturbing 
the block to be verified (its output), without significantly changing 
the output (block) if what the program has encoded is nonrandom and 
therefore causally/recursively/algorithmically generated. Of course all 
the theory is defined in terms of binary objects, but for purposes of 
illustration and with no loss of generality we have shown actual 
programs operating on larger alphabets (ASCII). And we also decided 
to perform perturbations on what seems to be the program data and not 
the program itself (though we have seen that this distinction is not 
essential) for illustration purposes, to avoid the worst case in which 
the actual computer program becomes non-functional.

Yet, this means that the algorithmic calculus is actually more relevant, 
because it can tell us which elements in the program break it completely 
and which ones do not. But what happens when changes are made to the 
program output and not the program instructions? Say we exchange an 
arbitrary e for an a in our simple sequence consisting of a single 
letter, e.g. the third entry ('a' for `e'):

If we were to look to the generating program of the perturbed sequence,
this would need to account for the 'a', e.g. 

\begin{table}[H]
\begin{lstlisting}[style=MathematicaStyle]
In := ReplacePart[Table["e",100],3->"a"]
Out = {e,e,a,e,e,e,e,e,e,e,e,e,e,e,e,e,e,e,e,e,e,e,e,e,e,e,e,e,e,e,e,e,e,e,e,e,e,e,e,e,e,e,e,e,e,e,e,e,e,e,e,e,e,e,e,e,e,e,e,e,e,e,e,e,e,e,e,e,e,e,e,e,e,e,e,e,e,e,e,e,e,e,e,e,e,e,e,e,e,e,e,e,e,e,e,e,e,e,e,e}
\end{lstlisting}
\caption{Creation of a sequence of 100 identical characters, all "e", 
using the \textbf{Table} function, and then modifies it with \textbf{ReplacePart} 
to substitute the character at position 3 with "a". 
The result is a uniform sequence with a single variation.}
\label{}
\end{table}

\noindent where the second program is longer than the original one, and 
has to be, if the sequence is simple, but the program remains unchanged 
if the file is random because the shortest program of a random sequence
is the random sequence itself, and random perturbations keep the 
sequence random. Furthermore, every element in the simple example 
consisting of repetitions of `e' has exactly the same algorithmic 
content contribution when changed or removed, as all programs after 
perturbation are of the form: 

\begin{table}[H]
\begin{lstlisting}[style=MathematicaStyle]
ReplacePart[Table["e",100],n->x]
\end{lstlisting}
\caption{Code for replacement of the character at position \emph{n} with `x'}
\label{}
\end{table}

Notice also how this is related to $\phi$ and possibly any measure of 
integrated information based on the same principles. 

We can now apply all these ideas to the language of networks, with respect 
to which IIT has, for the most part, been defined. We have shown before 
that networks with different topologies have different algorithmic complexity 
values~\cite{physicaa}, in accordance with the theoretical expectation. 
In this way, random ER graphs, for example, display the highest values, 
while highly regular and recursive graphs display the lowest~\cite{zenilreview}. 
Some more probabilistic, but yet recursively generated graphs are located between
these 2 extremes~\cite{zenilmethods}. Indeed, the algorithmic complexity 
$K$ of a regular graph grows by $O(logN)$, where $N$ is the number of 
nodes of the graph, as in a highly compressible complete graph. Conversely, 
in a truly random ER graph, however, $K$ will grow by $O(\log E)$, where $E$ 
is the number of edges, because the information about the location of every 
edge has to be specified.

In what follows we will perform some numerical tests strengthening our 
analytic derivations.


\subsection{Finding simple rules in complex behaviour}

A perturbation test is applied to systems which IIT is interested
in. The set of answers is analyzed in order to find the rules that
1) make it possible to simulate the behaviour of the system, 
2) define their computability power, that is, rules that give an account of 
what the system can and cannot compute, and 
3) rules able to describe and predict behaviour of the same system. 
The following procedure was applied to estimate $\phi_{K}$. 

\begin{enumerate}
    \item The perturbation test was applied to systems used in IIT to 
        obtain detailed behaviour of the systems. 
    \item Results in step one were analyzed in order to reduce the 
        dynamics of a system to a set of simple rules. That is, in 
        keeping with the claims of  natural computation, we found 
        simple rules to describe a system's behaviour. 
    \item Rules found in step 2 were used to generate descriptions 
        of what a system is or is not capable of computing and under 
        what initial conditions, without having to calculate the whole 
        output repertoire. 
    \item A combination of rules found in steps 2 and 3 was used to develop
        procedures for predicting the behaviour of a system, that is, whether 
        it is possible to have reduced forms that express complex behaviour. 
        Knowing what conditions are necessary for the system to compute something,
        it is possible to pinpoint where in the whole map of all possible 
        inputs (questions) of a system such conditions may be found. 
    \item Once rules in steps 2 and 3 are formalized, $\phi_{K}$ was turned 
        into a kind of interrogator whose purpose was to ask questions of a 
        system about its own computational capabilities and behaviour.
\end{enumerate}

This kind of analysis allowed us to find that the information distribution 
in the complex behaviour of systems analyzed in IIT followed a \textcolor{black}{distribution
replicated at several scales that is usually and informally identified as 
`nested' or `fractal', and means that it is susceptible of being summarized 
in simple rules by iteration or recursion, just as is the case with fractals proper. 
These properties are used to find compressed forms to
express answers given by a system when asked for explanations of its
own behaviour.}

This means that, as noted before, $\phi_{K}$ does not compute the
whole output repertoire for a system but uses simple rules to express
the whole behaviour of the system. Interestingly, the way in which
we proceed appears to be connected to whether or not the system itself
can explain its behaviour, or rather whether it can see itself to be
capable of producing its behaviour from an internal experience (configuration)
which is then evaluated by an observer. So $\phi_{K}$ takes the form
of an automatic interrogator that, in imitation of the perturbation
test, asks questions of the form \emph{are you capable of this specific
configuration? (pattern), and if so, say where, in the map of the
behavioural repertoire, I can find it}.

The benefit of representing systems using simple rules is that it allows 
an alternative calculation closer to algorithmic complexity and the 
potential to reduce the number of calculations to derive an educated 
estimation as compared to the original version of IIT 3.0.

At this point, it is not possible to explain how simple rules define
a system in the context of $\phi_{K}$ without talking about the pattern
of distribution of information in the behaviour of systems like those studied in IIT. 

\subsection{Simple rules and the \textcolor{black}{pattern of }distribution of
information}

As shown in~\cite{zenil2015causality}, despite deriving from a very
simple program, without knowing the source code of the program, a
partial view and a limited number of observations can be misleading
or uninformative, illustrating how difficult it can be to reverse-engineer
a system from its evolution in order to discover its generating code~\cite{zenil2015causality}.

In the context of IIT, when we talk about a complex network we find
that there are different levels of understanding complex phenomena,
such as knowing the rules implemented by each node in a system and finding
the rules that describe its behaviour over time. To achieve the second,
as perhaps could be done for the ``whole {[}of{]} scientific practice''
\cite{zenil2015causality}, we found it useful to perform perturbation 
tests in order to deduce the behaviour of the subject systems. Results
were analyzed and a \textcolor{black}{pattern in the }distribution of
information was found to characterize the behaviour of these kinds of 
systems. Then, as was to be expected, replicating behaviours were
amenable to being expressed with simple formulae.

In order to explain how simple rules were found and implemented in
$\phi_{K}$, consider as an example the 7-node system shown in Figure
\ref{fig:7nodesNet} whose behaviour is computed by perturbing the system 
on all possible inputs. The code for the computational definition of the
network is shown in Appendix in Table \ref{Code:Def7NodeNet},
while the results, or the complete output repertoire, is shown in Figure 
\ref{fig:7NodeFullRepertoire} in the Appendix.

\begin{figure}[ht]´
    \begin{centering}
    \includegraphics[scale=0.4]{Capitulo4/figs/Fig07-7nodesSystem.png} 
    \includegraphics[scale=0.4]{Capitulo4/figs/Fig7B-Network7Nodes.png} 
    \par\end{centering}
    \caption{7-node system. \textbf{Left: } Adjacency matrix. \textbf{Right:} Network representation.}
    \label{fig:7nodesNet}
\end{figure}

\begin{table}[H]
\begin{lstlisting}[style=MathematicaStyle]
In := cm07 = {{0, 0, 1, 0, 0, 0, 1}, {0, 0, 1, 0, 0, 1, 0}, {1, 0, 0, 0, 1, 0, 1}, {1, 0, 1, 0, 1, 0, 1}, {0, 0, 1, 1, 0, 1, 1}, {1, 1, 1, 0, 0, 0, 0}, {0, 1, 0, 1, 1, 1, 0}};
In := dyn07 = {"AND", "OR", "OR", "AND", "OR", "OR", "AND"};
In := analysis07 = runDynamic[cm07, dyn07]["RepertoireOutputs"]
\end{lstlisting}
\caption{Mathematica code for computing the complete repertoire of 
inputs and outputs for a network, based on its dynamics. \textbf{Line 1:} 
Defines the adjacency matrix, where each row represents a node within the 
network. A value of one indicates a connection between a given node and a 
specific other node. \textbf{Line 2:} Specifies the dynamics, representing 
the function that each node executes upon receiving input from its connected 
nodes. \textbf{Line 3:} The `runDynamic` function calculates the exhaustive 
repertoire of inputs, determined by the network's order, feeds these 
into the defined network, and returns the corresponding output repertoire.
The output for this specific case is shown in Figure \ref{fig:7NodeFullRepertoire}}
\label{Code:Def7NodeNet}
\end{table}


The strategy adopted to find rules that govern a system's behaviour is the 
same used in almost any branch of science, which is to say we separately observe 
the behaviour of some of the components of a phenomenon, in this case nodes, 
while bearing in mind that this behaviour is not isolated but rather the by-product 
of interacting elements, or in other words, we observe individual behaviours 
without losing sight of the whole.

\textcolor{black}{When we observe the behaviour shown in Figure \ref{fig:7NodeFullRepertoire}
corresponding to the system shown in \ref{fig:7nodesNet}, we notice mostly chaotic behaviour but with subtle 
repetitions of certain patterns}. 

\begin{figure*}[htp]
	\begin{centering}
	\includegraphics[scale=0.5]{Apendice1/figs/FigTable2-Results7nodesSys.png} 
	\par\end{centering}
	\caption{Output repertoire for 7-Node network defined in Table \ref{Code:Def7NodeNet}.}
	\label{fig:7NodeFullRepertoire}
\end{figure*}

\textcolor{black}{When the behaviour of elements is isolated, the picture 
appears clearer. For example, Tables \ref{Table:isolatedNode4} and \ref{Table:isolatedNode5} 
shows the isolated behaviour of nodes 
\{4\} and \{5\} of the same subject system as consequence of running code in Table \ref{table:codeIsolationNodse4-5}}.

\begin{table}[H]
\begin{lstlisting}[style=MathematicaStyle]
analysis07[[All, 4]]
analysis07[[All, 5]]
\end{lstlisting}
\caption{Mathematica code for isolation of behaviour of node 4 and 5 of the 7-node system shown in \ref{fig:7nodesNet}.}
\label{table:codeIsolationNodse4-5}
\end{table}

\begin{table}[H]
\begin{lstlisting}[style=MathematicaStyle]
{0, 0, 0, 0, 0, 0, 0, 0, 0, 0, 0, 0, 0, 0, 0, 0, 0, 0, 0, 0, 0, 0, 0, 0, 0, 0, 0, 0, 0, 0, 0, 0, 0, 0, 0, 0, 0, 0, 0, 0, 0, 0, 0, 0, 0, 0, 0, 0, 0, 0, 0, 0, 0, 0, 0, 0, 0, 0, 0, 0, 0, 0, 0, 0, 0, 0, 0, 0, 0, 0, 0, 0, 0, 0, 0, 0, 0, 0, 0, 0, 0, 0, 0, 0, 0, 1, 0, 1, 0, 0, 0, 0, 0, 1, 0, 1, 0, 0, 0, 0, 0, 0, 0, 0, 0, 0, 0, 0, 0, 0, 0, 0, 0, 0, 0, 0, 0, 1, 0, 1, 0, 0, 0, 0, 0, 1, 0, 1}
\end{lstlisting}
\caption{Isolated outputs for node 4 of the 7-node system shown in \ref{fig:7nodesNet} after perturbation.}
\label{Table:isolatedNode4}
\end{table}

\begin{table}[H]
\begin{lstlisting}[style=MathematicaStyle]
{0, 0, 0, 0, 1, 1, 1, 1, 1, 1, 1, 1, 1, 1, 1, 1, 0, 0, 0, 0, 1, 1, 1, 1, 1, 1, 1, 1, 1, 1, 1, 1, 1, 1, 1, 1, 1, 1, 1, 1, 1, 1, 1, 1, 1, 1, 1, 1, 1, 1, 1, 1, 1, 1, 1, 1, 1, 1, 1, 1, 1, 1, 1, 1, 1, 1, 1, 1, 1, 1, 1, 1, 1, 1, 1, 1, 1, 1, 1, 1, 1, 1, 1, 1, 1, 1, 1, 1, 1, 1, 1, 1, 1, 1, 1, 1, 1, 1, 1, 1, 1, 1, 1, 1, 1, 1, 1, 1, 1, 1, 1, 1, 1, 1, 1, 1, 1, 1, 1, 1, 1, 1, 1, 1, 1, 1, 1, 1}
\end{lstlisting}
\caption{Isolated outputs for node 5 of the 7-node system shown in \ref{fig:7nodesNet} after perturbation.}
\label{Table:isolatedNode5}
\end{table}

% \textcolor{black}{In Table 2, the isolated behaviour of two nodes of the system in Figure 6 is shown,
%  where it is possible to observe that isolated behaviours for \{4\} and \{5\} follow a sort of order. 
%  Such patterns are summarized in what we call behaviour tables, shown in Figure 7.}

Notice that now that behaviour of nodes is isolated, it is possible to expressions
that describe the behaviour of the subsystems or isolated nodes.

Mathematical descriptions are shown in Figure \ref{fig:behaviourTables} in what we call \textbf{Behaviour Tables}.
It is important to note that these Behaviour Tables are used here just as a mathematical artifact to explain how 
regularities in the behaviour of the system emerge and can be explained in terms of the organisational definition
of the system itsef.

\begin{figure}
    \begin{centering}
    \includegraphics[scale=0.35]{Capitulo4/figs/Fig8-BehaviourTableNode4Sys7Nodes.png}
    \includegraphics[scale=0.4]{Capitulo4/figs/Fig9-BehavTableNode4Sys07Nodes.png} 
    \par\end{centering}
    \centering{}\caption{Behaviour Tables for 7-Node system shown in Figure \ref{fig:7nodesNet}, and
    defined computationally in \ref{Code:Def7NodeNet}.
    \textbf{Left:} Behaviour Table for node 4 that descrive output shown at \ref{Table:isolatedNode4}.
    \textbf{Right:} Behaviour Table for node 5 that descrive output shown at \ref{Table:isolatedNode5}.
    From left to right and up to down, \textbf{Node} column lists input-nodes that feed the target node. 
    \textbf{node-1=pow}: Stays as mnemonic for ``Content of \textbf{Node} column minus one, which determines the 
    power to use''. This column computes the power used to transform a pattern in the world of the 
    target n-node systems from binary to decimal. \textbf{2\textasciicircum (pow-1)} column is the result 
    of the binary to decimal transformation operation, and is equal to 2 powered by column \textbf{node-1=pow}. 
    The fourth column contains divisions between indexed elements of the column \textbf{2\textasciicircum (pow-1)}
    where $n$ is the index, whose value is equal to the element $n+1$ divided by the element indexed as $n$}
    \label{fig:behaviourTables}
\end{figure}


In behaviour tables shown in \ref{fig:behaviourTables} the lowest rows (within braces) correspond 
to a compressed representation of behaviours shown in Tables \ref{Table:isolatedNode4} and \ref{Table:isolatedNode5}.

Let's consider first the compressed expressions of behaviour for node \{4\} in Behaviour Table for node \{4\} shown
in the left side of the Figure \ref{fig:behaviourTables}. The compressed expression of the behaviour of the 
node \{4\} is shown in the expression \ref{CompressedBehaviourNode4}:

\begin{equation}
\label{CompressedBehaviourNode4}
\small{85 \rightarrow 0, \{\{\{1 \rightarrow 1, 1 \rightarrow 0\}\} \rightarrow 2\}, 4 \rightarrow 0\} \rightarrow 2, 16 \rightarrow 0, \{\{\{1 \rightarrow 1, 1 \rightarrow 0\}\} \rightarrow 2, 4 \rightarrow 0\} \rightarrow 2}
\end{equation}

The expression \ref{CompressedBehaviourNode4} must be readed as follow:

\begin{itemize}
    \item 85 occurrences of the digit 0, followed by
    \item Twice the patter $1, $0, followed by 4 repetitions of digit 0, followed by
    \item 16 repetitions of digit 0, followed by
    \item Twice the patter $1, $0, followed by 4 repetitions of digit 0.
\end{itemize}


Now, take in account that the purpose of Behaviour Tables and its compressed expressions are not 
to give a formal descritption of the isolated patters in wn in Tables \ref{Table:isolatedNode4} and 
\ref{Table:isolatedNode5}, but to create intuition on the hightlights of our method.

Notice also that in the explanation of the compressed expression \ref{CompressedBehaviourNode4} the 
following list of regularities for the column \textbf{2\textasciicircum (pow-1)} respect to 
occurrences of number zero:

\begin{itemize}
    \item 85 (the total sum of values in the column), followed by
    \item 1 acurrence of number zero, followed by
    \item 16 repetitions of digit 0, followed by
    \item 4 repetitions of the digit 0.
\end{itemize}
The occurrences of the digit 0 and  how they appear in the patter of the expression 
\ref{CompressedBehaviourNode4} correspond to the regularities found in the column 
\textbf{2\textasciicircum (pow-1)} in Behaviour Table for node \{4\}.

Same analysis can be done easily for node \{5\}.

Notice also that the representation used in this isolation of behaviours is expressed in terms of the 
nodes that ``feed'' into target nodes of this example (\textbf{Node} column in the Behaviour Tables shown
in \ref{fig:behaviourTables}), namely nodes \{4, 5\} 
whose inputs, according to Figure \ref{fig:behaviourTables} are: for node \{4\}: \{1,3,5,7\}, and for 
node 5: \{3,4,6,7\}.

\textcolor{black}{This first shallow analysis works to yield the intuition that the behaviour of 
an isolated node can be expressed as a series of regularities
in terms of its inputs. In this context, intuition tells us that
the greater the number of regularities, the shorter the description;
then if no patterns are detected the chances of a causal relationship
are lower.}

\textcolor{black}{Perspective changes when rule/algorithm or compressed expression 
of behaviour is not constructed from regularities identified at first sight, but from 
intrinsic algorithmic properties. In this latter case, behaviour of systems can be 
expressed as patterns of information with a distribution
replicable at different scales, what we here call  \emph{fractal representation}
or \emph{fractal behaviour}. To explain what we mean by fractal, we introduce characteristics 
of distribution of information for the 7-node system shown in Figure \ref{fig:7nodesNet} 
analyzed using $\phi_{K}$. This implementation is shown in Table \ref{Table:UnfoldingFractalInfo}.}

\begin{table}[H]
\begin{lstlisting}[style=MathematicaStyle]
In := cm07 = {{0, 0, 1, 0, 0, 0, 1}, 
                {0, 0, 1, 0, 0, 1, 0}, 
                {1, 0, 0, 0, 1, 0, 1}, 
                {1, 0, 1, 0, 1, 0, 1}, 
                {0, 0, 1, 1, 0, 1, 1}, 
                {1, 1, 1, 0, 0, 0, 0}, 
                {0, 1, 0, 1, 1, 1, 0}};
In := dyn07 = {"AND", "OR", "OR", "AND", "OR", "OR", "AND"};

(* Computing places in output repertoire where node 4 = 0 *)
In := res070 = onPossibleBehaviour[{4}, {0}, dyn07, cm07]

(* Summarized representation of fractal behaviour *)
In := gp = givePlaces[res070["DecimalRepertoire"], res070["Sumandos"]]

(* Compressed representation of fractal behaviour *)
Out = <|"DecimalRepertoire"-> {0, 1, 4, 5, 16, 17, 20, 21, 64, 65, 68, 69,80, 81, 84}, "Sumandos"-> {0, 2, 8, 10, 32, 34, 40, 42}|>

(* Unfolded representation of fractal behaviour *)
Out = {0, 1, 2, 3, 4, 5, 6, 7, 8, 9, 10, 11, 12, 13, 14, 15, 16, 17, 18, 19, 20, 21, 22, 23, 24, 25, 26, 27, 28, 29, 30, 31, 32, 33, 34, 35, 36, 37, 38, 39, 40, 41, 42, 43, 44, 45, 46, 47, 48, 49, 50, 51, 52, 53, 54, 55, 56, 57, 58, 59, 60, 61, 62, 63, 64, 65, 66, 67, 68, 69, 70, 71, 72, 73, 74, 75, 76, 77, 78, 79, 80, 81, 82, 83, 84, 86, 88, 89, 90, 91, 92, 94, 96, 97, 98, 99, 100, 101, 102, 103, 104, 105, 106, 107, 108, 109, 110, 111, 112, 113, 114, 115, 116, 118, 120, 121, 122, 123, 124, 126}
\end{lstlisting}
\caption{$\phi_{K}$ asking for accounts of information distribution in
behaviour of 4th node of the system shown in Figure \ref{fig:7nodesNet}. \textbf{Lines
1-8}: Definition of the 7-node system by means of adjacency matrix and
its internal dynamics. \textbf{Line 11:} $\phi_{K}$'s code asking
for zero digit location in the whole behaviour of node 4. \textbf{Line
17:} Compressing answer given by the system in line 11. \textbf{Line 20: } unfolded
answer of the system in Line 17.}
\label{Table:UnfoldingFractalInfo}
\end{table}


\textcolor{black}{Table \ref{Table:UnfoldingFractalInfo} shows how behaviour of 
the system shown in Figure \ref{fig:7nodesNet} can be expressed as simple rules 
following an analysis based on a querying scheme that results in a reduced 
form to express its information distribution as a pattern replicated at different 
scales or as a fractal form.
Answers given by systems join facts explored above on regularities and the fractal 
distribution of information. It is important to note 
that the querying scheme has to be computable and algorithmically
random in order to avoid introducing an artificially random-looking
behaviour from the observer (experimenter/interrogator) to the observed
(the system in question).}

\textcolor{black}{In \ref{Table:UnfoldingFractalInfo}, after defining the target 
system by means of an adjacency matrix and a dynamics vector (lines 1 to 8), $\phi_{K}$ can be regarded
as testing: \emph{how 0 is distributed in node \{4\} in the system of
seven nodes }(line 11).}

\textcolor{black}{The target system reacts to the $\phi_{K}$'s query and it ``answers''
in a compressed form (Table \ref{Table:UnfoldingFractalInfo}, line 17). The result
can be represented in compressed form, expressed as a tiny rule that
represents what we have called a fractal pattern. Such an expression is defined,
as can be seen in the  line 17 in Table \ref{Table:UnfoldingFractalInfo}, by two variables: 
\emph{DecimalRepertoire} that holds points distanced in different proportions where 
the patterns defined by the \emph{Sumandos} variable must be reproduced. 
This means that in order to unfold the whole distribution (of digit zero), the
pattern of numbers in \emph{Sumandos} must be added to each value
in \emph{DecimalRepertoire}.}

\textcolor{black}{Once this `fractal' simple rule is unfolded, we obtain the ordinal
places where, in the whole behaviour of node \{4\}, digit 0 can be
found (see line 20 in \ref{Table:UnfoldingFractalInfo}). The accuracy of this answer can
be verified by counting ordinals where, for node \{4\}, its output = 0 in
Table shown in the Figure \ref{fig:7NodeFullRepertoire}, taking into 
account that counting starts at 0.}

% Probably I can put here as example of integration, speed, complexity and perturbation
% the example of 16 nodes which taks just some secons to be computed, something impossible
% for phi implementation.
% also to show that we can ask for set of nodes, for the combination of questions
% thanks to the combination of questions.

In summary, $\phi_{K}$ is turned into a kind of interrogator that asks a system about 
its own behaviour. On the other hand, a system is an analyzer and self evaluator capable
to implement the set of rules that answers in different ways, depending on the information 
requested. This is unlikely with traditional approaches to $\phi$, whose representation 
of the system consists of the whole output repertoire of the system, which might represent 
an important disadvantage when large networks are analyzed. $\phi_{K}$'s answers use 
compressed forms taking advantage of the fractal distribution of the information in the
 behaviour of the system, for which the answering interface is a function of its input 
 related to each node in question.

Obviously the whole behaviour of a system is not about isolated elements, but about 
elements interacting in a non-linear manner, as IIT 3.0 makes clear. This last, broader 
view is also addressed in  terms of $\phi_{K}$, and explained in the following sections. 
In the next one, the advantages of simple rules over classical/naive approaches based 
on an exhaustive calculus and review of whole repertoires held in memory will be established.


\subsubsection{Automatic meta-perturbation test}

It can be seen that this querying system is similar to the programmability tests 
suggested in~\cite{zenilturingtest,zenilturingtest2,maininfo} based on questions designed 
to ascertain the programmability of a dynamical system.

The last section shows that systems implemented as simple rules that give rise to 
complex behaviour enable the system itself to ``respond'' to questions about where, 
in the chain of digits that conform to its behaviour (of a specific node), a certain 
pattern is to be found. And the fractal nature of information distribution in behaviour 
allows us to answer complex distribution questions in short forms. In this section, 
we show the advantages of using an (automatic) perturbation test based on simulation 
of behaviour using simple rules over the original version of IIT 3.0 based on the 
``bottleneck principle''
\cite{oizumi2014fromthe} in computing integrated information.

Taking up the original perturbation test, questions take the form: 
\emph{what is the output (answer) given this query (input)?.} 
But in $\phi_{K}$, since questions look for explanations of the behaviour of the 
system itself, they take the form: 
\emph{tell me if this pattern is reachable, and if so, tell me where, in the behavioural 
map, it is possible to find it}.

The following example aims to show that even is possible to implement an interrogator
and its corresponding analyzer that can answer questions about the behaviour of the
target system, $\phi_{K}$ surpaces the capabilities of traditional approaches as $\phi$
first and importantly minimizing the time and computational sources.

One naivie implementation of an analyzer for a system of 9 nodes is defined in Table \ref{Table:Naivie9NodeNet} 
while the structure of the network is show in Figure \ref{fig:9NodeNet}.
Table \ref{Table:Naivie9NodeNet} shows this approach using filters to find patters.
Lines 3-12 defines the adjacency matrix and dynamics vector of the system. Line 14 
runs the internal dynamics over the exhaustive repertoire of inputs obtaining the exhaustive
repertoire of outputs. Then lines 16-18 filter over all outputs cases where nodes \{8, 9\} = \{1,1\}.


\begin{figure}[ht!]
    \begin{centering}
    \includegraphics[scale=0.4]{Capitulo4/figs/Figure9NodesNetwork.png} 
    \par\end{centering}
    \caption{Network example with 9 nodes}
    \label{fig:9NodeNet}
\end{figure}

\begin{table}[H]
\begin{lstlisting}[style=MathematicaStyle]
mmu=MemoryInUse[];
AbsoluteTiming[
In := cmTest = {{0,1,1,1,1,1,1,1,1},
                {1,0,1,1,1,1,1,1,1},
                {1,1,0,1,1,1,1,1,1},
                {1,1,1,0,1,1,1,1,1},
                {1,1,1,1,0,1,1,1,1},
                {1,1,1,1,1,0,1,1,1},
                {1,1,1,1,1,1,0,1,1},
                {1,0,1,0,1,1,0,0,0},
                {1,0,1,0,1,0,1,0,0}};
In := dynTest = {"OR","XOR","AND","XOR","AND","AND","OR","OR", "AND"};

In := tdTest=runDynamic[cmTest, dynTest];

In := inTest = tdTest["RepertoireInputs"];
In := outTest = tdTest["RepertoireOutputs"];

(*looking for pattern where {8,9}={1,1}*)
In :=  For[i=1, i<=Length[outTest],i++,
    If[(outTest[[i]][[8]]===1 && outTest[[i]][[9]]===1), 
        outTest[[i]]=Style[outTest[[i]],{Bold,Red}]
    ];
];
(*Showing results*)
In :=  assoc=AssociationThread[inTest, outTest]
]
MemoryInUse[]-mmu
\end{lstlisting}
\caption{Mathematica code for definition of the 9-Node system shown in Figure \ref{fig:9NodeNet}.
\textbf{Lines 1-12:} Definition of the 9-Node system by means of adjacency matrix and its internal dynamics.
\textbf{Line 14:} Instruction for runing the internal dynamics overl exhaustive input repertoire.
\textbf{Lines 17, 18:}: Retrieving exhaustive repertoores of inputs and outputs.
\textbf{Lines 20-24:} Filtering over all cases where \{8,9\}={1,1}.
\textbf{Lines 1, 28:} Measuring time and memory used by the program.
}
\label{Table:Naivie9NodeNet}
\end{table}
    
Results of running the code in \ref{Table:Naivie9NodeNet} are shown in \ref{Table:OutNaivie9NodeNet}, where
time in seconds is given, followed by the cases in repertoire of inputs and outputs where where the
pattern \{8,9\}={1,1} is found. Finally the memory used by the program is given in bytes.

\begin{table}[H]
\begin{lstlisting}[style=MathematicaStyle]
Out = 0.024581
Out = <|runDynamic[{{0, 1, 1, 1, 1, 1, 1, 1, 1}, {1, 0, 1, 1, 1, 1, 1, 1, 
    1}, {1, 1, 0, 1, 1, 1, 1, 1, 1}, {1, 1, 1, 0, 1, 1, 1, 1, 
    1}, {1, 1, 1, 1, 0, 1, 1, 1, 1}, {1, 1, 1, 1, 1, 0, 1, 1, 
    1}, {1, 1, 1, 1, 1, 1, 0, 1, 1}, {1, 0, 1, 0, 1, 1, 0, 0, 
    0}, {1, 0, 1, 0, 1, 0, 1, 0, 0}}, {"OR", "XOR", "AND", "XOR", 
    "AND", "AND", "OR", "OR", "AND"}]["RepertoireInputs"] -> 
Out = runDynamic[{{0, 1, 1, 1, 1, 1, 1, 1, 1}, {1, 0, 1, 1, 1, 1, 1, 1, 
    1}, {1, 1, 0, 1, 1, 1, 1, 1, 1}, {1, 1, 1, 0, 1, 1, 1, 1, 
    1}, {1, 1, 1, 1, 0, 1, 1, 1, 1}, {1, 1, 1, 1, 1, 0, 1, 1, 
    1}, {1, 1, 1, 1, 1, 1, 0, 1, 1}, {1, 0, 1, 0, 1, 1, 0, 0, 
    0}, {1, 0, 1, 0, 1, 0, 1, 0, 0}}, {"OR", "XOR", "AND", "XOR", 
    "AND", "AND", "OR", "OR", "AND"}]["RepertoireOutputs"]|>}
Out = 17776
\end{lstlisting}
\caption{Results of running code shown in Table \ref{Table:Naivie9NodeNet}. This outputs are 1) execution 
time in seconds, 2) filtered input cases where \{8,9\}={1,1} is found, 3) filtered output cases 
where \{8,9\}={1,1} is found, and 4) memory used by the programin bytes}.
\label{Table:OutNaivie9NodeNet}
\end{table}

For comparing purposes, ono more things have to be pointed respect the results shown in 
Table \ref{Table:Naivie9NodeNet}, and this is the format of the outputs and inputs, where
the rough cases are shown. This is, with no showing null analysis or highlights that give
and idea how these inputs and outputs are related with the targeted nodes.


The implementation of $\phi_{K}$ for specifically turnint it into an automatic interrogator 
is shown in Table \ref{Table:phiK_implementation}. Notice that this approach is also based on 
analyze the system networks shown in Figure \ref{fig:9NodeNet}. 
In line 1, in the list code shown in \ref{Table:phiK_implementation}, it is possible to see how 
at low level of implementation $\phi_{K}$ asks questions of a system. This line should be
interpreted as, \emph{Can you compute the pattern \{8,9\} = \{1,1\} when \{8,9\}-> \{"OR'', "AND''\}? 
If yes, tell me under what conditions you can do so}.
Notice that here we are showing the low level implementation of this function. What really happens
is that our high level implementation easily identiy input nodes of the targeted nodes, and its internal
dynamics from the especification of the system.

\begin{table}[H]
\begin{lstlisting}[style=MathematicaStyle]
In := combiningRepersWithSharedInputs[{1, 3, 5, 6}, "OR", 1, {1, 5, 7, 3}, "AND", 1]

Out = 0.000736
Out = <|"Combination" -> {{1, 0, 0, 0, 0}, {1, 0, 0, 0, 1}, {0, 1, 0, 0, 
    0}, {0, 1, 0, 0, 1}, {1, 1, 0, 0, 0}, {1, 1, 0, 0, 1}, {0, 0, 1, 
    0, 0}, {0, 0, 1, 0, 1}, {1, 0, 1, 0, 0}, {1, 0, 1, 0, 1}, {0, 1, 
    1, 0, 0}, {0, 1, 1, 0, 1}, {1, 1, 1, 0, 0}, {1, 1, 1, 0, 1}, {0, 
    0, 0, 1, 0}, {0, 0, 0, 1, 1}, {1, 0, 0, 1, 0}, {1, 0, 0, 1, 
    1}, {0, 1, 0, 1, 0}, {0, 1, 0, 1, 1}, {1, 1, 0, 1, 0}, {1, 1, 0, 
    1, 1}, {0, 0, 1, 1, 0}, {0, 0, 1, 1, 1}, {1, 0, 1, 1, 0}, {1, 0, 
    1, 1, 1}, {0, 1, 1, 1, 0}, {0, 1, 1, 1, 1}, {1, 1, 1, 1, 0}, {1, 
    1, 1, 1, 1}}, 
"Filtered" -> {{1, 1, 1, 0, 1}, {1, 1, 1, 1, 1}}, 
"jn" -> {1, 3, 5, 6, 7}, 
"nn" -> {1, 2, 3, 4, 5}, 
"DecRep" -> {85, 117}|>

Out = 2440
\end{lstlisting}
\caption{$\phi_{K}$ algorithmic querying of the system about its own behaviour as shown in Figure \ref{fig:9NodeNet}. 
\textbf{Line 1}: Query: Is it possible for this system to compute \emph{\{8,9\} = \{1,1\} when \{8,9\}-> \{"OR'', "AND''\} and whose input nodes are \{1,3,5,6\} and \{1,5,7,3\}
respectively?. } The results show that the system does compute 1) time of computation in seconds, 2) possible candidates for combination of conditions,
3) The especific input patters for specific notes ("Filtered" and "jn" keys), this is when \{1,3,5,6,7\} = \{\{1,1,1,0,1\},\{1,1,1,1,1\}\}
4) Decimal representations of summandos to use for unfolding the complete dynamics of the system and 5) the memory used in bytes.}

\label{Table:phiK_implementation}
\end{table}


In this example, in the first place $\phi_{K}$ tries to find conditions needed to 
compute a specific output. As Table \ref{Table:phiK_implementation} shows, the answer 
is: \emph{Yes! I can. This may happen when \{1,3,5,6,7\} = \{\{1,1,1,0,1\},\{1,1,1,1,1\}\}}. 
In this answer \{1,3,5,6,7\} is the set of inputs to the subsystem \{8,9\}.

The reader would note here that the answers offered by the system actually are conditions 
or inputs needed by the system to compute specific input in a format equivalent to Holland's schemas. 
The schemas' equivalent form for this case would be: \{\{1,{*},1,{*},1,0,1,{*},{*}\},
\{1,{*},1,{*},1,1,1,{*},{*}\}\}, where `{*}' is a wildcard that means 0/1 (any symbol). 
Such schemas correspond exactly to the generalized answer offered by the system, 
that is: \{1,3,5,6,7\} = \{\{1,1,1,0,1\},\{1,1,1,1,1\}\}.

This answer, like the Holland's schema theorem \cite{holland1975adaptation}, works by 
imitating genetics, where a set of genes are responsible for specific features in phenotypes. 
What $\phi_{K}$ retrieves is the general information that yields specific inputs 
for the current system.

Probably the greatest advantage of the approach using $\phi_{K}$ in querying samples
has to do with the computation time needed to retrieve such information, as compared 
with a traditional/naivie (Table \ref{Table:OutNaivie9NodeNet}) approach: 1/10 in the
case shown just right above. But this effects are magnified when the number of nodes
increases, this due the fractal distriution of the information. This is causede by
the continous addition of nodes into a particular analysis, since is possible to 
know that, if a system is strongly integrated, most of the parts of the system definition
and its behaviour is involved in generating patters and answers about itself.
This is shown in Table \ref{Table:16NodeNet}, that shows the definition and results of
runing $\phi_{K}$ quering on a network of 16 nodes, a size untractable but the implementation 
of IIT 3.0.
In the results shown in Table \ref{Table:16NodeNet} the calculation amount of time and memory
required for computation of answers are shown in lines 26 and 27, respectively.


\begin{table}[H]
\begin{lstlisting}[style=MathematicaStyle]
mmu=MemoryInUse[];
AbsoluteTiming[
cm16 = {{0, 0, 0, 0, 0, 0, 1, 0, 0, 0, 0, 0, 1, 0, 0, 0}, 
        {0, 0, 1, 0, 0, 1, 0, 0, 0, 1, 0, 0, 0, 1, 0, 1},
        {0, 0, 0, 0, 0, 0, 0, 0, 1, 0, 0, 0, 1, 0, 1, 0},
        {1, 0, 1, 0, 1, 0, 1, 0, 0, 0, 0, 1, 0, 0, 0, 0},
        {0, 0, 0, 0, 0, 0, 0, 0, 1, 0, 1, 0, 0, 0, 1, 0}, 
        {0, 1, 0, 0, 0, 0, 0, 0, 0, 0, 0, 0, 1, 0, 0, 0},
        {0, 0, 0, 1, 0, 0, 0, 0, 1, 1, 0, 0, 0, 1, 0, 1},
        {1, 0, 1, 0, 0, 0, 1, 0, 0, 0, 0, 1, 0, 0, 0, 0},
        {0, 0, 1, 0, 0, 1, 0, 0, 0, 1, 0, 0, 0, 1, 1, 0}, 
        {1, 0, 0, 0, 0, 0, 0, 0, 1, 0, 0, 1, 0, 0, 1, 0}, 
        {0, 0, 0, 1, 0, 0, 0, 0, 0, 0, 0, 0, 0, 0, 0, 1},
        {0, 0, 0, 0, 0, 0, 1, 0, 0, 0, 0, 0, 1, 0, 1, 0},
        {0, 1, 0, 0, 0, 0, 0, 1, 0, 0, 1, 0, 0, 0, 0, 0},
        {0, 0, 0, 0, 1, 0, 0, 0, 0, 1, 0, 0, 0, 0, 0, 0},
        {1, 0, 0, 1, 0, 0, 0, 0, 0, 0, 1, 0, 1, 0, 0, 0},
        {0, 1, 0, 0, 0, 0, 0, 1, 0, 0, 0, 0, 0, 0, 1, 0}};

dyn16 = {"AND", "OR", "OR", "AND", "OR", "OR", "AND", "OR", "OR", "AND", "OR", "OR", "AND", "AND", "OR", "AND"};
res = onPossibleBehaviour[{5, 7, 9, 10}, {0, 0, 1, 0}, dyn16, cm16];
givePlaces[res["DecimalRepertoire"], res["Sumandos"]];]
]
MemoryInUse[]-mmu

Out = {0.000011, {7.*10^-6, Null}}
Out = 5888

\end{lstlisting}
\caption{Definition of a network of 16 nodes and its results of running $\phi_{K}$ quering. Results shown the 
effect of the fractal distribution and the level of integration in a network}
\label{Table:16NodeNet}
\end{table}
    

This last suffices as proof that compression and generalization of systems in the form 
of simple rules based on naturally fractal information distribution has advantages over 
common sense or classical approaches to the analysis of complex systems, particularly
in terms of the computational resources needed to compute integrated information.

All the above were applied to analyzing isolated or very simple cases.
In the next section the generalization of $n$ nodes of the system is addressed, and how 
this works to compute integrated information according to IIT.

\subsubsection{Shrinking after dividing to rule}

In previous sections it was shown how $\phi_{K}$, applying a perturbation test, 
can deduce, firstly, what a system is capable of computing and the conditions under 
which a computation could be performed, and secondly, that by means of simple rules 
specifying a system it is possible to obtain descriptions of its behaviour in the form 
of rules that say how information is distributed, or in other words, where, in ordinal terms, 
such conditions can be found.

The ultimate objective of obtaining this kind of description of the behaviour of a system is 
to know how many times specific patterns appear in whole repertoires, and thus to construct 
probability distributions without need of exorbitant computational resources, since 
these probability distributions are a key piece used by IIT to compute integrated information.

$\phi_{K}$ addressed such challenges using a two-pronged strategy consisting firstly of 
parallelizing the analytical process-- which is no more than a technical strategy available 
to be implemented in almost any computer language and that falls beyond the scope of 
this paper--and secondly of the partition of the target sets. This latter part of $\phi_{K}$'s 
strategy consists of two parts: 1) given a target set to be analyzed, to divide this into 
parts to be interrogated by $\phi_{K}$ via the implementation of an automatic test, 
and 2) to find the MIP or the Maximal Information Partition using the algorithm 
proposed and proved by Oizumi in \cite{kitazono2018efficient}.

In the context of $\phi_{K}$, when a partition of a subject system is being analyzed, 
the search space for the remaining parts is significantly reduced, facilitating and 
acelerating the analysis of the remaining parts.

In order to illustrate this idea, take for example the code shown in Table \ref{Table:cosecitiveQuering} and its
restuls in Table \ref{Table:resultsConsecitiveQuering}.

\begin{table}[H]
\begin{lstlisting}[style=MathematicaStyle]
cm07 = {
{0, 0, 1, 0, 0, 0, 1},
{0, 0, 1, 0, 0, 1, 0},
{1, 0, 0, 0, 1, 0, 1},
{1, 0, 1, 0, 1, 0, 1},
{0, 0, 1, 1, 0, 1, 1},
{1, 1, 1, 0, 0, 0, 0},
{0, 1, 0, 1, 1, 1, 0}};
dyn07 = {"AND", "OR", "OR", "AND", "OR", "OR", "AND"}; 
onPossibleBehaviour[{1,2,3},{0,0,0},dyn07,cm07]//AbsoluteTiming
onPossibleBehaviour[{2,3,4},{0,0,0},dyn07,cm07]//AbsoluteTiming
onPossibleBehaviour[{1,2,3,4},{0,0,0,0},dyn07,cm07]//AbsoluteTiming
onPossibleBehaviour[{5,6,7},{0,0,0},dyn07,cm07]//AbsoluteTiming
onPossibleBehaviour[{1,2,3,4,5,6,7},{0,0,0,0,0,0,0},dyn07,cm07]//AbsoluteTiming
\end{lstlisting}
\caption{Definition of a network of 7 nodes and consecutive application of $\phi_{K}$ quering at different
levels of complexity.}
\label{Table:cosecitiveQuering}
\end{table}


\begin{table}[H]
\begin{lstlisting}[style=MathematicaStyle]
{0.00076,<|"DecimalRepertoire"->{0},"Sumandos"->{0,2,8,10}|>}
{0.000647,<|"DecimalRepertoire"->{0},"Sumandos"->{0,2,8,10}|>}
{0.0009,<|"DecimalRepertoire"->{0},"Sumandos"->{0,2,8,10}|>}
{0.000656,<|"DecimalRepertoire"->{0,16},"Sumandos"->{0}|>}
{0.001248,<|"DecimalRepertoire"->{0},"Sumandos"->{0}|>}
\end{lstlisting}
\caption{Comparing processing time when a system is divided to compute outputs. \textbf{Line 10-14:} 
$\phi_{K}$ asking the system defined in lines 1-9 for patterns filled with zeros with 
different lengths (3, 4 and 7) and combinations. \textbf{Lines 1-5} show, time in 
seconds taken for computations and answers in terms of indexes using 
compressed notation. In first data of this results square it
can be observed that the larger the node wanted, the greater the amount of time 
required to perform the computation, while the time ratio decreases.}
\label{Table:resultsConsecitiveQuering}
\end{table}

Table \ref{Table:cosecitiveQuering} shows the definition of a system of 7 nodes (lines 1-9), 
where a set of a progressively growing length is searched (lines 10-14). 
In this example $\phi_{K}$  repeatedly asks the system if it is capable of finding a 
growing pattern of zeros. If it is, the system is requested to show where it is possible 
to find the desired pattern. Obviously, larger patterns need more computations, but as can be seen in 
Table \ref{Table:resultsConsecitiveQuering}, in the results square, the time used by $\phi_{K}$ 
increases as the pattern's length increases, but it grows linearly in contrast to IIT 3.0, 
where it grows exponentially. 


%%%%%%%%%%%%%%%%%%%%%%%%%%%%%%%%%%%%%%%%%%%%%%%%%%%%%%%%%%
%%%%%%%%%%%%%%%%% NEW SECTION %%%%%%%%%%%%%%%%%%%%%%%%%%%%
%%%%%%%%%%%%%%%%%%%%%%%%%%%%%%%%%%%%%%%%%%%%%%%%%%%%%%%%%%

\chapter{Integration in Complex Networks Using Algorithmic Complexity}

\section{Metacompression}
In the study of complex systems, measuring integration—the extent to which components 
of a system work collectively to form a unified whole—is a fundamental challenge. 
Integrated Information Theory (IIT), proposed by Tononi, quantifies integration 
through $\phi$, which measures the information generated by a system beyond the 
sum of its parts by identifying the Minimum Information Partition (MIP). However, 
IIT's computational complexity grows exponentially with system size ($2^N$ for $N$ 
nodes), rendering it infeasible for large-scale networks. Inspired by Zenil's work 
\cite{zenil2015causality}, which explores algorithmic complexity as a tool for analysing 
network dynamics, we introduce a novel measure of integration, $\phi_K$, that leverages 
the intrinsic spreading of information within a fractal dynamic, reflected in attractors. 
These attractors represent the natural convergence points of information flow, as defined 
by the inherent structure of a system or network. By employing concise programs derived 
from querying activities, the fractal distribution of information and algorithmic complexity 
can be utilised to efficiently capture the essence of integration..

Our approach is a pattern-based perturbation approach, that inpired on natural distribution
of infomation expressed as fractal distribution that evolved to an attractor-based method,
and culminated in a compressed, rule-based formulation that avoids IIT's exhaustive 
partitioning. This section details the conceptual evolution, mathematical 
formulations, and implementation in Mathematica, culminating in a scalable method to 
measure integration in networks of varying sizes and topologies.

\subsection{Step 1: Initial Pattern-Based Approach}
We start with a network of \(N\) nodes and \(E\) edges, aiming to measure 
integration by perturbing the network and observing its response. 
For a 9-node network with 20 edges, we initially proposed querying the network's ability 
to produce all \(2^N = 2^9 = 512\) possible binary output patterns 
(e.g., \(\{1,2\} = \{0,0\}\), \(\{3,4,5\} = \{1,0,1\}\)) (see Table \ref{table:9NodeSchemataConditions}). 
For each pattern, we would:

\begin{enumerate}
    \item Compute the rule (program) needed to generate the pattern, 
    measuring its complexity \(K(s_0)\) in bytes 
    (see bellow).
    \item Remove one edge, recompute the rule, and measure the new complexity \(K(s_1)\).
    \item Calculate the sensitivity as \(\Delta K = |K(s_0) - K(s_1)|\).
\end{enumerate}

For each edge (\(E = 20\)) and pattern (\(2^9 = 512\)), this required 
\(20 \times 512 = 10,240\) simulations per network, totaling 1,024,000 for 100 networks. 
While more efficient than IIT's \(2^{N-1}\) bipartitions 
(e.g., \(2^8 = 256\) for 9 nodes), this was still computationally expensive.
 We also considered partitioning the network (like IIT's MIP), but this would replicate 
 IIT's exponential cost (\(B_N\), the Bell number), defeating our goal of efficiency.

To reduce costs, we proposed a hybrid approach:
\begin{itemize}
    \item Select a subset of patterns (e.g., 10 subset patterns + 1 whole-system pattern, 
    such as \(\{1,2\} = \{0,0\}\), \(\{1,2,3,4,5,6,7,8,9\} = \{0,1,0,1,0,1,0,1,0\}\)).
    \item Compute \(\Delta K\) for each edge and pattern, averaging to get \(\phi_K\):
    \[
    \Delta K_i = \frac{1}{P} \sum_{j=1}^P \Delta K_{i,j}, \quad \phi_K = \frac{1}{E} \sum_{i=1}^E \Delta K_i
    \]
    where \(P = 11\) patterns, \(E = 20\) edges. This reduced simulations to \(20 \times 11 = 220\) per network, or 22,000 for 100 networks—a significant improvement.
\end{itemize}

However, the pattern selection was arbitrary, and the approach still required multiple queries, 
potentially missing the network's intrinsic dynamics.

\subsection{Step 2: Shifting to Attractors}
Recognizing the inefficiency of querying all \(2^N\) patterns, we shifted to a more natural 
approach: using the network's attractors—the set of stable states (fixed points or cycles) 
under its dynamics (e.g., Boolean rules like AND, OR). Attractors represent the system's 
possible outputs without exhaustively testing all input-output pairs, using minimal descriptions.

For a 5-node network defined by:
\begin{itemize}
    \item Adjacency matrix: \(\text{cm05} = 
    \begin{pmatrix} 
        0 & 1 & 1 & 1 & 1 \\ 1 & 1 & 1 & 0 & 0 \\ 1 & 0 & 0 & 1 & 0 \\ 1 & 1 & 0 & 0 & 0 \\ 0 & 1 & 1 & 1 & 0 
    \end{pmatrix}\)
    \item Dynamics: 
    \(\text{dyn05} = \{\text{``AND''}, \text{``AND''}, \text{``OR''}, \text{``AND''}, \text{``OR''}\}\)
\end{itemize}

Tononi's approach yields \(2^5 = 32\) output patterns, many redundant 
(e.g., \(\{0,0,0,0,0\}\) at positions 0, 16). Our attractor method, using algorithmic 
complexity principles, compressed this to 8 unique attractors:
\[
\text{AttractorsByPosition} = \langle| 0 \to \{0, 16\}, 4 \to \{1, 17\}, 16 \to \{2, 4, 6, 18, 20, 22\}, \ldots, 31 \to \{31\} |\rangle
\]
where keys are decimal representations of binary states (e.g., \(0 = \{0,0,0,0,0\}\)), 
and values are positions in the full repertoire.

This reduced the problem space from 32 to 8 queries per perturbation. We proposed:
\begin{enumerate}
    \item Compute the baseline attractor map, measure its complexity \(K(s_0)\) (e.g., using \(\text{ByteCount}\) in Mathematica).
    \item Perturb each edge, recompute attractors, and measure \(K(s_1)\).
    \item Compute \(\Delta K_i = |K(s_0) - K(s_1)|\) and \(\Delta A_i = |A_0 - A_1|\) (change in attractor count).
    \item Define sensitivity per edge as:
    \[
    \text{Sensitivity}_i = \Delta K_i \cdot \log_2(\Delta A_i + 1)
    \]
    where \(\log_2\) scales the attractor count change informationally, and \(+1\) ensures the term is defined when \(\Delta A_i = 0\).
    \item Compute \(\phi_K\) as the average sensitivity:
    \[
    \phi_K = \frac{1}{E} \sum_{i=1}^E \text{Sensitivity}_i
    \]
\end{enumerate}

For 5 nodes and 11 edges, this required 1 baseline + 11 perturbations = 12 simulations 
per network, or 1,920 for 160 networks—a massive improvement over 10,240.


\subsection{Why Attractors?}
\begin{itemize}
    \item They capture the network's intrinsic dynamics, avoiding arbitrary pattern queries.
    \item Perturbing edges tests integration causally: high integration means large 
    \(\Delta K\) and \(\Delta A\) (attractors collapse), while low integration means 
    small changes (attractors persist).
    \item No need for IIT's partitions—edge perturbations act as localized cuts, 
    approximating the MIP's effect dynamically.
\end{itemize}

\subsection{Step 3: Rule-Based Refinement}
The attractor map provided a global view, but we noticed that each attractor could be 
further compressed into a rule describing its positions in the repertoire—a meta-compression 
explained as a fractal distribution of the information summarized as \(\text{DecimalRepertoire}\) and 
\(\text{Sumandos}\) (Table \ref{Table:UnfoldingFractalInfo}). For the attractor \(\{0,0,0,0,0\}\):

\[
\text{Rule} = \langle| \text{``DecimalRepertoire''} \to \{0, 16\}, \text{``Sumandos''} \to \{0\} |\rangle
\]
This rule encodes how to generate the attractor's positions, offering a finer measure of dynamic sensitivity.

We refined \(\phi_K\) by:
\begin{enumerate}
    \item Computing rules for each attractor before and after perturbation.
    \item Measuring rule complexity \(K_{\text{rule}}\) (via \(\text{ByteCount}\)), averaging across attractors:
    \[
    K_{\text{rule,avg}} = \frac{1}{A} \sum_{a=1}^A K_{\text{rule}}(a)
    \]
    \item Calculating the rule sensitivity:
    \[
    \Delta K_{\text{rule,avg}} = |K_{\text{rule,avg}}(s_0) - K_{\text{rule,avg}}(s_1)|
    \]
    \item Combining with the map sensitivity using a weight \(w = 1\):
    \[
    \phi_K = \left( \frac{1}{E} \sum_{i=1}^E \text{Sensitivity}_i \right) + w \cdot \Delta K_{\text{rule,avg}}
    \]
\end{enumerate}

This dual approach—global (map) and local (rules)—captures integration at multiple scales, 
enhancing \(\phi_K\)'s sensitivity to subtle dynamic changes.

\subsection{Implementation in Mathematica}
We implemented this approach for some networks of variable architecture and 
dynamics, generating \(n\) networks per size and type (Barabási-Albert, Watts-Strogatz, Ring, Complete).
The detailed algorithm can be seen at algorithm \ref{algo-fullMetacompression}.

A high level run can be see as follows:

\begin{enumerate}
    \item \textbf{Network Generation}:
    \begin{itemize}
        \item For each node size \(N\) (\emph{min} to \emph{max}), generate \(E\) edges randomly.
        \item Create graphs using:
        \begin{itemize}
            \item Barabási-Albert (scale-free)%: \(\text{RandomGraph[BarabasiAlbertGraphDistribution[nodeSize, noEdges]]}\)
            \item Watts-Strogatz (small-world)%: \(\text{RandomGraph[WattsStrogatzGraphDistribution[nodeSize, 0.3]]}\)
            \item Ring%: \(\text{CycleGraph[nodeSize]}\)
            \item Complete%: \(\text{CompleteGraph[nodeSize]}\)
        \end{itemize}
        \item Extract adjacency matrix%: \(\text{adjMat = AdjacencyMatrix[ranGraph] // Normal}\).
    \end{itemize}

    \item \textbf{Dynamics}:
    \begin{itemize}
        \item Assign random Boolean rules%: \(\text{dyn = createSimpleRandomDynamic[nodeSize][``Dynamic'']}\).
    \end{itemize}

    \item \textbf{Full Repertoire (for PyPhi)}:
    \begin{itemize}
        \item Compute all \(2^N\) output patterns%: \(\text{outRep = runDynamic[adjMat, dyn][``RepertoireOutputs'']}\).
    \end{itemize}

    \item \textbf{Random State}:
    \begin{itemize}
        \item Generate a random input and output%: \(\text{oneInput = createBinRandomInput[nodeSize]}\), \(\text{oneOut = calculateOneOutputOfNetwork[oneInput, adjMat, dyn][``Output'']}\).
    \end{itemize}

    \item \textbf{Attractors}:
    \begin{itemize}
        \item Compute attractors%: \(\text{ff = calculatingPattsInDivisionsOfCM[adjMat, dyn, 2]}\), \(\text{binAttractors = calculatingAttractors[ff[``Locations''], ff[``Sumandos'']][``BinAttractorsByPosition'']}\).
        \item Measure complexity%: \(K(s_0) = \text{ByteCount}[binAttractors]\), \(A_0 = \text{Length}[binAttractors]\).
    \end{itemize}

    \item \textbf{Perturbations}:
    \begin{itemize}
        \item For each edge, set \(\text{adjMat[i,j] = 0}\), recompute attractors, and calculate:
        \[
        \Delta K_i = |K(s_0) - K(s_1)|, \quad \Delta A_i = |A_0 - A_1|, \quad \text{Sensitivity}_i = \Delta K_i \cdot \log_2(\Delta A_i + 1)
        \]
    \end{itemize}

    \item \textbf{Rule Refinement}:
    \begin{itemize}
        \item Compute rules%: \(\text{rules0 = Map[onPossibleBehaviour[Range[nodeSize], \#, dyn, adjMat] \&, Keys[binAttractors]]}\).
        \item Measure \(\Delta K_{\text{rule,avg}}\) and add to \(\phi_K\) with \(w = 1\).
    \end{itemize}

    \item \textbf{Save Data}:
    \begin{itemize}
        \item Export to \texttt{network\_data.csv} with NumPy-compatible formatting for PyPhi comparison.
    \end{itemize}
\end{enumerate}

\textbf{Scalability}: For 5 nodes (11 edges), 12 simulations per network; for 1000 networks, 
\(\sim 12,000\) simulations—orders of magnitude fewer than IIT's \(2^{N-1}\) bipartitions.

\subsection{insights on Mathematica code}
\begin{itemize}
    \item \textbf{Attractors Over Patterns}: Attractors reflect intrinsic dynamics, 
    reducing queries from \(2^N\) to the number of unique stable states.
    \item \textbf{Algorithmic Complexity}: Using \(\text{ByteCount}\) as a proxy for \(K\) 
    aligns with Zenil's Algorithmic Complexity but considering compression sensitivity, capturing 
    rule complexity changes.
    \item \textbf{No Partitions}: Edge perturbations approximate MIP dynamically, avoiding exponential costs.
    \item \textbf{Rule Refinement}: Adds local sensitivity, making \(\phi_K\) more robust.
    \item \textbf{Scalability}: Linear in edges, not exponential in nodes, unlike IIT.
\end{itemize}

\textbf{Formulas Recap}:
\begin{itemize}
    \item Sensitivity per edge: \(\text{Sensitivity}_i = \Delta K_i \cdot \log_2(\Delta A_i + 1)\)
    \item Initial \(\phi_K\): \(\phi_K = \frac{1}{E} \sum_{i=1}^E \text{Sensitivity}_i\)
    \item Refined \(\phi_K\): \(\phi_K = \left( \frac{1}{E} \sum_{i=1}^E \text{Sensitivity}_i \right) + w \cdot \Delta K_{\text{rule,avg}}\), \(w = 1\)
\end{itemize}

\textbf{Overview of implementation}

Our \(\phi_K\) offers a computationally efficient alternative to IIT's \(\phi\), capturing integration 
through attractor dynamics and algorithmic complexity. 

When we visualize the behaviour of a system (or subsystem like an isolated node), and 
take into account its implementation, from the point of view of optimization of computational 
resources, running rules to generate the behaviour of the whole is still a challenge 
because it is an expensive process in terms of time and memory. Hence for large systems, 
analysis based on exhaustive reviews of such behaviour could eventually become intractable.

In order to overcome this limitation, $\phi_{K}$ attempted to find rules that not 
only give an account of the computability capabilities of a system, but also 
describe its own behaviour. In other words, we wanted to know about possibilities 
for finding "shortcuts to express the behaviour'' of a whole system.

One other obvious limitation inherited from computability and algorithmic complexity 
is that of the semi-computability of the process of trying to find simple 
representations of behaviour. However, we are not required to find the shortest 
(simplest) one but simply a set of possible short (simple) ones, which would be 
an indication of the kind of system we are dealing with. While one can find shorter 
descriptions using popular lossless compression algorithms, the more powerful 
the algorithms to find shortcuts and fractal descriptions, the faster the 
computation and the more telling the results,
something that is to be expected for a relationship between the way in which
integrated information is estimated, on the one hand, and algorithmic complexity.

\section{Demonstration of Integration Measurement via \(\phi_K\)}

To justify the effectiveness of the \(\phi_K\) approach, we demonstrate its application 
on a specific 5-node network, defined by its adjacency matrix and dynamics. 
This step-by-step example illustrates how integration is measured through perturbation, a
ttractor dynamics, and algorithmic complexity, showcasing the computational efficiency 
and interpretability of our method compared to traditional approaches like IIT.

\subsection{Step 0: Network Setup}
We consider a 5-node network with the following parameters:
\begin{itemize}
    \item \textbf{Adjacency Matrix}: A complete graph (with self-loops removed), represented as:
    \[
    \text{adjMat} = \begin{pmatrix}
        0 & 1 & 1 & 1 & 1 \\
        1 & 0 & 1 & 1 & 1 \\
        1 & 1 & 0 & 1 & 1 \\
        1 & 1 & 1 & 0 & 1 \\
        1 & 1 & 1 & 1 & 0
    \end{pmatrix}
    \]
    This matrix indicates 10 edges (since it is symmetric and the diagonal is 0), reflecting a highly connected network expected to exhibit significant integration.
    \item \textbf{Dynamics}: Each node is assigned a Boolean rule:
    \[
    \text{dyn} = \{\text{``OR''}, \text{``AND''}, \text{``OR''}, \text{``AND''}, \text{``OR''}\}
    \]
    These rules govern how nodes update their states based on inputs from connected nodes, introducing variability in the network's behavior.
\end{itemize}

\subsection{Step 1: Compute Baseline Attractors}
First, we compute the network's baseline attractors—the stable states or cycles under the given dynamics. The attractor map is:
{\small
\begin{align*}
\text{Baseline Attractors} = \langle| 
&\{0,0,0,0,0\} \to \{0\}, \{0,0,1,0,1\} \to \{1\}, \{1,0,0,0,1\} \to \{4\}, \\
&\{1,0,1,0,0\} \to \{16\}, \{1,0,1,0,1\} \to \{2,3,5,6,7,8,9,10,11,12,13,14, \\
&15,17,18,19,20,21,22,24,25,26,27,28,30\}, \{1,0,1,1,1\} \to \{23\}, \\
&\{1,1,1,0,1\} \to \{29\}, \{1,1,1,1,1\} \to \{31\} |\rangle
\end{align*}
}
Here, each key (e.g., \(\{0,0,0,0,0\}\)) represents a binary state (converted from decimal positions in the repertoire), and the values (e.g., \(\{0\}\)) indicate positions in the full \(2^5 = 32\) state space where this state appears as a stable attractor. Notably, the state \(\{1,0,1,0,1\}\) dominates, appearing at 21 positions, suggesting a strong attractor basin.

We measure:
\begin{itemize}
    \item \textbf{Baseline Complexity (\(K_0\))}: \(K_0 = 3160\) bytes, computed using \(\text{ByteCount}\) in Mathematica, serving as a proxy for Kolmogorov complexity \cite{zenil2015causality}.
    \item \textbf{Baseline Attractor Count (\(A_0\))}: \(A_0 = 8\), the number of unique attractors, reflecting the network's dynamic diversity.
\end{itemize}

\subsubsection{Step 2: Count Edges}
The network has 10 edges (counted as 1s in the upper triangular part of the adjacency matrix, excluding the diagonal):
\[
\text{Number of Edges} = 10
\]
This determines the number of perturbations we will perform.

\subsubsection{Step 3: Perturb Edges and Compute Sensitivities}
We perturb each edge by setting its connectivity to 0 and recompute the attractors, measuring the sensitivity to each perturbation. The process is illustrated for select edges:

\begin{itemize}
    \item \textbf{Perturbation of Edge (1,2)}:
    \[
    \text{Perturbed Adjacency Matrix} = \begin{pmatrix}
        0 & 0 & 1 & 1 & 1 \\
        1 & 0 & 1 & 1 & 1 \\
        1 & 1 & 0 & 1 & 1 \\
        1 & 1 & 1 & 0 & 1 \\
        1 & 1 & 1 & 1 & 0
    \end{pmatrix}
    \]
    {\small
        \begin{align*}
        \text{Perturbed Attractors} = \langle| 
        &\{0,0,0,0,0\} \to \{0\}, \{0,0,1,0,1\} \to \{1,2,3\}, \{1,0,0,0,1\} \to \{4\}, \\
        &\{1,0,1,0,0\} \to \{16\}, \{1,0,1,0,1\} \to \{5,6,7,8,9,10,11,12,13,14,15, \\
        &17,18,19,20,21,22,24,25,26,27,28,30\}, \{1,0,1,1,1\} \to \{23\}, \\
        &\{1,1,1,0,1\} \to \{29\}, \{1,1,1,1,1\} \to \{31\} |\rangle
        \end{align*}
        }
    \[
    K_1 = 3176 \text{ bytes}, \quad A_1 = 8
    \]
    \[
    \Delta K = |3160 - 3176| = 16, \quad \Delta A = |8 - 8| = 0
    \]
    \[
    \text{Sensitivity}_{(1,2)} = \Delta K \cdot \log_2(\Delta A + 1) = 16 \cdot \log_2(0 + 1) = 16 \cdot 0 = 0
    \]
    The zero sensitivity indicates that removing this edge does not significantly alter the network's dynamics, as both the number and complexity of attractors remain stable.

    \item \textbf{Perturbation of Edge (2,4)} (a more impactful example):
    \[
    \text{Perturbed Adjacency Matrix} = \begin{pmatrix}
        0 & 1 & 1 & 1 & 1 \\
        1 & 0 & 1 & 0 & 1 \\
        1 & 1 & 0 & 1 & 1 \\
        1 & 1 & 1 & 0 & 1 \\
        1 & 1 & 1 & 1 & 0
    \end{pmatrix}
    \]
    {\small
        \begin{align*}
        \text{Perturbed Attractors} = \langle| 
        &\{0,0,0,0,0\} \to \{0\}, \{0,0,1,0,1\} \to \{1\}, \{1,0,0,0,1\} \to \{4\}, \\
        &\{1,0,1,0,0\} \to \{16\}, \{1,0,1,0,1\} \to \{2,3,5,6,7,8,9,10,11,12,13,14, \\
        &15,17,18,19,20,22,24,25,26,27,28,30\}, \{1,1,1,0,1\} \to \{21,29\}, \\
        &\{1,1,1,1,1\} \to \{23,31\} |\rangle
        \end{align*}
        }
    \[
    K_1 = 2808 \text{ bytes}, \quad A_1 = 7
    \]
    \[
    \Delta K = |3160 - 2808| = 352, \quad \Delta A = |8 - 7| = 1
    \]
    \[
    \text{Sensitivity}_{(2,4)} = \Delta K \cdot \log_2(\Delta A + 1) = 352 \cdot \log_2(1 + 1) = 352 \cdot 1 = 352
    \]
    Here, the sensitivity is significant, reflecting a notable change in the attractor map: the number of attractors decreases from 8 to 7, and the complexity drops by 352 bytes, indicating that edge (2,4) plays a critical role in the network's integration.

    \item \textbf{Remaining Edges}: Similar perturbations are performed for all 10 edges. Most edges (e.g., (1,3), (1,4), (1,5), etc.) yield \(\text{Sensitivity}_i = 0\), as \(\Delta A = 0\), meaning the attractor count remains stable. However, edge (4,2) also yields \(\text{Sensitivity}_{(4,2)} = 352\), mirroring the impact of edge (2,4) due to the symmetry of the undirected graph.
\end{itemize}

The sensitivities across all edges are:
\[
\text{Sensitivities} = \{0, 0, 0, 0, 0, 0, 352, 0, 0, 0, 0, 0, 352, 0, 0, 0, 0, 0, 0, 0\}
\]
Filtering out zeros (as per our methodology), we retain:
\[
\text{Sensitivities} = \{352, 352\}
\]

\subsubsection{Step 4: Compute Initial \(\phi_K\)}
The initial \(\phi_K\) is the mean of the non-zero sensitivities:
\[
\phi_K = \frac{1}{|\text{Sensitivities}|} \sum \text{Sensitivity}_i = \frac{352 + 352}{2} = 352
\]
This value indicates that, on average, the network’s dynamics are moderately sensitive to perturbations, with only two edges significantly affecting the attractor map.

\subsubsection{Step 5: Compute Rules and Refine \(\phi_K\)}
To refine \(\phi_K\), we compute the rule-based sensitivity:
\begin{itemize}
    \item \textbf{Baseline Rules}: For each attractor, we generate a rule encoding its positions (e.g., for \(\{0,0,0,0,0\} \to \{0\}\), the rule is \(\langle| \text{``DecimalRepertoire''} \to \{0\}, \text{``Sumandos''} \to \{0\} |\rangle\)).
    \[
    K_{\text{rule,0}} = 585 \text{ bytes (average across 8 rules)}
    \]
    \item \textbf{Perturbed Rules}: Perturbing edge (1,2), we recompute rules for the original attractors:
    \[
    K_{\text{rule,1}} = 587 \text{ bytes}
    \]
    \[
    \Delta K_{\text{rule}} = |585 - 587| = 2
    \]
\end{itemize}
This small \(\Delta K_{\text{rule}}\) suggests that the rules are relatively stable, but it adds a local sensitivity component to our measure.

The refined \(\phi_K\) with \(w = 1\) is:
\[
\phi_K = 352 + 1 \cdot 2 = 354
\]

\subsubsection{Interpretation}
The final \(\phi_K = 354\) reflects the network’s integration, combining global (attractor map) and local (rule) sensitivities. Notably:
\begin{itemize}
    \item Most perturbations (e.g., edge (1,2)) result in \(\text{Sensitivity}_i = 0\), indicating that the network’s dynamics are robust to many single-edge removals, a sign of high integration where the system remains cohesive.
    \item Edges (2,4) and (4,2) yield high sensitivity (352), showing that these connections are critical to maintaining the attractor structure. Removing them reduces the number of attractors, collapsing some states into others, which aligns with our expectation for a highly integrated network: perturbations to key edges have a significant impact.
    \item The small \(\Delta K_{\text{rule}} = 2\) adds nuance, capturing subtle changes in the rules generating attractors, enhancing the robustness of our measure.
\end{itemize}

This demonstration validates our \(\phi_K\) approach, showing how it captures integration through dynamic sensitivity in a computationally efficient manner (\(O(E) = 10\) perturbations vs. IIT’s \(2^{5-1} = 16\) bipartitions). The result (\(\phi_K = 354\)) contrasts with IIT’s \(\phi = 0\) for this network, highlighting our method’s focus on dynamic sensitivity over irreducible information, making it more suitable for large-scale systems.


\subsection{Conclusion}
Our \(\phi_K\) approach provides a novel, scalable method to measure integration in complex networks, leveraging attractor dynamics and algorithmic complexity. The step-by-step demonstration on a 5-node network illustrates its practical application, capturing integration through sensitivity to perturbations in a way that aligns with the network’s intrinsic dynamics. By avoiding IIT’s exponential computational cost, \(\phi_K\) paves the way for studying integration in larger, more complex systems, with potential applications in social, economic, and governance contexts.

% % Add bibliography references if needed
% \begin{thebibliography}{9}
% \bibitem{mayner2018pyphi}
% Mayner O, et al. PyPhi: A toolbox for integrated information theory. \emph{PLoS Computational Biology}. 2018;14(7):e1006343.
% \end{thebibliography}

% \end{document}   % ~20 páginas - Presentar los resultados tal cual son, y analizarlos.
\chapter{Conclusions}
\textcolor{black}{Here we have sketched connections and developed first approaches towards a 
calculus to $\phi$ metrics based on algorithmic perturbation analysis, which in turn has a 
solid mathematical foundation that must be further studied. Our computational approach 
targeted what is referred to as the IIT 3.0, defined as a calculus of probability distributions. 
Instead of considering distances between statistical distributions, we formulated 
the problem as a distance in an algorithmic complexity space, properly approximated, 
in response to perturbations of the system and introduced a meta-test whose answers may 
provide a guidance on the algorithmic complexity and integrated information of the system. 
More exploration of the theoretical and practical connections between these theories are 
still needed.}

\textcolor{black}{Interestingly, such a perturbation programmability test--initially 
inspired by the Turing test (establishing another interesting connection between these n
ew theories of consciousness and past ones)-- as applied to physical systems, is a working 
strategy to find explanations for the behaviour of systems. It remains for future work to 
make conceptual and computational connections to what Oizumi and Tononi et al. called the MIP 
(Minimum Information Partition)~\cite{oizumi2014fromthe} of a system. Having this first 
version of $\phi_{K}$, we conjecture that MIP definitions also obey and are connected to 
algorithmic complexity in about the same way, as they should remain based on rules of an 
algorithmic nature. Thus, the next step is to go further in the application of the test 
introduced in this paper to discover simple rules that would help to find MIP in a more 
natural and a faster way. Another possible direction is to systematize the finding of 
these simple rules and apply more powerful methods to enable computation of larger systems. 
However, here we have merely established the first principles and the directions that can be 
explored following these ideas.}

Finally, we think that these ideas about self-explanatory systems capable of providing 
answers to questions about their own behaviour can help in devising techniques to make 
other methods, in areas such as machine and deep learning, explain their own, often obscure, 
behaviour.

\textcolor{black}{While the primary contributions of this work are theoretical and computational, they resonate with deeper philosophical themes. The development of $\phi_K$ intersects with questions traditionally explored in cognitive philosophy, such as the nature of emergence, the unity of conscious experience, and the explanatory power of algorithmic representations. By modeling the way systems integrate information through minimal algorithmic descriptions and perturbative self-querying, this thesis gestures toward a mechanistic account of how subjectivity and self-coherence may arise in both biological and artificial systems. Future interdisciplinary work could further investigate whether these algorithmic frameworks serve not only as tools for analysis but as candidates for foundational explanatory models of consciousness.}

            % ~5 páginas - Resumir lo que se hizo y lo que no y comentar trabajos futuros sobre el tema

%%%%%%%%%%%%%%%%%%%%%%%%%%%%%%%%%%%%%%%%%%%%%%%%%%%%%
%                   APÉNDICES                       %
%%%%%%%%%%%%%%%%%%%%%%%%%%%%%%%%%%%%%%%%%%%%%%%%%%%%%
\appendix
% this file is called up by thesis.tex
% content in this file will be fed into the main document
\chapter{Schemas of Information}
% top level followed by section, subsection

In order to explain the advantages of the generalization of information in 
the form of schemas computed by simple rules, Table  \ref{table:9NodeSchemataConditions} is introduced.
In this Table are shown all possible cases where the pattern \{8,9\}=\{1,1\}
than can be found in the whole output repertoire of the 9-Node system introduced
in Figure \ref{fig:9NodeNet} in the main text.

\begin{table}[H]
	\centering
	\small
	\begin{tabular}{p{0.5\textwidth}} 
	\{\textbf{1},0,\textbf{1},0,\textbf{1,0,1},0,0\}->\{1,1,0,1,0,0,1,\textbf{1,1}\} \\
	\{\textbf{1},1,\textbf{1},0,\textbf{1,0,1},0,0\}->\{1,1,0,1,0,0,1,\textbf{1,1}\} \\
	\{\textbf{1},0,\textbf{1},1,\textbf{1,0,1},0,0\}->\{1,1,0,1,0,0,1,\textbf{1,1}\} \\
	\{\textbf{1},1,\textbf{1},1,\textbf{1,0,1},0,0\}->\{1,1,0,1,0,0,1,\textbf{1,1}\} \\
	\{\textbf{1},0,\textbf{1},0,\textbf{1,1,1},0,0\}->\{1,1,0,1,0,0,1,\textbf{1,1}\} \\
	\{\textbf{1},1,\textbf{1},0,\textbf{1,1,1},0,0\}->\{1,1,0,1,0,0,1,\textbf{1,1}\} \\
	\{\textbf{1},0,\textbf{1},1,\textbf{1,1,1},0,0\}->\{1,1,0,1,0,0,1,\textbf{1,1}\} \\
	\{\textbf{1},1,\textbf{1},1,\textbf{1,1,1},0,0\}->\{1,1,0,1,0,0,1,\textbf{1,1}\} \\
	\{\textbf{1},0,\textbf{1},0,\textbf{1,0,1},1,0\}->\{1,1,0,1,0,0,1,\textbf{1,1}\} \\
	\{\textbf{1},1,\textbf{1},0,\textbf{1,0,1},1,0\}->\{1,1,0,1,0,0,1,\textbf{1,1}\} \\
	\{\textbf{1},0,\textbf{1},1,\textbf{1,0,1},1,0\}->\{1,1,0,1,0,0,1,\textbf{1,1}\} \\
	\{\textbf{1},1,\textbf{1},1,\textbf{1,0,1},1,0\}->\{1,1,0,1,0,0,1,\textbf{1,1}\} \\
	\{\textbf{1},0,\textbf{1},0,\textbf{1,1,1},1,0\}->\{1,1,0,1,0,0,1,\textbf{1,1}\} \\
	\{\textbf{1},1,\textbf{1},0,\textbf{1,1,1},1,0\}->\{1,1,0,1,0,0,1,\textbf{1,1}\} \\
	\{\textbf{1},0,\textbf{1},1,\textbf{1,1,1},1,0\}->\{1,1,0,1,0,0,1,\textbf{1,1}\} \\
	\{\textbf{1},1,\textbf{1},1,\textbf{1,1,1},1,0\}->\{1,1,0,1,0,0,1,\textbf{1,1}\} \\
	\{\textbf{1},0,\textbf{1},0,\textbf{1,0,1},0,1\}->\{1,1,0,1,0,0,1,\textbf{1,1}\} \\
	\{\textbf{1},1,\textbf{1},0,\textbf{1,0,1},0,1\}->\{1,1,0,1,0,0,1,\textbf{1,1}\} \\
	\{\textbf{1},0,\textbf{1},1,\textbf{1,0,1},0,1\}->\{1,1,0,1,0,0,1,\textbf{1,1}\} \\
	\{\textbf{1},1,\textbf{1},1,\textbf{1,0,1},0,1\}->\{1,1,0,1,0,0,1,\textbf{1,1}\} \\
	\{\textbf{1},0,\textbf{1},0,\textbf{1,1,1},0,1\}->\{1,1,0,1,0,0,1,\textbf{1,1}\} \\
	\{\textbf{1},1,\textbf{1},0,\textbf{1,1,1},0,1\}->\{1,1,0,1,0,0,1,\textbf{1,1}\} \\
	\{\textbf{1},0,\textbf{1},1,\textbf{1,1,1},0,1\}->\{1,1,0,1,0,0,1,\textbf{1,1}\} \\
	\{\textbf{1},1,\textbf{1},1,\textbf{1,1,1},0,1\}->\{1,1,0,1,0,0,1,\textbf{1,1}\} \\
	\{\textbf{1},0,\textbf{1},0,\textbf{1,0,1},1,1\}->\{1,1,0,1,0,0,1,\textbf{1,1}\} \\
	\{\textbf{1},1,\textbf{1},0,\textbf{1,0,1},1,1\}->\{1,1,0,1,0,0,1,\textbf{1,1}\} \\
	\{\textbf{1},0,\textbf{1},1,\textbf{1,0,1},1,1\}->\{1,1,0,1,0,0,1,\textbf{1,1}\} \\
	\{\textbf{1},1,\textbf{1},1,\textbf{1,0,1},1,1\}->\{1,1,0,1,0,1,1,\textbf{1,1}\} \\
	\{\textbf{1},0,\textbf{1},0,\textbf{1,1,1},1,1\}->\{1,1,0,1,0,0,1,\textbf{1,1}\} \\
	\{\textbf{1},1,\textbf{1},0,\textbf{1,1,1},1,1\}->\{1,1,0,0,0,0,1,\textbf{1,1}\} \\
	\{\textbf{1},0,\textbf{1},1,\textbf{1,1,1},1,1\}->\{1,0,0,1,0,0,1,\textbf{1,1}\} \\
	\{\textbf{1},1,\textbf{1},1,\textbf{1,1,1},1,1\}->\{1,0,1,0,1,1,1,\textbf{1,1}\} \\
	Time: \{0.166334\} ,	Memory: \{217920\} \\
	\end{tabular}
	\caption{Reperoire of inputs and outputs where the condition \{8,9\}=\{1,1\} fullfill
	for the system shown in Figure \ref{fig:9NodeNet} of the main text}
	\label{table:9NodeSchemataConditions}
	\end{table}


On the right side of the set contained in Table \ref{table:9NodeSchemataConditions}, 
outputs where \{8,9\}=\{1,1\} are highlighted in bold. On the left side the 9-length are inputs
that yield to outputs containing the desired pattern. On this left
side, in bold, are the corresponding inputs that are particularly
responsible for causing the desired pattern, that is, all possible patterns
for the inputs \{1,3,5,6,7\}

In order to obtain the results in Table \ref{table:9NodeSchemataConditions} 
using the naive (brute force) approach show in Table \ref{Table:Naivie9NodeNet} of the 
main text,
it was necessary to define the whole set of all $2^{9}$ possible
inputs and compute the whole set of outputs; then an exhaustive search
for \{8,9\}=\{1,1\} was carried out. Notice that time and memory used
are at least 10 times greater than those used in the $\phi_{K}$ approach.
These results are shown in the last two rows in Table \ref{table:9NodeSchemataConditions}.


% % -------------------------------
% % THIS IS MY ALGORITHM 0: Computes information integration information
% % --------------------------------
% \IncMargin{1em}
% \begin{algorithm}[!htb]
% 	\SetKwData{Pud}{pud}
% 	\SetKwData{Fud}{fud}
% 	\SetKwData{Fbupd}{fbupd}
% 	\SetKwData{Distros}{distros}
% 	\SetKwData{ConceptualSpace}{conceptualSpace}
% 	\SetKwData{Nodes}{nodes}
% 	\SetKwData{MechaSet}{mechaSet}
% 	\SetKwData{OneConcept}{OneConcept}
% 	\SetKwData{BipartitionsSet}{bipartitionsSet}
% 	\SetKwData{IntegratedInformationValue}{integratedInformationValue}
% 	\SetKwData{Aux}{aux}
% 	\SetKwData{UPPD}{UPPD}
% 	\SetKwData{UFPD}{UFPD}
	
	
% 	\SetKwFunction{computeDistros}{computeDistros}
% 	\SetKwFunction{Subsets}{Subsets}
% 	\SetKwFunction{ComputeInputBitProbabilityDistro}{ComputeInputBitProbabilityDistro}
% 	\SetKwFunction{ComputeOutputBitProbabilityDistro}{ComputeOutputBitProbabilityDistro}
% 	\SetKwFunction{GetNodes}{GetNodes}
% 	\SetKwFunction{ComputesPastProbabilityDistribution}{ComputesPastProbabilityDistribution}
% 	\SetKwFunction{ComputesFutureProbabilityDistribution}{ComputesFutureProbabilityDistribution}
% 	\SetKwFunction{ComputeConceptOfAMechanism}{ComputeConceptOfAMechanism}
% 	\SetKwFunction{Append}{Append}
% 	\SetKwFunction{ComputesConceptualSpace}{ComputesConceptualSpace}
% 	\SetKwFunction{Bipartitions}{Bipartitions}
% 	\SetKwFunction{EMD}{EMD}
% 	\SetKwFunction{GetNodes}{GetNodes}
	
	
	
% 	\SetKwInOut{Input}{input}
% 	\SetKwInOut{Output}{output}
	
% 	\Input{AdjacencyMatrix,Dynamic,CurrentState}
% 	\Output{Information Integration Value}
% 	\BlankLine
% 	\Nodes $\leftarrow$ \GetNodes{AdjacencyMatrix}\;
% 	\tcp*[h]{UPPD: Unrestricted Past Probability Distribution}\;
% 	\tcp*[h]{UFPD: Unrestricted Future Probability Distribution}\;
% 	\UPPD $\leftarrow$  \ComputesPastProbabilityDistribution{\Nodes,CurrentState,$\emptyset$,Dynamic,am}\;
% 	\UFPD $\leftarrow$  \ComputesFutureProbabilityDistribution{\Nodes,CurrentState,$\emptyset$,Dynamic,am}\;
% 	\tcp*[h]{am=AdjacencyMatrix; cs=CurrentState}\;
% 	\ConceptualSpace$\leftarrow$ \ComputesConceptualSpace{am,Dynamic,cs,\UPPD,\UFPD}\;
	
% 	\IntegratedInformationValue$\leftarrow$ 0\;
% 	\BipartitionsSet$\leftarrow$ \Bipartitions{\ConceptualSpace}\;
	

% 	\ForEach{bipartition $b_i \in \BipartitionsSet$}{%
% 		\Aux$\leftarrow$\EMD{$b_i,\ConceptualSpace$}\;
% 		\uIf{ \Aux > \IntegratedInformationValue}{%
% 			\IntegratedInformationValue$\leftarrow$ \Aux\;
% 		}
% 	}
	
	
	
% 	\caption{computeIntegratedInformation}\label{algo_computeIntegratedInformation}
% \end{algorithm}
% \DecMargin{1em}



% % -------------------------------
% % THIS IS MY ALGORITHM 2: GENERAL COMPUTATION OF CONCEPTUAL STRUCTURE
% % --------------------------------
% \IncMargin{1em}
% \begin{algorithm}[!htb]
% 	\SetKwData{Pud}{pud}
% 	\SetKwData{Fud}{fud}
% 	\SetKwData{Fbupd}{fbupd}
% 	\SetKwData{Distros}{distros}
% 	\SetKwData{ConceptualSpace}{conceptualSpace}
% 	\SetKwData{Nodes}{nodes}
% 	\SetKwData{MechaSet}{mechaSet}
% 	\SetKwData{OneConcept}{OneConcept}
	
	
% 	\SetKwFunction{computeDistros}{computeDistros}
% 	\SetKwFunction{Subsets}{Subsets}
% 	\SetKwFunction{ComputeInputBitProbabilityDistro}{ComputeInputBitProbabilityDistro}
% 	\SetKwFunction{ComputeOutputBitProbabilityDistro}{ComputeOutputBitProbabilityDistro}
% 	\SetKwFunction{GetNodes}{GetNodes}
% 	\SetKwFunction{ComputesPastProbabilityDistribution}{ComputesPastProbabilityDistribution}
% 	\SetKwFunction{ComputesFutureProbabilityDistribution}{ComputesFutureProbabilityDistribution}
% 	\SetKwFunction{ComputeConceptOfAMechanism}{ComputeConceptOfAMechanism}
% 	\SetKwFunction{Append}{Append}

	
	
% 	\SetKwInOut{Input}{input}
% 	\SetKwInOut{Output}{output}
	
% 	\Input{AdjacencyMatrix,Dynamic,CurrentState,UPPD,UFPD}
% 	\Output{conceptualStructure}
% 	\BlankLine
% 	\tcp*[h]{UPPD: Unrestricted Past Probability Distribution}\;
% 	\tcp*[h]{UFPD: Unrestricted Future Probability Distribution}\;
% 	\Nodes$\leftarrow$ \GetNodes{AdjacencyMatrix}\;
% 	\MechaSet$\leftarrow$\Subsets{\Nodes}\;
	
% 	\ForEach{mechanism $mecha_i \in \MechaSet$}{%
% 		\OneConcept$\leftarrow$\ComputeConceptOfAMechanism{$mecha_i,\Nodes,CurrentState,UPPD,UFPD$}\;
% 		\Append{\ConceptualSpace,\OneConcept}
% 	}
	

	
% 	\caption{computeConceptualSpace}\label{algo_computeConceptualStructure}
% \end{algorithm}
% \DecMargin{1em}


% % -------------------------------
% % THIS IS MY ALGORITHM 3: COMPUTE CONCEPT OF A MECHANISM
% % --------------------------------
% \IncMargin{1em}
% \begin{algorithm}[!htb]
	
% 	% VARIABLES SECTION
% 	% -------------------------
% 	\SetKwData{PurviewsSet}{purviewsSet}
% 	\SetKwData{APurview}{aPurview}
% 	\SetKwData{APurviewMIP}{aPurviewMIP}
% 	\SetKwData{Distros}{distros}
% 	\SetKwData{Cs}{cs}
% 	\SetKwData{Connected}{connected}
% 	\SetKwData{MIP}{MIP}
% 	\SetKwData{SmallAlpha}{smallAlpha}
% 	\SetKwData{ConceptualInfo}{ConceptualInfo}
% 	\SetKwData{PastDistribution}{pastDistribution}
% 	\SetKwData{FutureDistribution}{futureDistribution}
	
% 	% FUNCTIONS SECTION
% 	% -----------------------
% 	\SetKwFunction{Subsets}{Subsets}
% 	\SetKwFunction{Length}{Length}
% 	\SetKwFunction{Extract}{Extract}
% 	\SetKwFunction{Part}{Part}
% 	\SetKwFunction{FullyConnectedQ}{FullyConnectedQ}
% 	\SetKwFunction{ComputesMIP}{ComputesMIP}
% 	\SetKwFunction{ComputesDistros}{ComputesDistros}
% 	\SetKwFunction{EMD}{EMD}
% 	\SetKwFunction{ComputesPastProbabilityDistribution}{ComputesPastProbabilityDistribution}
% 	\SetKwFunction{ComputesFutureProbabilityDistribution}{ComputesFutureProbabilityDistribution}
	
% 	\SetKwInOut{Input}{input}
% 	\SetKwInOut{Output}{output}
	
% 	% INPUTS-OUTPUTS SECTION
% 	% ------------------------------
% 	\Input{mechanism,nodesForPurviews,currentState,pastDistro,futDistro,Dynamic, AdjacencyMatrix}
% 	\Output{concept for current mechanism}
% 	\BlankLine
	
% 	% CODE SECTION
% 	% -----------------------------
% 	\tcp*[h]{nodes where all purviews will be taken from}
	
% 	\PurviewsSet$\leftarrow$\Subsets{$nodesForPurviews$}\;
% 	\BlankLine
	
% 	\For{$j\leftarrow 1$ \KwTo \Length{$PurviewsSet$}}
% 	{\label{forins}
		
% 		\APurview$\leftarrow$ \Part{$j,PurviewsSet$}\;
% 		\Connected$\leftarrow$ \FullyConnectedQ{$mechanism,APurview$}\;
% 		\lIf{\Connected}{
			
% 			\tcp*[h]{MIP: Maximal Information Partition}
			
% 			\SmallAlpha$\leftarrow$ \ComputesMIP{$mechanism,APurview$}
% 		}
% 	}
	
% 	\tcp*[h]{APurviewMIP: Purview responsable to cause MIP for current mechanism}\;
	
% 	\APurviewMIP$\leftarrow$ \SmallAlpha{$"PurviewMIP"$}\;
% 	\tcp*[h]{Following sum is formalized in Figure 4, In Text S2 from Oizumi(2014)}\;
% 	\tcp*[h]{cs=CurrentState; am = AdjacencyMatrix}\;
% 	\PastDistribution$\leftarrow$ \ComputesPastProbabilityDistribution{$mechanism,cs,\APurviewMIP,\newline
% 	Dynamic,am$}\;
% 	\FutureDistribution$\leftarrow$ \ComputesFutureProbabilityDistribution{$mechanism,cs,\APurviewMIP,\newline
% 	Dynamic,am$}\;
% 	\ConceptualInfo$\leftarrow$ \EMD{$pastDistro,\PastDistribution$}+\EMD{$futDistro,\FutureDistribution$}\;
	

	
% 	\caption{computeConceptOfAMechanism}\label{algo_computeConceptOfMecha}
% \end{algorithm}
% \DecMargin{1em}



% % -------------------------------
% % THIS IS MY ALGORITHM 3: COMPUTE MIP FOR A MECHA AND A PURVIEW
% % --------------------------------
% \IncMargin{1em}
% \begin{algorithm}[!htb]
	
% 	% VARIABLES SECTION
% 	% -------------------------
% 	\SetKwData{PurviewsSet}{purviewsSet}	
% 	\SetKwData{APurview}{aPurview}
% 	\SetKwData{APurviewMIP}{aPurviewMIP}
% 	\SetKwData{Distros}{distros}
% 	\SetKwData{Cs}{cs}
% 	\SetKwData{Connected}{connected}
% 	\SetKwData{MIP}{MIP}
% 	\SetKwData{SmallAlpha}{smallAlpha}
% 	\SetKwData{ConceptualInfo}{ConceptualInfo}
% 	\SetKwData{MechaChildren}{mechaChildren}
% 	\SetKwData{PurviewChildren}{PurviewChildren}
% 	\SetKwData{Ci}{ci}
% 	\SetKwData{Ei}{ei}
% 	\SetKwData{Cei}{cei}
% 	\SetKwData{PastMIP}{pastMIP}
% 	\SetKwData{FutMIP}{futMIP}
% 	\SetKwData{PastDistribution}{pastDistribution}
% 	\SetKwData{FutureDistribution}{futureDistribution}
	
% 	% FUNCTIONS SECTION
% 	% -----------------------
% 	\SetKwFunction{Subsets}{Subsets}
% 	\SetKwFunction{Length}{Length}
% 	\SetKwFunction{Extract}{Extract}
% 	\SetKwFunction{Part}{Part}
% 	\SetKwFunction{FullyConnectedQ}{FullyConnectedQ}
% 	\SetKwFunction{ComputesMIP}{ComputesMIP}
% 	\SetKwFunction{ComputesDistros}{ComputesDistros}
% 	\SetKwFunction{EMD}{EMD}
% 	\SetKwFunction{ComputesCEI}{ComputesCEI}
% 	\SetKwFunction{ComputesPastProbabilityDistribution}{ComputesPastProbabilityDistribution}
% 	\SetKwFunction{ComputesFutureProbabilityDistribution}{ComputesFutureProbabilityDistribution}
	
	
% 	\SetKwInOut{Input}{input}
% 	\SetKwInOut{Output}{output}
	
% 	% INPUTS-OUTPUTS SECTION
% 	% ------------------------------
% 	\Input{mechanism,purview}
% 	\Output{MIP structure}
% 	\BlankLine
	
% 	% CODE SECTION
% 	% -----------------------------
% %	\tcp*[h]{To find a MIP for a concept involves search between all possible }
% %	\tcp*[h]{combinations of of all possible partitions of a mechanism and a }
% %	\tcp*[h]{purview, where MIP is such combination with the maximal }
% %	\tcp*[h]{information possible value}
	
	
% 	\MechaChildren$\leftarrow$ \Subsets{mechanism}\;
% 	\PurviewChildren$\leftarrow$ \Subsets{purview}\;
% %	\tcp*[h]{As we want to find the minimum value of causal information (ci) }
% %	\tcp*[h]{and effect information (ei), any big value works here, since every }
% %	\tcp*[h]{time a lower value of ci and ei be found, these are updated with }
% %	\tcp*[h]{the new lower value found}
	
% 	\Ci $\leftarrow$ 10000\;
% 	\Ei $\leftarrow$ 10000\;

% 	\ForEach{mecha  $m_i \in \MechaChildren$}{%
% 		\ForEach{purview $p_i \in \PurviewChildren$}{%
% 			\tcp*[h]{cs=CurrentState, am=AdjacencyMatrix}\;
% 			\PastDistribution $\leftarrow$ \ComputesPastProbabilityDistribution{$m_i$,cs,$p_i$,Dynamic,am}\;
% 			\FutureDistribution $\leftarrow$ \ComputesFutureProbabilityDistribution{$m_i$,cs,$p_i$,Dynamic,am}\;
% 			\Cei $\leftarrow$ \ComputesCEI{mecha,purview,\PastDistribution,\FutureDistribution}\;
			
% 			\If{\Cei("ci") < \Ci}{%
% 				\Ci $\leftarrow$ \Cei("ci")\;
% 				\PastMIP $\leftarrow$ (mecha,purview)\;
% 			}
		
% 			\If{\Cei("ei") < \Ei}{%
% 				\Ei $\leftarrow$ \Cei("ei")\;
% 				\FutMIP $\leftarrow$ (mecha,purview)\;
% 			}

% 		}
% 	}
	
% \caption{ComputesMIP}\label{algo_computesMIP}
% \end{algorithm}
% \DecMargin{1em}



% % -------------------------------
% % THIS IS MY ALGORITHM 4: computeCEI4Partition
% % --------------------------------
% \IncMargin{1em}
% \begin{algorithm}[!htb]
	
% 	% VARIABLES SECTION
% 	% -------------------------
% 	\SetKwData{PurviewsSet}{purviewsSet}	
% 	\SetKwData{APurview}{aPurview}
% 	\SetKwData{APurviewMIP}{aPurviewMIP}
% 	\SetKwData{Distros}{distros}
% 	\SetKwData{Cs}{cs}
% 	\SetKwData{Connected}{connected}
% 	\SetKwData{MIP}{MIP}
% 	\SetKwData{SmallAlpha}{smallAlpha}
% 	\SetKwData{ConceptualInfo}{ConceptualInfo}
% 	\SetKwData{MechaChildren}{mechaChildren}
% 	\SetKwData{PurviewChildren}{PurviewChildren}
% 	\SetKwData{Ci}{ci}
% 	\SetKwData{Ei}{ei}
% 	\SetKwData{Cei}{cei}
% 	\SetKwData{PastMIP}{pastMIP}
% 	\SetKwData{FutMIP}{futMIP}
% 	\SetKwData{MechaComplement}{mechaComplement}
% 	\SetKwData{PurviewComplement}{purviewComplement}
% 	\SetKwData{PastDistribution}{pastDistribution}
% 	\SetKwData{FutureDistribution}{futureDistribution}
% 	\SetKwData{PastDistributionComp}{pastDistributionComp}
% 	\SetKwData{FutureDistributionComp}{futureDistributionComp}
	
% 	% FUNCTIONS SECTION
% 	% -----------------------
% 	\SetKwFunction{Subsets}{Subsets}
% 	\SetKwFunction{Length}{Length}
% 	\SetKwFunction{Extract}{Extract}
% 	\SetKwFunction{Part}{Part}
% 	\SetKwFunction{FullyConnectedQ}{FullyConnectedQ}
% 	\SetKwFunction{ComputesMIP}{ComputesMIP}
% 	\SetKwFunction{ComputesDistros}{ComputesDistros}
% 	\SetKwFunction{EMD}{EMD}
% 	\SetKwFunction{ComputesCEI}{ComputesCEI}
% 	\SetKwFunction{Complement}{ComplementI}
% 	\SetKwFunction{CPPD}{CPPD}
% 	\SetKwFunction{CFPD}{CFPD}
% 	\SetKwFunction{Normalize}{Normalize}
% 	\SetKwFunction{Min}{Min}
	
	
% 	\SetKwInOut{Input}{input}
% 	\SetKwInOut{Output}{output}
	
% 	% INPUTS-OUTPUTS SECTION
% 	% ------------------------------

% 	\Input{ChildMecha, ChildPurview, ParentMecha, ParentPurview, ParentPastDistro, ParentFutDistro, UnconstrainedPastDistro, UnconstrainedFutDistro}
% 	\Output{Causal and Effect information values}
% 	\BlankLine
	
% 	% CODE SECTION
% 	% -----------------------------
% 	\DontPrintSemicolon
	
% 	\MechaComplement$\leftarrow$ \Complement{$ChildMecha, ParentMecha$}\;
% 	\PurviewComplement$\leftarrow$ \Complement{$ChildPurview, ParentPurview$}\;
	
% 	\MechaChildren$\leftarrow$ \Subsets{$mechanism$}\;
% 	\PurviewChildren$\leftarrow$ \Subsets{$purview$}\;

	
% 	\ForEach{mecha  $m_i \in \MechaChildren$}{%
% 		\ForEach{purview $p_i \in \PurviewChildren$}{%
% 			ChildMecha $\leftarrow$  $m_i$\;
% 			ChildPurview $\leftarrow$  $p_i$\;
			
% 			\uIf{ChildMecha=$\emptyset$}{%
% 				\PastDistribution $\leftarrow$ UnconstrainedPastDistro\;
% 				\FutureDistribution $\leftarrow$ UnconstrainedFutDistro\;
				
% 				\uElseIf{ChildPurview=$\emptyset$}{%
% 					\PastDistribution $\leftarrow$ 1\;
% 					\FutureDistribution $\leftarrow$ 1\;
					
% 					\uElseIf{}{%
% 						\PastDistribution $\leftarrow$ \ComputesPastProbabilityDistribution{ChildMecha,ChildPurview,cm,am}\;
% 						\FutureDistribution $\leftarrow$ \ComputesFutureProbabilityDistribution{ChildMecha,ChildPurview,cm,am}\;\;
% 					}
					
% 				}
				
% 			}
		
		
		
% 			\uIf{\MechaComplement=$\emptyset$}{%
% 				\PastDistributionComp $\leftarrow$ UnconstrainedPastDistro\;
% 				\FutureDistributionComp $\leftarrow$ UnconstrainedFutDistro\;
				
% 				\uElseIf{\PurviewComplement=$\emptyset$}{%
% 					\PastDistributionComp $\leftarrow$ 1\;
% 					\FutureDistributionComp $\leftarrow$ 1\;
					
% 					\uElseIf{}{%
% 						\tcp*[h]{CPPD=ComputesPastProbabilityDistribution }\;
% 						\tcp*[h]{CFPD=ComputesFutureProbabilityDistribution}\;
% 						\PastDistributionComp $\leftarrow$ \CPPD{MechaComplement,PurviewComplement,cm,am}\;
% 						\FutureDistributionComp $\leftarrow$ \CFPD{MechaComplement,PurviewComplement,cm,am}\;
% 					}
					
% 				}
				
% 			}
		
		
% 			\PastDistribution $\leftarrow$ \Normalize{\PastDistribution*\PastDistributionComp}\;
% 			\FutureDistribution $\leftarrow$ \Normalize{\FutureDistribution*\FutureDistributionComp}\;
			
% 			\Ci $\leftarrow$ \EMD{ParentPastDistro,\PastDistribution}\;
% 			\Ei $\leftarrow$ \EMD{ParentFutDistro,\FutureDistribution}\;
% 			\Cei $\leftarrow$ \Min{\Ci,\Ei}\;
% 		}%\ForEach{purview $p_i
% 	}%\ForEach{mecha  $m_i
			
% 	\caption{computesCEI}\label{algo_computesCEI}
% \end{algorithm}
% \DecMargin{1em}




% % -------------------------------
% % THIS IS MY ALGORITHM 5: Position of patterns into the output repertoire
% % --------------------------------
% \IncMargin{1em}
% \begin{algorithm}[!htb]
	
% 	% VARIABLES SECTION
% 	% -------------------------
% 	\SetKwData{PurviewsSet}{purviewsSet}	
% 	\SetKwData{APurview}{aPurview}
% 	\SetKwData{APurviewMIP}{aPurviewMIP}
% 	\SetKwData{Distros}{distros}
% 	\SetKwData{Cs}{cs}
% 	\SetKwData{Connected}{connected}
% 	\SetKwData{MIP}{MIP}
% 	\SetKwData{SmallAlpha}{smallAlpha}
% 	\SetKwData{ConceptualInfo}{ConceptualInfo}
% 	\SetKwData{Mechanism}{mechanism}
% 	\SetKwData{Purview}{purview}
% 	\SetKwData{MechaChildren}{mechaChildren}
% 	\SetKwData{PurviewChildren}{PurviewChildren}
% 	\SetKwData{Intersec}{intersec}
% 	\SetKwData{JoinedNames}{joinedNames}
% 	\SetKwData{Ci}{ci}
% 	\SetKwData{Ei}{ei}
% 	\SetKwData{Cei}{cei}
% 	\SetKwData{PastMIP}{pastMIP}
% 	\SetKwData{FutMIP}{futMIP}
% 	\SetKwData{MechaComplement}{mechaComplement}
% 	\SetKwData{PurviewComplement}{purviewComplement}
% 	\SetKwData{Sumandos}{sumandos}
% 	\SetKwData{Powers}{powers}
% 	\SetKwData{Ins}{ins}
% 	\SetKwData{Repertoire}{repertoire}
% 	\SetKwData{Aux}{Aux}
% 	\SetKwData{Indexes}{indexes}
	
% 	% FUNCTIONS SECTION
% 	% -----------------------
% 	\SetKwFunction{Subsets}{Subsets}
% 	\SetKwFunction{Length}{Length}
% 	\SetKwFunction{Extract}{Extract}
% 	\SetKwFunction{Part}{Part}
% 	\SetKwFunction{FullyConnectedQ}{FullyConnectedQ}
% 	\SetKwFunction{ComputesMIP}{ComputesMIP}
% 	\SetKwFunction{ComputesDistros}{ComputesDistros}
% 	\SetKwFunction{EMD}{EMD}
% 	\SetKwFunction{ComputesCEI}{ComputesCEI}
% 	\SetKwFunction{Complement}{Complement}	
% 	\SetKwFunction{Range}{Range}
% 	\SetKwFunction{Append}{Append}	
% 	\SetKwFunction{Join}{Join}
% 	\SetKwFunction{Inputs}{Inputs}
% 	\SetKwFunction{FirstNode}{FirstNode}
% 	\SetKwFunction{RepertoireByOutput}{RepertoireByOutput}
% 	\SetKwFunction{Intersection}{Intersection}
% 	\SetKwFunction{CreateRepertoire}{CreateRepertoire}
% 	\SetKwFunction{Combine}{Combine}
% 	\SetKwFunction{FilterRepertoireByOutput}{FilterRepertoireByOutput}
% 	\SetKwFunction{Sum}{Sum}
	
	
	
% 	\SetKwInOut{Input}{input}
% 	\SetKwInOut{Output}{output}
	
	
% 	% INPUTS-OUTPUTS SECTION
% 	% ------------------------------
% 	\Input{Mechanism,Purview,AdjacencyMatrix,CurrentState,Dynamic}
% 	\Output{positions (indexes) where current state of mechanism is found into the output repertoire}
% 	\BlankLine
	
% 	% CODE SECTION
% 	% -----------------------------
% 	\DontPrintSemicolon
	
% 	\tcp*[h]{work as nodes that send inputs to the mechanism. }\;
% 	\tcp*[h]{This remaning nodes are actually powers that define a pattern }\;
% 	\tcp*[h]{of distribution of the wanted pattern defined by mechanism}\;
%     \BlankLine
% 	\JoinedNames$\leftarrow$ \Join{\Inputs{$Mechanism$}}\;
% 	\Powers$\leftarrow$ \Complement{\Range{\Length{AdjacencyMatrix}}, \JoinedNames}-1\;
% 	\ForEach{node $n_i \in \Powers$}{%
% 		$\Append(\Sumandos,2^{n_i})$\;
% 	}
% 	\Sumandos$\leftarrow$\Subsets{\Sumandos}\;
% 	\ForEach{sumando $s_i \in \Sumandos$}{%
% 		$\Append(\Aux,\Sum(s_i))$\;
% 	}
% 	\Sumandos$\leftarrow$\Aux\;
	
% 	\BlankLine
% 	\BlankLine
% 	\BlankLine
% 	\Ins$\leftarrow$ \Inputs{\FirstNode{Mechanism}}\;
% 	\tcp*[h]{Given a dynamic it computes all possible inputs of defined size that}\;
% 	\tcp*[h]{results in a defined output (cs)}\;
	
% 	\Repertoire$\leftarrow$ \RepertoireByOutput{\Length{\Ins},Dynamic,\Cs}\;
	
	
% 	\ForEach{node $m_i \in (\Mechanism-\FirstNode{\Mechanism})$}{%
% 		\Intersec$\leftarrow$ \Intersection{\Ins,\Inputs{$m_i$}}\;
% 		\Ins$\leftarrow$ \Ins+(\Inputs{$m_i$}-\Intersec)\;
% 		\Repertoire$\leftarrow$\Combine{\Repertoire,CreateRepertoire(\Inputs{$m_i$}-\Intersec)}\;
% 		\Repertoire$\leftarrow$\FilterRepertoireByOutput{\Repertoire,\Cs}\;
% 	}


% 	\ForEach{sumando $s_i \in \Sumandos$}{%
% 		\ForEach{repert $r_i \in \Repertoire$}{%
% 			$\Append(\Indexes,\Sum(s_i,r_i))$\;
% 		}	
% 	}


	
% 	\caption{computesPositionsOfAPatternInOutputs}\label{algo_computesPositionsOfAPatternInOutputs}
% \end{algorithm}
% \DecMargin{1em}





% % -------------------------------
% % THIS IS MY ALGORITHM 6: Compute past probabity distribution
% % --------------------------------
% \IncMargin{1em}
% \begin{algorithm}[!htb]
	
% 	% VARIABLES SECTION
% 	% -------------------------
% 	\SetKwData{PurviewsSet}{purviewsSet}	
% 	\SetKwData{APurview}{aPurview}
% 	\SetKwData{APurviewMIP}{aPurviewMIP}
% 	\SetKwData{Distros}{distros}
% 	\SetKwData{Cs}{cs}
% 	\SetKwData{Connected}{connected}
% 	\SetKwData{MIP}{MIP}
% 	\SetKwData{SmallAlpha}{smallAlpha}
% 	\SetKwData{ConceptualInfo}{ConceptualInfo}
% 	\SetKwData{Mechanism}{mechanism}
% 	\SetKwData{Purview}{purview}
% 	\SetKwData{MechaChildren}{mechaChildren}
% 	\SetKwData{PurviewChildren}{PurviewChildren}
% 	\SetKwData{Intersec}{intersec}
% 	\SetKwData{JoinedNames}{joinedNames}
% 	\SetKwData{Ci}{ci}
% 	\SetKwData{Ei}{ei}
% 	\SetKwData{Cei}{cei}
% 	\SetKwData{PastMIP}{pastMIP}
% 	\SetKwData{FutMIP}{futMIP}
% 	\SetKwData{MechaComplement}{mechaComplement}
% 	\SetKwData{PurviewComplement}{purviewComplement}
% 	\SetKwData{Sumandos}{sumandos}
% 	\SetKwData{Powers}{powers}
% 	\SetKwData{Ins}{ins}
% 	\SetKwData{Repertoire}{repertoire}
% 	\SetKwData{Aux}{Aux}
% 	\SetKwData{Indexes}{indexes}
% 	\SetKwData{Locations}{locations}
% 	\SetKwData{CorrectedLocations}{correctedLocations}
% 	\SetKwData{Probability}{probability}
% 	\SetKwData{AllInputs}{allInputs}
% 	\SetKwData{ProbabilityDistribution}{probabilityDistribution}
	
% 	% FUNCTIONS SECTION
% 	% -----------------------
% 	\SetKwFunction{Subsets}{Subsets}
% 	\SetKwFunction{Length}{Length}
% 	\SetKwFunction{Extract}{Extract}
% 	\SetKwFunction{Part}{Part}
% 	\SetKwFunction{FullyConnectedQ}{FullyConnectedQ}
% 	\SetKwFunction{ComputesMIP}{ComputesMIP}
% 	\SetKwFunction{ComputesDistros}{ComputesDistros}
% 	\SetKwFunction{EMD}{EMD}
% 	\SetKwFunction{ComputesCEI}{ComputesCEI}
% 	\SetKwFunction{Complement}{Complement}	
% 	\SetKwFunction{Range}{Range}
% 	\SetKwFunction{Append}{Append}	
% 	\SetKwFunction{Join}{Join}
% 	\SetKwFunction{Inputs}{Inputs}
% 	\SetKwFunction{FirstNode}{FirstNode}
% 	\SetKwFunction{RepertoireByOutput}{RepertoireByOutput}
% 	\SetKwFunction{Intersection}{Intersection}
% 	\SetKwFunction{CreateRepertoire}{CreateRepertoire}
% 	\SetKwFunction{Combine}{Combine}
% 	\SetKwFunction{FilterRepertoireByOutput}{FilterRepertoireByOutput}
% 	\SetKwFunction{Sum}{Sum}
% 	\SetKwFunction{ComputesPositionsOfAPattern}{computesPositionsOfAPattern}
% 	\SetKwFunction{FindPatternInInputs}{FindPatternInInputs}
% 	\SetKwFunction{Extract}{Extract}
% 	\SetKwFunction{ComputesProbabilityForElements}{ComputesProbabilityForElements}
	
	
	
	
% 	\SetKwInOut{Input}{input}
% 	\SetKwInOut{Output}{output}
	
	
% 	% INPUTS-OUTPUTS SECTION
% 	% ------------------------------

% 	\Input{Mechanism,Purview,AdjacencyMatrix,CurrentState,Dynamic}
% 	\Output{Probability distribution for a mechanism}
% 	\BlankLine
	
% 	% CODE SECTION
% 	% -----------------------------
% 	\DontPrintSemicolon
	
% 	\BlankLine
% 	\Locations$\leftarrow$\ComputesPositionsOfAPattern{$Mechanism,CurrentState,Purview,AdjacencyMatrix$}\;
% 	\AllInputs $\leftarrow$ \Extract{\Locations,Purview}\;
% 	\Probability$\leftarrow$ 1/(\Length{Locations})\;
	
% 	\ForEach{input $in_i \in \AllInputs$}{%
% 		\CorrectedLocations$\leftarrow$\FindPatternInInputs{$Purview,in_i,\Length{AdjacencyMatrix}$}
% 	}
	
% 	\ProbabilityDistribution $\leftarrow$ \ComputesProbabilityForElements{\CorrectedLocations}
	
	
	
% 	\caption{computesPastProbabilityDistribution}\label{algo_computesPastProbDistribution}
% \end{algorithm}
% \DecMargin{1em}




% % -------------------------------
% % THIS IS MY ALGORITHM 7: Compute future (effect) probabity distribution
% % --------------------------------
% \IncMargin{1em}
% \begin{algorithm}[!htb]
	
% 	% VARIABLES SECTION
% 	% -------------------------
% 	\SetKwData{PurviewsSet}{purviewsSet}	
% 	\SetKwData{APurview}{aPurview}
% 	\SetKwData{APurviewMIP}{aPurviewMIP}
% 	\SetKwData{Distros}{distros}
% 	\SetKwData{Cs}{cs}
% 	\SetKwData{Connected}{connected}
% 	\SetKwData{MIP}{MIP}
% 	\SetKwData{SmallAlpha}{smallAlpha}
% 	\SetKwData{ConceptualInfo}{ConceptualInfo}
% 	\SetKwData{Mechanism}{mechanism}
% 	\SetKwData{Purview}{purview}
% 	\SetKwData{MechaChildren}{mechaChildren}
% 	\SetKwData{PurviewChildren}{PurviewChildren}
% 	\SetKwData{Intersec}{intersec}
% 	\SetKwData{JoinedNames}{joinedNames}
% 	\SetKwData{Ci}{ci}
% 	\SetKwData{Ei}{ei}
% 	\SetKwData{Cei}{cei}
% 	\SetKwData{PastMIP}{pastMIP}
% 	\SetKwData{FutMIP}{futMIP}
% 	\SetKwData{MechaComplement}{mechaComplement}
% 	\SetKwData{PurviewComplement}{purviewComplement}
% 	\SetKwData{Sumandos}{sumandos}
% 	\SetKwData{Powers}{powers}
% 	\SetKwData{Ins}{ins}
% 	\SetKwData{Repertoire}{repertoire}
% 	\SetKwData{Aux}{Aux}
% 	\SetKwData{Indexes}{indexes}
% 	\SetKwData{Locations}{locations}
% 	\SetKwData{CorrectedLocations}{correctedLocations}
% 	\SetKwData{Probability}{probability}
% 	\SetKwData{AllOutputs}{allOutputs}
% 	\SetKwData{AllInputs}{allInputs}
% 	\SetKwData{ProbabilityDistribution}{probabilityDistribution}
	
% 	% FUNCTIONS SECTION
% 	% -----------------------
% 	\SetKwFunction{Subsets}{Subsets}
% 	\SetKwFunction{Length}{Length}
% 	\SetKwFunction{Extract}{Extract}
% 	\SetKwFunction{Part}{Part}
% 	\SetKwFunction{FullyConnectedQ}{FullyConnectedQ}
% 	\SetKwFunction{ComputesMIP}{ComputesMIP}
% 	\SetKwFunction{ComputesDistros}{ComputesDistros}
% 	\SetKwFunction{EMD}{EMD}
% 	\SetKwFunction{ComputesCEI}{ComputesCEI}
% 	\SetKwFunction{Complement}{Complement}	
% 	\SetKwFunction{Range}{Range}
% 	\SetKwFunction{Append}{Append}	
% 	\SetKwFunction{Join}{Join}
% 	\SetKwFunction{Inputs}{Inputs}
% 	\SetKwFunction{FirstNode}{FirstNode}
% 	\SetKwFunction{RepertoireByOutput}{RepertoireByOutput}
% 	\SetKwFunction{Intersection}{Intersection}
% 	\SetKwFunction{CreateRepertoire}{CreateRepertoire}
% 	\SetKwFunction{Combine}{Combine}
% 	\SetKwFunction{FilterRepertoireByOutput}{FilterRepertoireByOutput}
% 	\SetKwFunction{Sum}{Sum}
% 	\SetKwFunction{ComputesPositionsOfAPattern}{computesPositionsOfAPattern}
% 	\SetKwFunction{FindPatternInInputs}{FindPatternInInputs}
% 	\SetKwFunction{Extract}{Extract}
% 	\SetKwFunction{ComputesProbabilityForElements}{ComputesProbabilityForElements}
% 	\SetKwFunction{ComputesOutputs}{ComputesOutputs}
	
	
	
	
% 	\SetKwInOut{Input}{input}
% 	\SetKwInOut{Output}{output}
	
	
% 	% INPUTS-OUTPUTS SECTION
% 	% ------------------------------
	
% 	\Input{Mechanism,Purview,AdjacencyMatrix,CurrentState,Dynamic}
% 	\Output{Probability distribution for a mechanism}
% 	\BlankLine
	
% 	% CODE SECTION
% 	% -----------------------------
% 	\DontPrintSemicolon
	
% 	\BlankLine
% 	\Locations$\leftarrow$(\FindPatternInInputs{$Purview,CurrentState,\Length{AdjacencyMatrix}$})-1\;
% 	\AllOutputs $\leftarrow$ \ComputesOutputs{\Locations}\;
% 	\AllInputs $\leftarrow$ \Extract{\Locations,Purview}\;

% 	\ProbabilityDistribution $\leftarrow$ \ComputesProbabilityForElements{\AllInputs}
	
	
	
% 	\caption{computesFutureProbabilityDistribution}\label{algo_computesFutureProbDistribution}
% \end{algorithm}
% \DecMargin{1em}









% % -------------------------------
% % THIS IS MY ALGORITHM 8: Find patterns in inputs
% % --------------------------------
% \IncMargin{1em}
% \begin{algorithm}[!htb]
	
% 	% VARIABLES SECTION
% 	% -------------------------
% 	\SetKwData{PurviewsSet}{purviewsSet}	
% 	\SetKwData{APurview}{aPurview}
% 	\SetKwData{APurviewMIP}{aPurviewMIP}
% 	\SetKwData{Distros}{distros}
% 	\SetKwData{Cs}{cs}
% 	\SetKwData{Connected}{connected}
% 	\SetKwData{MIP}{MIP}
% 	\SetKwData{SmallAlpha}{smallAlpha}
% 	\SetKwData{ConceptualInfo}{ConceptualInfo}
% 	\SetKwData{Mechanism}{mechanism}
% 	\SetKwData{Purview}{purview}
% 	\SetKwData{MechaChildren}{mechaChildren}
% 	\SetKwData{PurviewChildren}{PurviewChildren}
% 	\SetKwData{Intersec}{intersec}
% 	\SetKwData{JoinedNames}{joinedNames}
% 	\SetKwData{Ci}{ci}
% 	\SetKwData{Ei}{ei}
% 	\SetKwData{Cei}{cei}
% 	\SetKwData{PastMIP}{pastMIP}
% 	\SetKwData{FutMIP}{futMIP}
% 	\SetKwData{MechaComplement}{mechaComplement}
% 	\SetKwData{PurviewComplement}{purviewComplement}
% 	\SetKwData{Sumandos}{sumandos}
% 	\SetKwData{Powers}{powers}
% 	\SetKwData{Ins}{ins}
% 	\SetKwData{Repertoire}{repertoire}
% 	\SetKwData{Aux}{Aux}
% 	\SetKwData{Indexes}{indexes}
% 	\SetKwData{Locations}{locations}
% 	\SetKwData{Probability}{probability}
% 	\SetKwData{Limit}{limit}
% 	\SetKwData{Repetitions}{repetitions}
% 	\SetKwData{Longi}{longi}
% 	\SetKwData{Serie}{serie}
% 	\SetKwData{Found}{Found}
	
	
% 	% FUNCTIONS SECTION
% 	% -----------------------
% 	\SetKwFunction{Subsets}{Subsets}
% 	\SetKwFunction{Length}{Length}
% 	\SetKwFunction{Extract}{Extract}
% 	\SetKwFunction{Part}{Part}
% 	\SetKwFunction{FullyConnectedQ}{FullyConnectedQ}
% 	\SetKwFunction{ComputesMIP}{ComputesMIP}
% 	\SetKwFunction{ComputesDistros}{ComputesDistros}
% 	\SetKwFunction{EMD}{EMD}
% 	\SetKwFunction{ComputesCEI}{ComputesCEI}
% 	\SetKwFunction{Complement}{Complement}	
% 	\SetKwFunction{Range}{Range}
% 	\SetKwFunction{Append}{Append}	
% 	\SetKwFunction{Join}{Join}
% 	\SetKwFunction{Inputs}{Inputs}
% 	\SetKwFunction{FirstNode}{FirstNode}
% 	\SetKwFunction{RepertoireByOutput}{RepertoireByOutput}
% 	\SetKwFunction{Intersection}{Intersection}
% 	\SetKwFunction{CreateRepertoire}{CreateRepertoire}
% 	\SetKwFunction{Combine}{Combine}
% 	\SetKwFunction{FilterRepertoireByOutput}{FilterRepertoireByOutput}
% 	\SetKwFunction{Sum}{Sum}
% 	\SetKwFunction{ComputesPositionsOfAPattern}{computesPositionsOfAPattern}
% 	\SetKwFunction{CreateSerieOddNumbers}{CreateSerieOddNumbers}
% 	\SetKwFunction{CreateSerieEvenNumbers}{CreateSerieEvenNumbers}
	
		
	
% 	\SetKwInOut{Input}{input}
% 	\SetKwInOut{Output}{output}
	
	
% 	% INPUTS-OUTPUTS SECTION
% 	% ------------------------------
	
% 	\Input{Nodes,WantedPattern,sizeAdjacencyMatrix}
% 	\Output{Finds indexes in input repertoire where nodes fullfill wantedPattern}
% 	\BlankLine
	
% 	% CODE SECTION
% 	% -----------------------------
% 	\DontPrintSemicolon
	
% 	$\Limit \leftarrow 2^{sizeAdjacencyMatrix}$\;
	
% 	\ForEach{node $n_i \in Nodes$}{%
% 			$\Powers \leftarrow 2^{n_i-1}$\;
% 			$\Repetitions \leftarrow \Limit/\Powers$\;
% 			$\Longi \leftarrow \Limit/\Repetitions$\;
			
% 			\uIf{if expectedPatt = 1}{
% 				$\Serie \leftarrow \CreateSerieEvenNumbers(\Repetitions)$\;
% 			}
% 			\Else{
% 				$\Serie \leftarrow \CreateSerieOddNumbers(\Repetitions)$\;
% 			}
			
% 			\For{$i = 1;\ i < \Length{\Serie};\ i = i + 1$}{
% 				$\Found \leftarrow \Range{((\Powers*\Serie[i])-\Longi)+1, \Powers*\Serie[i]}$\;
% 			}
	
	        
			
% 	}
	
	
	
% 	\caption{findPatternInInputs}\label{algo_findPatternsInInputs}
% \end{algorithm}
% \DecMargin{1em}





% % -------------------------------
% % THIS IS MY ALGORITHM 9: Computes Bit probability for inputs
% % --------------------------------
% \IncMargin{1em}
% \begin{algorithm}[!htb]
	
% 	% VARIABLES SECTION
% 	% -------------------------
% 	\SetKwData{PurviewsSet}{purviewsSet}	
% 	\SetKwData{APurview}{aPurview}
% 	\SetKwData{APurviewMIP}{aPurviewMIP}
% 	\SetKwData{Distros}{distros}
% 	\SetKwData{Cs}{cs}
% 	\SetKwData{Connected}{connected}
% 	\SetKwData{MIP}{MIP}
% 	\SetKwData{SmallAlpha}{smallAlpha}
% 	\SetKwData{ConceptualInfo}{ConceptualInfo}
% 	\SetKwData{Mechanism}{mechanism}
% 	\SetKwData{Purview}{purview}
% 	\SetKwData{MechaChildren}{mechaChildren}
% 	\SetKwData{PurviewChildren}{PurviewChildren}
% 	\SetKwData{Intersec}{intersec}
% 	\SetKwData{JoinedNames}{joinedNames}
% 	\SetKwData{Ci}{ci}
% 	\SetKwData{Ei}{ei}
% 	\SetKwData{Cei}{cei}
% 	\SetKwData{PastMIP}{pastMIP}
% 	\SetKwData{FutMIP}{futMIP}
% 	\SetKwData{MechaComplement}{mechaComplement}
% 	\SetKwData{PurviewComplement}{purviewComplement}
% 	\SetKwData{Sumandos}{sumandos}
% 	\SetKwData{Powers}{powers}
% 	\SetKwData{Ins}{ins}
% 	\SetKwData{Repertoire}{repertoire}
% 	\SetKwData{Aux}{Aux}
% 	\SetKwData{Indexes}{indexes}
% 	\SetKwData{Locations}{locations}
% 	\SetKwData{Probability}{probability}
% 	\SetKwData{Limit}{limit}
% 	\SetKwData{Repetitions}{repetitions}
% 	\SetKwData{Longi}{longi}
% 	\SetKwData{Serie}{serie}
% 	\SetKwData{Found}{Found}
% 	\SetKwData{ZeroProbability}{zeroProbability}
% 	\SetKwData{OneProbability}{oneProbability}
	
	
% 	% FUNCTIONS SECTION
% 	% -----------------------
% 	\SetKwFunction{Subsets}{Subsets}
% 	\SetKwFunction{Length}{Length}
% 	\SetKwFunction{Extract}{Extract}
% 	\SetKwFunction{Part}{Part}
% 	\SetKwFunction{FullyConnectedQ}{FullyConnectedQ}
% 	\SetKwFunction{ComputesMIP}{ComputesMIP}
% 	\SetKwFunction{ComputesDistros}{ComputesDistros}
% 	\SetKwFunction{EMD}{EMD}
% 	\SetKwFunction{ComputesCEI}{ComputesCEI}
% 	\SetKwFunction{Complement}{Complement}	
% 	\SetKwFunction{Range}{Range}
% 	\SetKwFunction{Append}{Append}	
% 	\SetKwFunction{Join}{Join}
% 	\SetKwFunction{Inputs}{Inputs}
% 	\SetKwFunction{FirstNode}{FirstNode}
% 	\SetKwFunction{RepertoireByOutput}{RepertoireByOutput}
% 	\SetKwFunction{Intersection}{Intersection}
% 	\SetKwFunction{CreateRepertoire}{CreateRepertoire}
% 	\SetKwFunction{Combine}{Combine}
% 	\SetKwFunction{FilterRepertoireByOutput}{FilterRepertoireByOutput}
% 	\SetKwFunction{Sum}{Sum}
% 	\SetKwFunction{ComputesPositionsOfAPattern}{computesPositionsOfAPattern}
% 	\SetKwFunction{CreateSerieOddNumbers}{CreateSerieOddNumbers}
% 	\SetKwFunction{CreateSerieEvenNumbers}{CreateSerieEvenNumbers}
% 	\SetKwFunction{FindPatternInInputs}{FindPatternInInputs}
	
	
	
% 	\SetKwInOut{Input}{input}
% 	\SetKwInOut{Output}{output}
	
	
% 	% INPUTS-OUTPUTS SECTION
% 	% ------------------------------
	
% 	\Input{Nodes,AdjacencyMatrix,Dynamics}
% 	\Output{Bit probability for given nodes}
% 	\BlankLine
	
% 	% CODE SECTION
% 	% -----------------------------
% 	\DontPrintSemicolon
	
% 	$\Locations \leftarrow \FindPatternInInputs{Nodes,1,\Length{AdjacencyMatrix}}$\;
% 	$\OneProbability \leftarrow \Length{\Locations}/2^{\Length{AdjacencyMatrix}}$\;
% 	$\ZeroProbability \leftarrow 1-\OneProbability$\;
		
	
% 	\caption{computeInputBitProbabilityDistro}\label{algo_computeInputBitProbabilityDistro}
% \end{algorithm}
% \DecMargin{1em}






% % -------------------------------
% % THIS IS MY ALGORITHM 10: Computes Bit probability for outputs
% % --------------------------------
% \IncMargin{1em}
% \begin{algorithm}[!htb]
	
% 	% VARIABLES SECTION
% 	% -------------------------
% 	\SetKwData{PurviewsSet}{purviewsSet}	
% 	\SetKwData{APurview}{aPurview}
% 	\SetKwData{APurviewMIP}{aPurviewMIP}
% 	\SetKwData{Distros}{distros}
% 	\SetKwData{Cs}{cs}
% 	\SetKwData{Connected}{connected}
% 	\SetKwData{MIP}{MIP}
% 	\SetKwData{SmallAlpha}{smallAlpha}
% 	\SetKwData{ConceptualInfo}{ConceptualInfo}
% 	\SetKwData{Mechanism}{mechanism}
% 	\SetKwData{Purview}{purview}
% 	\SetKwData{MechaChildren}{mechaChildren}
% 	\SetKwData{PurviewChildren}{PurviewChildren}
% 	\SetKwData{Intersec}{intersec}
% 	\SetKwData{JoinedNames}{joinedNames}
% 	\SetKwData{Ci}{ci}
% 	\SetKwData{Ei}{ei}
% 	\SetKwData{Cei}{cei}
% 	\SetKwData{PastMIP}{pastMIP}
% 	\SetKwData{FutMIP}{futMIP}
% 	\SetKwData{MechaComplement}{mechaComplement}
% 	\SetKwData{PurviewComplement}{purviewComplement}
% 	\SetKwData{Sumandos}{sumandos}
% 	\SetKwData{Powers}{powers}
% 	\SetKwData{Ins}{ins}
% 	\SetKwData{Repertoire}{repertoire}
% 	\SetKwData{Aux}{Aux}
% 	\SetKwData{Indexes}{indexes}
% 	\SetKwData{Locations}{locations}
% 	\SetKwData{Probability}{probability}
% 	\SetKwData{Limit}{limit}
% 	\SetKwData{Repetitions}{repetitions}
% 	\SetKwData{Longi}{longi}
% 	\SetKwData{Serie}{serie}
% 	\SetKwData{Found}{Found}
% 	\SetKwData{ZeroProbability}{zeroProbability}
% 	\SetKwData{OneProbability}{oneProbability}
	
	
% 	% FUNCTIONS SECTION
% 	% -----------------------
% 	\SetKwFunction{Subsets}{Subsets}
% 	\SetKwFunction{Length}{Length}
% 	\SetKwFunction{Extract}{Extract}
% 	\SetKwFunction{Part}{Part}
% 	\SetKwFunction{FullyConnectedQ}{FullyConnectedQ}
% 	\SetKwFunction{ComputesMIP}{ComputesMIP}
% 	\SetKwFunction{ComputesDistros}{ComputesDistros}
% 	\SetKwFunction{EMD}{EMD}
% 	\SetKwFunction{ComputesCEI}{ComputesCEI}
% 	\SetKwFunction{Complement}{Complement}	
% 	\SetKwFunction{Range}{Range}
% 	\SetKwFunction{Append}{Append}	
% 	\SetKwFunction{Join}{Join}
% 	\SetKwFunction{Inputs}{Inputs}
% 	\SetKwFunction{FirstNode}{FirstNode}
% 	\SetKwFunction{RepertoireByOutput}{RepertoireByOutput}
% 	\SetKwFunction{Intersection}{Intersection}
% 	\SetKwFunction{CreateRepertoire}{CreateRepertoire}
% 	\SetKwFunction{Combine}{Combine}
% 	\SetKwFunction{FilterRepertoireByOutput}{FilterRepertoireByOutput}
% 	\SetKwFunction{Sum}{Sum}
% 	\SetKwFunction{ComputesPositionsOfAPattern}{computesPositionsOfAPattern}
% 	\SetKwFunction{CreateSerieOddNumbers}{CreateSerieOddNumbers}
% 	\SetKwFunction{CreateSerieEvenNumbers}{CreateSerieEvenNumbers}
% 	\SetKwFunction{ComputesPositionsOfAPatternInOutputs}{ComputesPositionsOfAPatternInOutputs}
	
	
	
% 	\SetKwInOut{Input}{input}
% 	\SetKwInOut{Output}{output}
	
	
% 	% INPUTS-OUTPUTS SECTION
% 	% ------------------------------
	
% 	\Input{Nodes,AdjacencyMatrix,Dynamics}
% 	\Output{Bit probability for given nodes}
% 	\BlankLine
	
% 	% CODE SECTION
% 	% -----------------------------
% 	\DontPrintSemicolon
	
% 	$\Locations \leftarrow \ComputesPositionsOfAPatternInOutputs{Nodes,1,Dynamics,AdjacencyMatrix}$\;
% 	$\OneProbability \leftarrow \Length{\Locations}/2^{\Length{AdjacencyMatrix}}$\;
% 	$\ZeroProbability \leftarrow 1-\OneProbability$\;
	
	
% 	\caption{computeOutputBitProbabilityDistro}\label{algo_computeOutputBitProbabilityDistro}
% \end{algorithm}
% \DecMargin{1em}

% %%%%%%%%%%%%%%%%%%%%%%%%%%%%%%%%%
% % PSEUDO CODE SECTION    --END--
% %%%%%%%%%%%%%%%%%%%%%%%%%%%%%%%%%
               % Colocar los circuitos, manuales, código fuente, pruebas de teoremas, etc.
\include{Apendice2/Apendice2}               % Colocar los circuitos, manuales, código fuente, pruebas de teoremas, etc.
\chapter{Meta-compression}

% -------------------------------
% THIS IS MY ALGORITHM: Generate Network Data and Export to CSV
% --------------------------------
\IncMargin{1em}
\begin{algorithm}[H]
    \small % Reduce font size
    \setstretch{0.8} % Reduce line spacing
    \resizebox{\textwidth}{!}{% Scale the algorithm to fit the page width
    \begin{minipage}{\textwidth} % Ensure content stays within bounds
    \SetKwInOut{Input}{Input}
    \SetKwInOut{Output}{Output}

    \Input{maxNodes}
    \Output{CSV file with network data}

    \BlankLine

    \textbf{Initialize} networks as an empty list\;
    \If{maxNodes is not provided or is not numeric}{
        Set maxNodes to 12\;
        Print "maxNodes set to default: 12"\;
    }

    \ForEach{nodeSize $\in$ \{5, \ldots, maxNodes\}}{
        Set nodeSize to current value of n\;
        Print "Processing nodeSize: " + nodeSize\;
        Calculate noEdges as a random integer between $\lfloor\frac{\text{nodeSize}}{2}\rfloor$ and $\lfloor\frac{\text{nodeSize} \times (\text{nodeSize}-1)}{2}\rfloor$\;

        \textbf{Generate a random network} with nodeSize nodes and noEdges edges\;

        Calculate initial complexity (k0) and length (a0) of binary attractors\;
        Initialize sensitivities as an empty list\;

        \ForEach{edge (i, j) in the adjacency matrix}{
            \If{adjMat[i, j] == 1}{
                Set adjMatPert[i, j] to 0\;
                Recalculate patterns and binary attractors with the perturbed adjacency matrix\;
                \If{the perturbed binary attractors are valid}{
                    Calculate new complexity (k1) and length (a1)\;
                    Calculate deltaK as $|k0 - k1|$, deltaA as $|a0 - a1|$\;
                    Add deltaK $\times \log_2(\text{deltaA} + 1)$ to sensitivities\;
                }
            }
        }

        Calculate phiK as the mean of sensitivities if sensitivities is not empty, otherwise 0\;

        Calculate rules0 as behaviors for each attractor in binAttractors\;
        Calculate kRule0 as the mean byte count of rules0\;
        Calculate rules1 as behaviors for each attractor in binAttPert with the perturbed adjacency matrix\;
        Calculate kRule1 as the mean byte count of rules1\;
        Calculate deltaKRule as $|\text{kRule0} - \text{kRule1}|$\;
        Calculate phiKRefined as phiK + deltaKRule\;
    }

    Export networks to a CSV file with columns: "NumberOfNodes", "OutputRepertoire", "AdjacencyMatrix", "OneOutput", "PhiK"\;

    \caption{Generate Network Data and Export to CSV}\label{algo-fullMetacompression}
    \end{minipage}
    }
\end{algorithm}
\DecMargin{1em}
  

%%%%%%%%%%%%%%%%%%%%%%%%%%%%%%%%%%%%%%%%%%%%%%%%%%%%%
%                   REFERENCIAS                     %
%%%%%%%%%%%%%%%%%%%%%%%%%%%%%%%%%%%%%%%%%%%%%%%%%%%%%
% existen varios estilos de bilbiografía, pueden cambiarlos a placer
%El formato trae otros estilos, o pueden agregar uno que les guste:
%\bibliographystyle{Latex/Classes/PhDbiblio-case} % title forced lower case
%\bibliographystyle{Latex/Classes/PhDbiblio-bold} % title as in bibtex but bold
%\bibliographystyle{Latex/Classes/PhDbiblio-url} % bold + www link if provided
%\bibliographystyle{Latex/Classes/jmb} % calls style file jmb.bst

\bibliographystyle{apalike} % otros estilos pueden ser abbrv, acm, alpha, apalike, ieeetr, plain, siam, unsrt
%\bibliographystyle{plainnat} % or another style like 'unsrt', 'alpha', etc.
\bibliography{Bibliografia/referencias}             % Archivo .bib


\end{document}
