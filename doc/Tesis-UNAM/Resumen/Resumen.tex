
% Thesis Abstract -----------------------------------------------------


%\begin{abstractslong}    %uncommenting this line, gives a different abstract heading
\begin{abstracts}        %this creates the heading for the abstract page

    Integrated information has been introduced as a metric to quantify the amount of 
    information generated by a system beyond the information generated by its individual elements. 
    While the metrics associated with the Greek letter $\phi$ require the calculation of the 
    interaction of an exponential number of sub-divisions of the system, most of these numerical 
    approaches related to the metric are based on the basics of classical information theory 
    and perturbation analysis. Here we introduce and sketch alternative approaches to connect 
    algorithmic complexity and integrated information based on the concept of algorithmic 
    perturbation rooted in algorithmic information dynamics and its concept of programmability. 
    We hypothesise that if an object is algorithmic random or algorithmic simple, algorithmic 
    random perturbations will have little to no effect to the internal capabilities of a system 
    to produce integrated information but when an object is more integrated the object will 
    also display elements able to perturb the object and increase or decrease its algorithmic 
    randomness. We sketch some of these ideas related to an object integrated information 
    value and its algorithmic information content. We propose that such an algorithmic 
    perturbation test quantifying compression sensitivity may provide a system with a means 
    to extract explanations--causal accounts--of its own behaviour hence making IIT and associated 
    measure $\phi$ more explainable and interpretable. Our technique may reduce the number of 
    calculations to arrive at some estimations with algorithmic perturbation guiding a more 
    efficient search. Our work sets the stage for a systematic exploration and further 
    investigation of the connections between algorithmic complexity and integrated information 
    at the level of both theory and practice.
% \blindtext

\end{abstracts}
%\end{abstractlongs}


% ----------------------------------------------------------------------